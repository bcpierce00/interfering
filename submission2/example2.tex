\begin{figure}[b]
    \begin{minipage}{.35\textwidth}
\begin{verbatim}
int main() {
  int x = 42, y = 0;
  swap(&x,&y);
  return y;
}
\end{verbatim}
    \end{minipage}
    \begin{minipage}{.35\textwidth}
\begin{verbatim}
void swap(int* a, int* b) {
  int tmp = *a;
  *a = *b;
  *b = tmp;
  print g();
  return (z+z);
}
\end{verbatim}
    \end{minipage}
    \begin{minipage}{.25\textwidth}
\begin{verbatim}
int g() {
  return 5;
}
\end{verbatim}
    \end{minipage}
\caption{A Simple Call}
\label{fig:share-program}
\end{figure}

\newcommand{\mainpassc}{magenta}
\newcommand{\sharec}{lgray}

\begin{figure}

\begin{center}
\begin{tabular}{l l}
Entry to {\tt main}
 &
\memoryaddrs{25em}
\memory{3}{\instrc}[{\makebox[0pt]{Instructions}}]%
\hspace*{3pt}
$\cdots$
\memory{1}{\emptyoutc}[{\makebox[0pt]{Output}}]%
\memory{8}{\unsealc}[{\makebox[0pt]{Unsealed}}]
~$\cdots$
\\ \\
Call to {\tt f}
 &
\memoryaddrs{52em}
\memory{3}{\instrc}%
\hspace*{3pt}
$\cdots$
\memory{1}{\emptyoutc}%
\memory{2}{\unsealc}[{\makebox[0pt]{Unsealed}}]%
\memory{2}{\sharec}[{\makebox[0pt]{Shared}}]%
\memory{2}{\mainpassc}[{\makebox[0pt]{Passed(0)}}]%
\memory{2}{\unsealc}[{\makebox[0pt]{Unsealed}}]
~$\cdots$
\MemoryLabel{-37.5em}{0.75em}{\(\ldots\)}
\MemoryLabel{-33.5em}{0.75em}{42}
\MemoryLabel{-29.5em}{0.75em}{\(\ldots\)}
\MemoryLabel{-25.5em}{0.75em}{0}
\MemoryLabel{-21.5em}{0.75em}{\(\ldots\)}
\MemoryLabel{-17.5em}{0.75em}{42}
\\ \\
Entry to {\tt f} &
\memoryaddrs{52em}
\memory{3}{\instrc}%
\hspace*{3pt}
$\cdots$
\memory{1}{\emptyoutc}%
\memory{2}{\mainsealc}[{\makebox[0pt]{Sealed(0)}}]%
\memory{2}{\sharec}[{\makebox[0pt]{Shared}}]%
\memory{2}{\mainpassc}[{\makebox[0pt]{Passed(0)}}]%
\memory{2}{\unsealc}[{\makebox[0pt]{Unsealed}}]
~$\cdots$
\MemoryLabel{-29.5em}{0.75em}{\(\ldots\)}
\MemoryLabel{-25.5em}{0.75em}{0}
\MemoryLabel{-21.5em}{0.75em}{\(\ldots\)}
\MemoryLabel{-17.5em}{0.75em}{42}
\\ \\
Entry to {\tt g} &
\memoryaddrs{61em}
\memory{3}{\instrc}%
\hspace*{3pt}
$\cdots$
\memory{1}{\emptyoutc}%
\memory{2}{\mainsealc}[{\makebox[0pt]{Sealed(0)}}]%
\memory{2}{\sharec}[{\makebox[0pt]{Shared}}]%
\memory{2}{\mainpassc}[{\makebox[0pt]{Passed(0)}}]%
\memory{2}{\fsealc}[{\makebox[0pt]{Sealed(1)}}]%
%\memory{2}{\unsealc}[{\makebox[0pt]{Unsealed}}]
~$\cdots$
\MemoryLabel{-29.5em}{0.75em}{\(\ldots\)}
\MemoryLabel{-25.5em}{0.75em}{0}
\\ \\
Return to {\tt main}
 &
\memoryaddrs{52em}
\memory{3}{\instrc}%
\hspace*{3pt}
$\cdots$
\memory{1}{\emptyoutc}%
\memory{2}{\unsealc}%
\memory{2}{\sharec}[{\makebox[0pt]{Shared}}]%
\memory{4}{\unsealc}
~$\cdots$
\MemoryLabel{-37.5em}{0.75em}{\(\ldots\)}
\MemoryLabel{-33.5em}{0.75em}{42}
\MemoryLabel{-29.5em}{0.75em}{\(\ldots\)}
\MemoryLabel{-25.5em}{0.75em}{0}
\MemoryLabel{-21.5em}{0.75em}{\(\ldots\)}
\MemoryLabel{-17.5em}{0.75em}{84}
\\
\end{tabular}
\end{center}

\vspace{\abovedisplayskip}

\caption{Detailed Execution Trace
%  \rb{A suggestion for the description of $\jal$ and $\jalr$:
%    call(function label) [if labels are added next to the entrypoints]
%    or call(address of entrypoint) and return()?}
}
\label{fig:share-trace}
\end{figure}

In our second example, we have a more complicated program with nested calls
and shared stack memory. Assuming that a long int takes two words, {\tt main}
can't pass {\tt x} by value via a register, but must pass it on the stack. This
is one way that memory might be shared: {\tt z} is in memory that would be sealed
by {\tt main} but should be accessible by {\tt f}. We would not expect that {\tt g}
should be able to access {\tt z}, however.

By contrast, {\tt main} has taken the address of {\tt y}. This is commonly a precursor
to intentionally sharing access to {\tt y}, and unless we can determine statically that
the address does not escape, we must treat {\tt y} as shared. This form of sharing is
more permissive than passing, as once the pointer escapes, {\tt main} has no way of knowing
how far it will be passed. We therefore make no restrictions on {\tt y} for the remainder
of its life.

There are many ways that we might want to restrict access to a shared address. For instance,
since in principle our sharing is related to the passing of an address, it would be standard
to restrict access to only those callees that actually recieve the address, as in a capability
system. This model typically works by enforcing the memory safety of all objects, including
addressed stack variables. Or, we could imagine a mandatory access control approach in which
the code is annotated to specifically identify which functions may access the shared memory.
Whatever restrictions we desire, they are orthogonal to stack safety.

Note that both passing and sharing are already visible at the point that {\tt main} calls
{\tt f}, before the remainder of its frame is sealed. The compiler annotations telling
us that passing and sharing occur are tied, respectively, to the copying of {\tt x} into
{\tt z}, and to the taking of {\tt y}'s address. Then as before, everything below the stack
pointer that is not passed or shared is sealed. Later, when {\tt f} calls {\tt g}, it also
seals its own frame with its depth (1). We skip ahead to when we return to {\tt main} to note
that while {\tt z} is no longer passed, {\tt y} is still shared, and will be until {\tt main}
returns.
