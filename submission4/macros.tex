\newif\ifdraft \drafttrue
\newif\iftext \textfalse
\newif\iflater \latertrue
\newif\ifspace \spacefalse
\newif\ifaftersubmission \aftersubmissiontrue
\newif\ifcameraready  \camerareadyfalse

% !!! PLEASE DON'T CHANGE THESE !!! INSTEAD DEFINE YOUR OWN texdirectives.tex !!!
\makeatletter \@input{texdirectives} \makeatother

%\IEEEoverridecommandlockouts
% The preceding line is only needed to identify funding in the first footnote. If that is unneeded, please comment it out.
\usepackage{cite}
\usepackage{amsmath,amsfonts}
\usepackage{algorithmic}
\usepackage{hyperref}
\usepackage{graphicx}
\usepackage{textcomp}
\usepackage[capitalize]{cleveref}
\usepackage[inline]{enumitem}

\usepackage{xcolor}
\newcommand{\bcp}[1]{\ifdraft\textcolor{violet}{{[BCP:~#1]}}\fi}
\newcommand{\BCP}[1]{\ifdraft\textcolor{violet}{{\bf [BCP:~#1]}}\fi}
\newcommand{\leo}[1]{\ifdraft\textcolor{teal}{{[LEO:~#1]}}\fi}
\newcommand{\apt}[1]{\ifdraft\textcolor{blue}{{[APT:~#1]}}\fi}
\newcommand{\rb}[1]{\ifdraft\textcolor{orange}{{[RB:~#1]}}\fi}
\definecolor{darkgreen}{RGB}{0,150,0}
\newcommand{\sna}[1]{\ifdraft\textcolor{darkgreen}{{[SNA:~#1]}}\fi}
\newcommand{\COQ}[1]{\ifdraft\textcolor{red}{{[COQ DIFFERENCE:~#1]}}\fi}

\usepackage[normalem]{ulem}
\newcommand{\proposecut}[1]{\ifdraft\textcolor{gray}{\sout{#1}}\fi}
\newcommand{\proposeadd}[1]{\ifdraft\textcolor{gray}{#1}\else{#1}\fi}

\usepackage{listings}

\usepackage{xspace}
\newcommand{\cn}{\ifdraft\textsuperscript{\textcolor{blue}{[citation needed]}}\xspace\fi}

\makeatletter
\begingroup
\lccode`\A=`\-
\lccode`\N=`\N
\lccode`\V=`\V
\lowercase{\endgroup\def\memory@noval{ANoValue-}}
\long\def\memory@fiBgb\fi#1#2{\fi}
\long\def\memory@fiTBb\fi#1#2#3{\fi#2}
\newcommand\memory@ifnovalF[1]%>>=
  {%
    \ifx\memory@noval#1%
      \memory@fiBgb
    \fi
    \@firstofone
  }%=<<
\newcommand\memory@ifnovalTF[1]%>>=
  {%
    \ifx\memory@noval#1%
      \memory@fiTBb
    \fi
    \@secondoftwo
  }%=<<
\newcommand\memory@Oarg[2]%>>=
  {%
    \@ifnextchar[{\memory@Oarg@{#2}}{#2{#1}}%
  }%=<<
\long\def\memory@Oarg@#1[#2]%>>=
  {%
    #1{#2}%
  }%=<<
\newcommand*\memory@oarg%>>=
  {%
    \memory@Oarg\memory@noval
  }%=<<
\newcommand*\memory@ifcoloropt%>>=
  {%
    \@ifnextchar[\memory@ifcoloropt@true\memory@ifcoloropt@false
  }%=<<
\long\def\memory@ifcoloropt@true#1\memory@noval#2#3%>>=
  {%
    #2%
  }%=<<
\long\def\memory@ifcoloropt@false#1\memory@noval#2#3%>>=
  {%
    #3%
  }%=<<
\newlength\memory@width
\newlength\memory@height
\setlength\memory@width{23pt}
\setlength\memory@height{14pt}
\newcount\memory@num
\newcommand*\memory@blocks[2]%>>=
  {%
    \memory@num#1\relax
    \fboxsep-\fboxrule
    \memory@ifcoloropt#2\memory@noval
      {\def\memory@color{\textcolor#2}}
      {\def\memory@color{\textcolor{#2}}}%
    \loop
    \ifnum\memory@num>0
      \fbox{\memory@color{\rule{\memory@width}{\memory@height}}}%
      \kern-\fboxrule
      \advance\memory@num\m@ne
    \repeat
  }%=<<
% memory:
%  [#1]: width
%   #2 : count
%  [#3]: height
%   #4 : colour
%  [#5]: label
\newcommand*\memory%>>=
  {%
    \begingroup
    \memory@oarg\memory@a
  }%=<<
\newcommand*\memory@a[2]%>>=
  {%
    % #1 width
    % #2 count
    \memory@ifnovalF{#1}{\memory@width#1\relax}%
    \memory@Oarg\memory@height{\memory@b{#2}}%
  }%=<<
\newcommand*\memory@b[3]%>>=
  {%
    % #1 count
    % #2 height
    % #3 colour
    \memory@ifnovalF{#2}{\memory@height#2\relax}%
    \memory@oarg{\memory@c{#1}{#3}}%
  }%=<<
\newcommand*\memory@c[3]%>>=
  {%
    % #1 count
    % #2 colour
    % #3 label
    \memory@ifnovalTF{#3}
      {\ensuremath{\memory@blocks{#1}{#2}}}
      {\ensuremath{\underbrace{\memory@blocks{#1}{#2}}_{\text{#3}}}}%
    \endgroup
  }%=<<

\makeatother

\newcommand{\judgment}[3][]{
  {\centering
  \smallskip
  \begin{tabular}{c}
    #2 \\
    \hline
    #3
  \end{tabular}{\sc #1}
  \smallskip\par}}

\newcommand{\judgmentbr}[4][]{
  {\centering
  \smallskip
  \begin{tabular}{c}
    #2 \\
    #3 \\
    \hline
    #4
  \end{tabular}{\sc #1}
   \smallskip\par}}

\newcommand{\judgmenttwobr}[6][]{
  {
    \centering
    \smallskip
    \begin{tabular}{c c}
       #2 & #3 \\
       #4 & #5 \\
       \hline
       \multicolumn{2}{c}{#6}
    \end{tabular}{\sc #1}
    \vspace{\belowdisplayskip}\par
  }}

\newcommand{\judgmenttwobrlong}[5][]{
  {
    \centering
    \smallskip
    \begin{tabular}{c c}
       #2 & #3 \\
       \multicolumn{2}{c}{#4} \\
       \hline
       \multicolumn{2}{c}{#5}
    \end{tabular}{\sc #1}
    \vspace{\belowdisplayskip}\par
  }}

\newcommand{\judgmentthreebr}[8][]{
  {
    \centering
    \smallskip
    \begin{tabular}{c c c}
       #2 & #3 & #4 \\
       #5 & #6 & #7 \\
       \hline
       \multicolumn{3}{c}{#8}
    \end{tabular}{\sc #1}
    \vspace{\belowdisplayskip}\par
  }}


\newcommand{\judgmenttwo}[4][]{
  {\centering
  \smallskip
  \begin{tabular}{c c}
    #2 & #3 \\
    \hline
    \multicolumn{2}{c}{#4}
  \end{tabular}{\sc #1}
  \smallskip\par}}

\newcommand{\judgmentthree}[5][]{
  {\centering
  \smallskip
  \begin{tabular}{c c c}
    #2 & #3 & #4 \\
    \hline
    \multicolumn{3}{c}{#5}
  \end{tabular}{\sc #1}
  \smallskip\par}}

\newcommand{\judgmentfour}[6][]{
  {\centering
  \smallskip
  \begin{tabular}{c c c c}
    #2 & #3 & #4 & #5 \\
    \hline
    \multicolumn{4}{c}{#6}
  \end{tabular}{\sc #1}
  \smallskip\par}}

% Notational conventions
\newcommand{\HIGHSEC}{\textsc{HC}}
\newcommand{\LOWSEC}{\textsc{LC}}
\newcommand{\HIGHINT}{\textsc{HI}}
\newcommand{\LOWINT}{\textsc{LI}}
\newcommand{\IDS}{{\mathcal{I}}}
\newcommand{\ID}{I}
\newcommand{\ME}{\textsc{S}}
\newcommand{\NOTME}{\textsc{O}}
\newcommand{\TRANS}{\ensuremath{-}}
\newcommand{\JAL}{\ensuremath{\mathit{JAL}}}
\newcommand{\ACCYES}{\ensuremath{A}}
\newcommand{\ACCNO}{\ensuremath{I}}
\newcommand{\ACCCODE}{\ensuremath{K}}
\newcommand{\CRCALL}{\ensuremath{\mathit{CALL}}}
\newcommand{\CRRET}{\ensuremath{\mathit{RETURN}}}
\newcommand{\CRBOT}{\ensuremath{\bot}}
\newcommand{\VIS}{\textsc{vis}}
\newcommand{\HID}{\textsc{hid}}
\newcommand{\word}{w}
\newcommand{\addr}{a}
\newcommand{\WORDS}{{\mathcal W}}
\newcommand{\reg}{r}
\newcommand{\REGS}{{\mathcal R}}
\newcommand{\mach}{m}
\newcommand{\nach}{n}
\newcommand{\machT}{M}
\newcommand{\nachT}{N}
\newcommand{\MACHS}{{\mathcal M}}
\newcommand{\MPT}{\mathit{MP}}
\newcommand{\obs}{o}
\newcommand{\obsT}{O}
\newcommand{\obsof}{\mathit{obsOf}}
\newcommand{\OBSS}{\mathit{Obs}}
\newcommand{\PC}[1]{\PCname(#1)}
\newcommand{\PCname}{\textsc{pc}}
\newcommand{\SP}{\textsc{sp}}
\newcommand{\pol}{p}
\newcommand{\POLS}{\mathcal{P}}
\newcommand{\pinit}{\pol_0}
\newcommand{\prop}{S}
\newcommand{\contour}{C}
\newcommand{\CONTOURS}{{\mathcal C}}
\newcommand{\component}{k}
\newcommand{\components}{K}
\newcommand{\COMPONENTS}{{\mathcal K}}
\newcommand{\trace}{T}
\newcommand{\observer}{O}
\newcommand{\stateobs}{\sigma}
\newcommand{\seq}[1]{\overline{#1}}
\newcommand{\SEQ}[1]{\overline{#1}}
\newcommand{\dstk}[1]{{#1}.\mbox{\it stack}}
\newcommand{\dpcd}[1]{{#1}.\mbox{\it PCdepth}}
\newcommand{\ddep}[2]{{#1}.\mbox{\it depth}({#2})}
\newcommand{\dinit}{\mbox{\it Dinit}}
\newcommand{\empstack}{\mbox{\it empty}}
\newcommand{\access}[2]{\mbox{\it accessible}_{#1}({#2})}
\newcommand{\norm}[1]{\lvert{#1}\rvert}
\newcommand{\MPS}{\mathit{MPState}}
\newcommand{\mpstate}[2]{#1,#2}
\newcommand{\mpostate}[3]{#1,#2,#3}
\newcommand{\mpstatename}{mp}
\newcommand{\mpcstate}[3]{#1,#2,#3}
\newcommand{\mpcstatename}{mc}
\newcommand{\npcstatename}{nc}
\newcommand{\MPCS}{\mathit{MCState}}
\newcommand{\MPCT}{\mathit{M}}
\newcommand{\NPCT}{\mathit{N}}
\newcommand{\notfinished}[2]{#1 \cdot #2}

\newcommand{\ret}[1]{\mathit{justret}\ #1}
\newcommand{\nextPC}{next}
\newcommand{\base}{b}
\newcommand{\stepstounder}[1]{\stackrel{\mbox{\tiny{$#1$}}}{\Longrightarrow}}
\newcommand{\harpoonunder}[1]{\stackrel{\mbox{\tiny{$#1$}}}{\rightharpoonup}}
\newcommand{\stepstounderfull}{\stepstounder{\textsc{RISCV}}}
\newcommand{\manystepsto}{\stepsto^\star}
\newcommand{\obstrace}{\mathit{obstrace}}
\newcommand{\funid}{f}
\newcommand{\FUNIDS}{\mathcal{F}}
\newcommand{\codemap}{\mathit{cm}}
\newcommand{\CODEMAPS}{\mathit{codeMap}}
\newcommand{\option}[1]{\mathbf{opt}(#1)}
\newcommand{\ann}{\mathit{ann}}
\newcommand{\anns}{\mathit{anns}}
\newcommand{\ANNS}{\mathit{Ann}}
\newcommand{\callmap}{\text{\sc call}}
\newcommand{\retmap}{\text{\sc ret}}
\newcommand{\entmap}{\text{\sc entry}}
\newcommand{\sharemap}{\text{\sc share}}
\newcommand{\passmap}{\text{\sc pass}}
\newcommand{\yieldmap}{\text{\sc yield}}
\newcommand{\PUT}{\mathit{Until}}
\newcommand{\Segment}{\mathit{Segment}}
\newcommand{\head}{\mathit{hd}}
\newcommand{\last}{\mathit{last}}

\newcommand{\stepsto}[1][]{\harpoonunder{#1}}
\newcommand{\polstep}{\rightharpoonup}
\newcommand{\stepstopol}[1][]{\overset{#1}{\rightharpoonup}}
%\newcommand{\stepstopol}[1]{\overset{#1}{\rightharpoonup}_P}
\newcommand{\constep}{\rightarrow_C}
\newcommand{\stepstocon}[1][]{\overset{#1}{\rightharpoonup}_C}

\newcommand{\stepplus}{\Rightarrow}
\newcommand{\stepkappa}{\Rightarrow_\kappa}
\newcommand{\induced}[2]{(#1, #2)^*}
\newcommand{\flows}{\sqsubseteq}
\newcommand{\flowsstrict}{\sqsubset}
\newcommand{\initmach}{\MACHS_{\mathit{init}}}
\newcommand{\initcontour}{\CONTOURS_{\mathit{init}}}
\newcommand{\closure}[1]{\textit{Close}#1}
\newcommand{\variant}[2]{\textit{Vars}(#1, #2)}
\newcommand{\isinf}{\mathit{inf}}

\newcommand{\stackDom}{\mathit{sd}}
\newcommand{\STKDOMS}{\mathit{SD}}
\newcommand{\topDom}{\mathit{td}}
\newcommand{\TOPDOMS}{\mathit{TD}}
\newcommand{\instack}[2]{\mathit{instk}(#1,#2)}
\newcommand{\outside}{\mathit{outside}}
\newcommand{\unsealed}{\mathit{unsealed}}
\newcommand{\sealed}[1]{\mathit{sealed(#1)}}
\newcommand{\passed}[1]{\mathit{passed(#1)}}
\newcommand{\shared}[1]{\mathit{shared(#1)}}
\newcommand{\unsaved}{\mathit{unsaved}}
\newcommand{\saved}{\mathit{saved}}
\newcommand{\depth}{d}
\newcommand{\domMap}{\mathit{dm}}
\newcommand{\depthMap}{\mathit{depm}}
\newcommand{\target}{\mathit{tar}}
\newcommand{\listT}{\mathit{list}}

\newcommand{\WF}{\mathit{WF}}

\newcommand{\stackid}{\mathit{stk}}
\newcommand{\stackof}{\mathit{stkOf}}
\newcommand{\STACKS}{\mathit{STKS}}

\newcommand{\rts}{\mathit{rts}}
\newcommand{\rtm}{\mathit{rm}}
\newcommand{\yts}{\mathit{yts}}
\newcommand{\push}[3]{#1 \downarrow^{#2}_{#3}}
\newcommand{\pop}[2]{#1 \uparrow^{#2}}
\newcommand{\sealcon}{\mathit{seal?}}
\newcommand{\retcon}{\mathit{isRet}}
\newcommand{\ycon}{\mathit{isBack}}

\newcommand{\context}{\mathit{ctx}}
\newcommand{\CONTEXTS}{\mathit{CTX}}

\newcommand{\Last}[1]{\mathit{Last}(#1)}

\newcommand{\HALT}{\textsc{HALT}}

\newcommand{\underscore}{\mbox{\_}}

\newcommand{\propdef}[1]{\text{\sc #1}}

\newcommand{\TRACE}[1]{\mathit{Traces}\,(#1)}
\newcommand{\MTRACE}{\TRACE{\MACHS}}
\newcommand{\MOTRACE}{\TRACE{\MACHS \times \OBSS}}
\newcommand{\MPOTRACE}{\TRACE{\MACHS \times \POLS \times \OBSS}}

\newcommand{\URL}[1]{#1}

\newcommand{\prot}{\mathit{protected}}
\newcommand{\secret}{\mathit{secret}}
\newcommand{\brk}{\mathit{brk}}

\newcommand*{\rsp}{\textsc{sp}}

\newcommand{\TAGS}{\mathcal{T}}
\newcommand{\tagname}{t}
\newcommand{\uP}{{\mu P}}

\newcommand*{\MemoryLabel}[3]{\raisebox{#2}{\makebox(0,0){\hspace{#1}#3}}}
\newcommand{\memoryaddrs}[2][]
  {
    \MemoryLabel{#2}{2em}{$\downarrow$\SP #1}
  }
\newcommand*{\tagInstr}{\textsc{instr}}
\newcommand*{\tagCall}{\textsc{call}}
\newcommand*{\tagHa}{\textsc{h1}}
\newcommand*{\tagHb}{\textsc{h2}}
\newcommand*{\tagRa}{\textsc{r1}}
\newcommand*{\tagRb}{\textsc{r2}}
\newcommand*{\tagRc}{\textsc{r3}}
\newcommand*{\tagNoDepth}{\textsc{unused}}
\newcommand*{\tagStackDepth}[1]{\textsc{stack} ~ #1}
\newcommand*{\tagPCDepth}[1]{\textsc{pc} ~ #1}
\newcommand*{\tagSP}{\textsc{sp}}
