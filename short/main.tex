%% For double-blind review submission, w/o CCS and ACM Reference (max submission space)
\documentclass[acmtog,review,anonymous]{acmart}\settopmatter{printfolios=true,printccs=false,printacmref=false}
%% For double-blind review submission, w/ CCS and ACM Reference
%\documentclass[acmsmall,review,anonymous]{acmart}\settopmatter{printfolios=true}
%% For single-blind review submission, w/o CCS and ACM Reference (max submission space)
%\documentclass[acmsmall,review]{acmart}\settopmatter{printfolios=true,printccs=false,printacmref=false}
%% For single-blind review submission, w/ CCS and ACM Reference
%\documentclass[acmsmall,review]{acmart}\settopmatter{printfolios=true}
%% For final camera-ready submission, w/ required CCS and ACM Reference
%\documentclass[acmsmall]{acmart}\settopmatter{}

%% Journal information
%% Supplied to authors by publisher for camera-ready submission;
%% use defaults for review submission.
\acmJournal{PACMPL}
\acmVolume{1}
\acmNumber{OOPSLA} % CONF = POPL or ICFP or OOPSLA
\acmArticle{1}
\acmYear{2021}
\acmMonth{1}
\acmDOI{} % \acmDOI{10.1145/nnnnnnn.nnnnnnn}
\startPage{1}

%% Copyright information
%% Supplied to authors (based on authors' rights management selection;
%% see authors.acm.org) by publisher for camera-ready submission;
%% use 'none' for review submission.
\setcopyright{none}
%\setcopyright{acmcopyright}
%\setcopyright{acmlicensed}
%\setcopyright{rightsretained}
%\copyrightyear{2018}           %% If different from \acmYear

%% Bibliography style
\bibliographystyle{ACM-Reference-Format}
%% Citation style
%% Note: author/year citations are required for papers published as an
%% issue of PACMPL.
\citestyle{acmnumeric}   %% For author/year citations


%%%%%%%%%%%%%%%%%%%%%%%%%%%%%%%%%%%%%%%%%%%%%%%%%%%%%%%%%%%%%%%%%%%%%%
%% Note: Authors migrating a paper from PACMPL format to traditional
%% SIGPLAN proceedings format must update the '\documentclass' and
%% topmatter commands above; see 'acmart-sigplanproc-template.tex'.
%%%%%%%%%%%%%%%%%%%%%%%%%%%%%%%%%%%%%%%%%%%%%%%%%%%%%%%%%%%%%%%%%%%%%%

%% Some recommended packages.
\usepackage{booktabs}   %% For formal tables:
                        %% http://ctan.org/pkg/booktabs
\usepackage{subcaption} %% For complex figures with subfigures/subcaptions
                        %% http://ctan.org/pkg/subcaption

\newif\ifdraft \drafttrue
\newif\iftext \texttrue

\IEEEoverridecommandlockouts
% The preceding line is only needed to identify funding in the first footnote. If that is unneeded, please comment it out.
\usepackage{cite}
\usepackage{amsmath,amssymb,amsfonts}
\usepackage{algorithmic}
\usepackage{hyperref}
\usepackage{graphicx}
\usepackage{textcomp}
\usepackage{cleveref}
\usepackage[inline]{enumitem}

\usepackage{xcolor}
\newcommand{\bcp}[1]{\ifdraft\textcolor{violet}{{[BCP:~#1]}}\fi}
\newcommand{\leo}[1]{\ifdraft\textcolor{teal}{{[LEO:~#1]}}\fi}
\newcommand{\apt}[1]{\ifdraft\textcolor{blue}{{[APT:~#1]}}\fi}

\usepackage{xspace}
\newcommand{\cn}{\ifdraft\textsuperscript{\textcolor{blue}{[citation needed]}}\xspace\fi}


% Notational conventions
\newcommand{\HIGH}{\textsc{H}}
\newcommand{\LOW}{\textsc{L}}
\newcommand{\HI}{\ensuremath{\top}}
\newcommand{\LO}{\ensuremath{\bot}}
\newcommand{\VIS}{\textsc{vis}}
\newcommand{\HID}{\textsc{hid}}
\newcommand{\word}{W}
\newcommand{\addr}{A}
\newcommand{\WORDS}{{\mathcal W}}
\newcommand{\reg}{R}
\newcommand{\REGS}{{\mathcal R}}
\newcommand{\mach}{M}
\newcommand{\MACHS}{{\mathcal M}}
\newcommand{\PC}[1]{\PCname(#1)}
\newcommand{\PCname}{\textsc{pc}}
\newcommand{\pol}{P}
\newcommand{\prop}{S}
\newcommand{\contour}{C}
\newcommand{\CONTOURS}{{\mathcal C}}
\newcommand{\component}{K}
\newcommand{\COMPONENTS}{{\mathcal K}}
\newcommand{\trace}{T}
\newcommand{\observer}{O}
\newcommand{\stateobs}{\sigma}
\newcommand{\seq}[1]{\overline{#1}}
\newcommand{\SEQ}[1]{\overline{#1}}
\newcommand{\dstk}[1]{{#1}.\mbox{\it stack}}
\newcommand{\dpcd}[1]{{#1}.\mbox{\it PCdepth}}
\newcommand{\ddep}[2]{{#1}.\mbox{\it depth}({#2})}
\newcommand{\dinit}{\mbox{\it Dinit}}
\newcommand{\empstack}{\mbox{\it empty}}
\newcommand{\access}[2]{\mbox{\it accessible}_{#1}({#2})}
\newcommand{\norm}[1]{\lvert{#1}\rvert}

\newcommand{\stepsto}{\Longrightarrow}
\newcommand{\stepstounder}[1]{\stackrel{\mbox{\tiny{$#1$}}}{\Longrightarrow}}
\newcommand{\stepstounderfull}{\stepstounder{\textsc{RISCV}}}
\newcommand{\manystepsto}{\stepsto^\star}
\newcommand{\obstrace}{\mathit{obstrace}}

\newcommand{\underscore}{\mbox{\_}}


\begin{document}

%% Title information
\title{Security Properties for Stack Safety}         %% [Short Title] is optional;
                                        %% when present, will be used in
                                        %% header instead of Full Title.
%\titlenote{with title note}             %% \titlenote is optional;
%                                        %% can be repeated if necessary;
%                                        %% contents suppressed with 'anonymous'
%\subtitle{Subtitle}                     %% \subtitle is optional
%\subtitlenote{with subtitle note}       %% \subtitlenote is optional;
%                                        %% can be repeated if necessary;
%                                        %% contents suppressed with 'anonymous'


%% Author information
%% Contents and number of authors suppressed with 'anonymous'.
%% Each author should be introduced by \author, followed by
%% \authornote (optional), \orcid (optional), \affiliation, and
%% \email.
%% An author may have multiple affiliations and/or emails; repeat the
%% appropriate command.
%% Many elements are not rendered, but should be provided for metadata
%% extraction tools.

%% Author with single affiliation.
%\author{Sean Noble Anderson}
%\authornote{with author1 note}          %% \authornote is optional;
%                                        %% can be repeated if necessary
%\orcid{nnnn-nnnn-nnnn-nnnn}             %% \orcid is optional
%\affiliation{
%  \position{Position1}
%  \department{Computer Science}              %% \department is recommended
%  \institution{Portland State University}            %% \institution is required
%  \streetaddress{Street1 Address1}
%  \city{City1}
%  \state{State1}
%  \postcode{Post-Code1}
%  \country{Country1}                    %% \country is recommended
%}
%\email{ander28@pdx.edu}          %% \email is recommended

%\author{Leonidas Lampropoulos}
%\affiliation{
%  \institution{University of Maryland, College Park}
%}
%\email{leonidas@umd.edu}

%\author{Roberto Blanco}
%\affiliation{
%  \institution{Max Planck Institute for Security and Privacy}
%}
%\email{roberto.blanco@mpi-sp.org}

%\author{Benjamin C. Pierce}
%\affiliation{
%  \institution{University of Pennsylvania}
%}
%\email{bcpierce@cis.upenn.edu}

%\author{Andrew Tolmach}
%\affiliation{
%  \institution{Portland State University}
%}
%\email{tolmach@pdx.edu}

%
%%% Author with two affiliations and emails.
%\author{First2 Last2}
%\authornote{with author2 note}          %% \authornote is optional;
%                                        %% can be repeated if necessary
%\orcid{nnnn-nnnn-nnnn-nnnn}             %% \orcid is optional
%\affiliation{
%  \position{Position2a}
%  \department{Department2a}             %% \department is recommended
%  \institution{Institution2a}           %% \institution is required
%  \streetaddress{Street2a Address2a}
%  \city{City2a}
%  \state{State2a}
%  \postcode{Post-Code2a}
%  \country{Country2a}                   %% \country is recommended
%}
%\email{first2.last2@inst2a.com}         %% \email is recommended
%\affiliation{
%  \position{Position2b}
%  \department{Department2b}             %% \department is recommended
%  \institution{Institution2b}           %% \institution is required
%  \streetaddress{Street3b Address2b}
%  \city{City2b}
%  \state{State2b}
%  \postcode{Post-Code2b}
%  \country{Country2b}                   %% \country is recommended
%}
%\email{first2.last2@inst2b.org}         %% \email is recommended

%% Abstract
%% Note: \begin{abstract}...\end{abstract} environment must come
%% before \maketitle command
%\begin{abstract}
%What exactly does ``stack safety'' mean? The phrase is associated with a
%variety of compiler,
%run-time, and hardware mechanisms for protecting stack
%memory.  But these mechanisms typically lack precise specifications,
%relying instead on informal descriptions and examples of bad
%behaviors that they prevent.

%We propose a formal characterization
%of stack safety, formulated with concepts from language-based security: a
%combination of an integrity property (``the private
%state in each caller's stack frame is held invariant by the callee''),
%a confidentiality property (``the callee's behavior is insensitive to the
%caller's private state''), and a well-bracketedness property (``each
%callee returns control to its immediate caller'').

%We use these properties to validate the stack-safety ``micro-policies''
%proposed by~\citet{DBLP:conf/sp/RoesslerD18}.  Specifically, we check (with
%property-based random testing) that Roessler and Dehon's ``eager''
%micro-policy, which catches violations as early as possible, enforces a
%simple ``stepwise'' variant of our properties and correctly detects several
%broken variants, and that (a repaired version of) their more performant
%``lazy'' micro-policy corresponds to a slightly weaker and more extensional
%``observational'' variant of our properties.

%\end{abstract}


%% 2012 ACM Computing Classification System (CSS) concepts
%% Generate at 'http://dl.acm.org/ccs/ccs.cfm'.
\begin{CCSXML}
<ccs2012>
<concept>
<concept_id>10011007.10011006.10011008</concept_id>
<concept_desc>Software and its engineering~General programming languages</concept_desc>
<concept_significance>500</concept_significance>
</concept>
<concept>
<concept_id>10003456.10003457.10003521.10003525</concept_id>
<concept_desc>Social and professional topics~History of programming languages</concept_desc>
<concept_significance>300</concept_significance>
</concept>
</ccs2012>
\end{CCSXML}

%\ccsdesc[500]{Software and its engineering~General programming languages}
%\ccsdesc[300]{Social and professional topics~History of programming languages}
%% End of generated code


%% Keywords
%% comma separated list
\ifcameraready
\keywords{Stack Safety, Micro-Policies}  %% \keywords are mandatory in final camera-ready submission
\fi


%% \maketitle
%% Note: \maketitle command must come after title commands, author
%% commands, abstract environment, Computing Classification System
%% environment and commands, and keywords command.
\maketitle

%\section{Introduction}

The call stack is a perennial target for low-level attacks, leading to
consequences ranging from leakage or corruption of private stack data to
control-flow hijacking. To prevent or detect such attacks, a profusion of
software and hardware protections have been proposed,
%
including stack canaries~\citep{Cowan+98},
bounds checking~\citep{NagarakatteZMZ09,NagarakatteZMZ10,DeviettiBMZ08},
split stacks~\citep{Kuznetsov+14},
shadow stacks~\citep{Dang+15,Shanbhogue+19},
capabilities~\citep{Woodruff+14,Chisnall+15,SkorstengaardLocal,SkorstengaardSTK,Georges+21},
and hardware tagging~\citep{DBLP:conf/sp/RoesslerD18}.
%
The protections offered by such mechanisms are commonly described in terms
of concrete examples of attacks that they can prevent.
At best, they define stack safety by reference to an idealized machine that
is arguably stack safe by construction \citep{SkorstengaardSTK}.
But these mechanisms can be intricate, and it would be useful to have a precise, generic, and formal
specification for stack safety, both to compare the security claims of different
enforcement techniques and to rigorously validate such claims.

We propose such a characterization,
using the technical framework of language-based security.
Stack safety protects a caller from its callee, guaranteeing
the {\em integrity} and {\em confidentiality} of the caller's local state
until it regains control. This formulation not only captures the intuition
that the callee cannot directly access the caller's state, but gives a novel
way of looking at control-flow attacks such as return-oriented-programming (ROP).
In an ROP attack, the callee pretends to return to its caller, thus accessing the
caller's privileges, but does so in such a way that it is still in control of
execution. Our model allows the caller to dictate the terms of a valid return --
typically, that the stack pointer is restored and execution proceeds from the
return address. An ROP attack that ``returns'' and reads the caller's state is
therefore treated precisely as if the callee had never returned, but attempted
to read the caller's state directly. It is a confidentiality violation. Likewise
an attack that would corrupt the caller's state is an integrity violation.

We formalize integrity and confidentiality as trace properties, in two different
variants: {\em stepwise} variants, in which a caller's data is {\em never}
read or modified during a call, and {\em observational} ones, in which
callees may read from and write to their caller's stack frame, as
long as these ``risky'' behaviors do not affect the system's observable
behavior. The observational properties are more extensional, and
any reasonable protection mechanism ought to enforce them,
even if it does not prevent every single dangerous read or write.
Confidentiality is especially interesting, as it is based on the traditional
notion of noninterference. But where noninterference is normally presented as an
end-to-end hyperproperty, stack confidentiality is noninterference applied over
multiple nested subtraces -- one for every call.

To demonstrate the utility of our properties, we use them
to validate an existing mechanism, the
{\em stack-safety micro-policies} of~\citet{DBLP:conf/sp/RoesslerD18}, re-implemented
in the Coq proof assistant on top of a RISC-V specification. We
use QuickChick~\citep{Denes:VSL2014,Pierce:SF4}, a property-based testing
tool for Coq, to generate random programs and check
that these micro-policies correctly abort programs that
would violate stack safety.

We find that Roessler and Dehon's {\em Depth Isolation} micro-policy, in
which memory cells within each stack frame are tagged with the identity of
the function activation that owns the frame and access to those locations is
then permitted only when that activation is currently executing, validates our
stepwise properties. On the other hand, we reason that \emph{Lazy Tagging and Clearing}
violates the temporal aspect of confidentiality in
corner cases where data can leak across repeated calls to the same callee,
and also violates integrity if the leak happens to use the caller's frame. We
propose a variant of {\em Lazy Tagging and Clearing} that \emph{should} enforce
confidentiality, albeit at some performance cost.

Finally, we demonstrate our model's flexibility by extending it with the passing
of variables on the stack and to a simple coroutine model.

\bibliography{bcp.bib,local.bib}

%% Appendix
%\appendix
%\section{Appendix}
%Text of appendix \ldots

\end{document}
