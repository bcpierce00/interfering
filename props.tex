\documentclass[conference]{IEEEtran}
\newif\ifdraft \drafttrue
\newif\iftext \texttrue

\IEEEoverridecommandlockouts
% The preceding line is only needed to identify funding in the first footnote. If that is unneeded, please comment it out.
\usepackage{cite}
\usepackage{amsmath,amssymb,amsfonts}
\usepackage{algorithmic}
\usepackage{hyperref}
\usepackage{graphicx}
\usepackage{textcomp}
\usepackage{cleveref}
\usepackage[inline]{enumitem}

\usepackage{xcolor}
\newcommand{\bcp}[1]{\ifdraft\textcolor{violet}{{[BCP:~#1]}}\fi}
\newcommand{\leo}[1]{\ifdraft\textcolor{teal}{{[LEO:~#1]}}\fi}
\newcommand{\apt}[1]{\ifdraft\textcolor{blue}{{[APT:~#1]}}\fi}

\usepackage{xspace}
\newcommand{\cn}{\ifdraft\textsuperscript{\textcolor{blue}{[citation needed]}}\xspace\fi}


% Notational conventions
\newcommand{\HIGH}{\textsc{H}}
\newcommand{\LOW}{\textsc{L}}
\newcommand{\HI}{\ensuremath{\top}}
\newcommand{\LO}{\ensuremath{\bot}}
\newcommand{\VIS}{\textsc{vis}}
\newcommand{\HID}{\textsc{hid}}
\newcommand{\word}{W}
\newcommand{\addr}{A}
\newcommand{\WORDS}{{\mathcal W}}
\newcommand{\reg}{R}
\newcommand{\REGS}{{\mathcal R}}
\newcommand{\mach}{M}
\newcommand{\MACHS}{{\mathcal M}}
\newcommand{\PC}[1]{\PCname(#1)}
\newcommand{\PCname}{\textsc{pc}}
\newcommand{\pol}{P}
\newcommand{\prop}{S}
\newcommand{\contour}{C}
\newcommand{\CONTOURS}{{\mathcal C}}
\newcommand{\component}{K}
\newcommand{\COMPONENTS}{{\mathcal K}}
\newcommand{\trace}{T}
\newcommand{\observer}{O}
\newcommand{\stateobs}{\sigma}
\newcommand{\seq}[1]{\overline{#1}}
\newcommand{\SEQ}[1]{\overline{#1}}
\newcommand{\dstk}[1]{{#1}.\mbox{\it stack}}
\newcommand{\dpcd}[1]{{#1}.\mbox{\it PCdepth}}
\newcommand{\ddep}[2]{{#1}.\mbox{\it depth}({#2})}
\newcommand{\dinit}{\mbox{\it Dinit}}
\newcommand{\empstack}{\mbox{\it empty}}
\newcommand{\access}[2]{\mbox{\it accessible}_{#1}({#2})}
\newcommand{\norm}[1]{\lvert{#1}\rvert}

\newcommand{\stepsto}{\Longrightarrow}
\newcommand{\stepstounder}[1]{\stackrel{\mbox{\tiny{$#1$}}}{\Longrightarrow}}
\newcommand{\stepstounderfull}{\stepstounder{\textsc{RISCV}}}
\newcommand{\manystepsto}{\stepsto^\star}
\newcommand{\obstrace}{\mathit{obstrace}}

\newcommand{\underscore}{\mbox{\_}}


\usepackage{geometry}
\usepackage{amssymb,amsmath}

\begin{document}

  \section{Preliminaries}

    Values and addresses are words:
    \[\word,\addr \in \WORDS\]

    Components are words (addresses) and registers, including special registers:
    \[\reg \in \REGS\]
    \[\component \in \COMPONENTS = \WORDS + \REGS + \SP + \PCname\]

    A machine state maps components to values:
    \[\mach \in \MACHS = \COMPONENTS \rightarrow \WORDS\]

    Our properties are defined in terms of observations, which may be silent or a word of data:
    \[\obs \in \OBSS = \tau | \word \in \WORDS\]
    
    Execution is by a step function \(\mach \stepstoobs{\obs} \mach' : \MACHS \rightarrow
    \OBSS \times \MACHS\).

    A policy is a general model of an enforcement mechanism, consisting of
    a set of policy states, \(\pol \in \POLS\), a policy initializer
    \(pinit : \MACHS \rightharpoonup \POLS\), and a policy step function \(\pol \stepstopol \pol' :
    \POLS \rightharpoonup \POLS\). The initializer and step function are both partial functions,
    the former to reflect static guarantees (some initial states may not be valid at all) and
    the latter to reflect dynamic fail-stop behavior. We then define machine-policy pairs and
    a step function on them:

    \[\MPS = \MACHS \times \POLS\]
    \[\stepstoobs{}_P \subseteq \MPS \times \MPS\]

    \[\frac{\mach_1 \stepstoobs{\obs} \mach_2 \hspace{.5cm} \pol_1 \stepstopol \pol_2}
           {\mpstate{\mach_1}{\pol_1} \stepstoobs{\obs}_P \mpstate{\mach_2}{\pol_2}}\]

  \paragraph{Traces}

    A trace is a potentially infinite, nonempty sequence. We define the append operation \(T \cdot T'\)
    where \(T'\) may be a trace or empty (\(\varepsilon\)):
    \[T_1 \cdot T_2 \triangleq
      \begin{cases}
        T_1 & \text{if } T_2 = \varepsilon \\
        t T_2 & \text{if } T_1 = t \\
        t T_1' \cdot T_2 & \text{if } T_1 = t T_1' \\
      \end{cases}\]

    We identify infinite traces with \(\isinf(T\). We define a ``trace-of'' operator \(\hookrightarrow\)
    that coinductively relates a machine state with the trace of machine states paired with observations
    produced by repeated application of step:

    \[\frac{\mach_0 \stepstoobs{\obs} \mach_1 \hspace{.5cm} \mach_1 \hookrightarrow \machT}
           {\mach_0 \hookrightarrow (\mach_0,\obs) \machT}\]

    Since the step function is total, if \(\mach \hookrightarrow \machT\), then \(\machT\) must be infinite.
    We will also wish to take a prefix of the trace from an initial state up through 
    the first state where some condition holds. Let \(f\) be a predicate on machine states.
    Then we define \(\mach \hookrightarrow \machT | f\) (read ``\(\machT\) is the prefix from
    \(\mach\) until \(f\)''):
    \[\frac{\mach_0 \stepstoobs{\obs} \mach_1 \hspace{.5cm} \neg f\ \mach_1 \hspace{.5cm} \mach_1
              \hookrightarrow \machT | f}
           {\mach_0 \hookrightarrow (\mach_0,\obs) \machT | f}\]
    \[\frac{f\ m}
           {m \hookrightarrow (m,\tau)}\]

    We define \(\hookrightarrow_P\) to relate a machine-policy pair to the trace of MP pairs and observations
    created by \(\stepstoobs{}_P\):

    \[\frac{\forall \pol_1 . \pol_0 \not \longrightarrow \pol_1}
           {(\mach_0,\pol_0) \hookrightarrow (\mach_0,\pol_0)}\]
    \[\frac{\mach_0 \stepstoobs{\obs} \mach_1 \hspace{.5cm} \pol_0 \longrightarrow \pol_1 \hspace{.5cm}
              (\mach_1,\pol_1) \hookrightarrow \machT}
           {(\mach_0,\pol_0) \hookrightarrow_P (\mach_0,\pol_0,\obs) \machT}\]
    
    \[\frac{\mach_0,\pol_0 \xrightarrow{\obs} \mach_1,\pol_1 \hspace{.5cm} \neg f\ \mach_1 \hspace{.5cm}
              \mach_1,\pol_1 \hookrightarrow_P \machT | f }
           {\mach_0,\pol_0 \hookrightarrow_P \mach_0,\pol_0 \machT | f}\]
    \[\frac{\mach_0,\pol_0 \xrightarrow{\obs} \mach_1,\pol_1 \hspace{.5cm} f\ \mach_1}
           {\mach_0,\pol_0 \hookrightarrow_P (\mach_0,\pol_0) (\mach_1,\pol_1)}\]

    [Coq difference alert: this operation steps until it reaches a state on which \(f\) holds,
    and is infinite if it doesn't. It doesn't compute the observation of the step after the final,
    which coq would. So coq may need a ``step until'' relation rather than ``prefix up to''.
    Still thinking about this.]

    We project the machine states from either machine trace with \(\pi_m\) and the observations with \(\pi_o\).
    An observational prefix \(\obsT_1 \lesssim \obsT_2\) means that the trace \(\obsT_1\) of observations is
    a prefix of \(\obsT_2\) up to the deletion of \(\tau\) observations:

    \[\frac{}
           {\obsT \lesssim \obsT}\]

    \[\frac{}
           {\tau \lesssim \obsT}\]

    \[\frac{\obsT_1 = \tau \obsT_1' \hspace{.5cm} \obsT_1' \lesssim \obsT_2}
           {\obsT_1 \lesssim \obsT_2}\]

    \[\frac{\obsT_2 = \tau \obsT_2' \hspace{.5cm} \obsT_1 \lesssim \obsT_2'}
           {\obsT_1 \lesssim \obsT_2}\]

    \[\frac{\obsT_1 = w \obsT_1' \hspace{.5cm} \obsT_2 = w \obsT_2'}
           {\obsT_1 \lesssim \obsT_2}\]

    [This is the same as the coq version minus a redundant case.]

    Equivalence of observation traces is defined as traces prefixing each other:
    \[O_1 \simeq O_2 \triangleq O_1 \lesssim O_2 \land O_2 \lesssim O_1\]
    
    [Coq difference alert: this should be equivalent to the coq version, and we should adopt it.]

    We define policy-sensitive observational equivalence, which relates a machine trace and
    an {\it mptrace} whose observable behavior match it, up to a possible halt due to policy failure.

    \[\frac{\isinf(\MPT) \hspace{.5cm} \mathit{Obs}(\MPT) \lesssim \mathit{Obs}(\machT)}
           {\MPT \sim_P \machT}\]
    
    \[\frac{\neg \isinf(\MPT) \hspace{.5cm} \mathit{Obs}(\MPT) \simeq \mathit{Obs}(\machT)}
           {\MPT \sim_P \machT}\]

    [Todo: update coq to use this.]

  \paragraph{Call Maps and Returns}

    A {\it call map} identifies machine states that represent a call, abstracting away the explicit
    calling convention. Formally it maps some machine states to a number of arguments:
    \[\callmap \in \CALLMAPS = \mach \rightharpoonup \mathbb{N}\]

    As it pertains to a typical calling convention, for purposes of understanding the properties,
    a call map may be considered to identify the state just before the \(\JAL\), in which space for
    arguments is already allocated. But the mechanism is more general.

    A state on which a call map is defined is referred to as a {\it call state}, and we consider
    the call to have returned the first time control returns to the instruction following the
    \(\JAL\) with the stack restored. Formally this is a relation between the states:
    
    \[\begin{split}
      \ret{\mach_c}\ \mach_r \triangleq & \mach_r(\PCname) = \mach_c(\PCname)+1 \land \\
                                        & \mach_r(\SP) = \mach_c(\SP)
    \end{split}\]

  \section{Observable and Eager Stack Safety}

    A stack safety property is a property of a transition system, a policy, and a callmap
    that is decomposed into an integrity property, a confidentiality property, and a safe
    initialization property. We will present eager and observational variants of stack safety.

    {\it Eager stack integrity} states that if a function call returns, the caller's stack data
    will be identical to what it was before the call. This implies the weaker {\it observable
    stack integrity}, which states that any changes to the caller's frame will produce the same
    observable behavior.

    {\it Eager stack confidentiality} states that data outside a callee's stack frame is secret,
    and does not interfere with its observable behavior or with the machine state at return. Its
    weaker variant {\it observable stack safety} lifts the latter restriction and states that
    the callee may not leak these secrets directly or by changing the behavior of its caller.

    Finally, {\it safe initialization} deals with the top level outside of a function call, in
    which uninitiaized data should not interfere with any observable behavior.

    [Coq difference alert: safe initialization is implemented differently in each property, since
     it's really just confidentiality. But because there's no return it's functionally the same.]

    Our top level properties are {\it eager stack safety} and {\it observable stack safety}:
    \[\begin{split}
      & \textit{observable stack safety} \triangleq \\
      & \textit{observable stack integrity} \land \\
      & \textit{observable stack confidentiality} \land \\
      & \textit{safe initialization}
    \end{split}\]
    \[\begin{split}
      & \textit{eager stack safety} \triangleq \\
      & \textit{eager stack integrity} \land \\
      & \textit{eager stack confidentiality} \land \\
      & \textit{safe initialization}
    \end{split}\]

  %  We unify these properties with a structure
  %  called a {\it contour} that maps components to levels of confidentiality and integrity.
  %  Specifically, we have confidentiality labels \(\mathit{labelC} ::= \{\HIGHSEC, \LOWSEC\}\)
  %  and integrity labels \(\mathit{labelI} ::= \{\HIGHINT, \LOWINT\}\), and a
  %  type \(\mathit{contour}\)

    \paragraph{Safe Initialization}

      In the initial state of a program, nothing has yet been allocated on the stack, and
      there is no guarantee of the initial value of memory in the stack. So there is no
      need to protect a caller's stack data, but reading uninitialized data is unsafe.

      Let \(\contour : \COMPONENTS \rightarrow \{\HIGHSEC,\LOWSEC\}\) map all addresses
      to \(\HIGHSEC\) (high confidentiality) and all registers to \(\LOWSEC\)
      (low confidentiality). In the style of non-interference, a pair of states \(\mach\)
      and \(\mach'\) are variants with regard to \(\contour\), \(\mach \approx_\contour \mach'\), if:
      \[\forall \component . \contour(\component) = \LOWSEC \rightarrow \mach(\component) = \mach'(\component)\]
      
      A system enjoys {\it safe initialization} if for any initial states \(\mach_0\)
      and \(\mach_0'\) such that \(\mach_0 \approx_\contour \mach_0'\):

      \begin{itemize}
        \item Let \(\mach_0,\mathit{pinit}(\mach_0) \hookrightarrow \MPT\) and \(\mach_0' \hookrightarrow \machT'\)
        \item \(\MPT \sim_P \machT'\)
      \end{itemize}

    \paragraph{Contours}

      Our remaining properties concern the behavior of function calls and require a dynamic treatment
      of confidentiality and integrity. We generalize \(C\) above to a {\it contour} mapping each component
      to a label:

      \[\mathit{label} ::= \{\HIGHSEC,\LOWSEC\} \times \{\HIGHINT,\LOWINT\}\]

      \[\contour \in \CONTOURS ::= \COMPONENTS \rightarrow \mathit{label}\]

      When \(\mpstate{\mach}{\pol}\) is a call state \(\callmap(\mach) = n\) for some \(n\),
      we can construct a contour based on the bounds of the callee's stack frame, as follows:
      \[\mathit{Cof}(\mach,n)(\component) =
      \begin{cases}
        (\HIGHSEC,\HIGHINT) & \text{if } \component \leq \mach(\SP) - n \\
        (\HIGHSEC,\LOWINT) & \text{if } \component \geq \mach(\SP) \\
        (\LOWSEC,\LOWINT) & \text{else} \\
      \end{cases}\]

      Above the stack pointer, memory is high integrity in an extension of safe initialization;
      from the perspective of the callee, that memory is not initialized. Below the old stack
      pointer, memory is high integrity and high confidentiality. Registers and the callee's frame
      are low integrity and low confidentiality. These categories inform our integrity and
      confidentiality properties.

    \paragraph{Eager Integrity}

      A system enjoys eager integrity with regard to \(\callmap\) if for any initial state \(\mach_0\),
      call state \(\mach_c\) where \(\callmap(\mach_c) = n\), and policy state \(\pol_c\), and
      return state \(\mpstate{\mach_r,\pol_r}\), where:
      \[\mach_0,\mathit{pinit}(\mach_0) \hookrightarrow ... \mpstate{\mach_c}{\pol_c,\_} ...
        \mpstate{\mach_r}{\pol_r,\_} | \ret{\mach_c}\]

      it is always the case that:
      \[\forall \component . \contour(\component) = \HIGHINT \rightarrow \mach_c(\component) = \mach_r(\component)\]

    \paragraph{Observable Integrity}

      Observable integrity is concerned with the changes a callee makes to high integrity components,
      and the effects those writes have on the observable behavior of its caller
      {\it after its return}. Given a contour, a call state, and a returned state, we define
      a ``rollback'' function \(\mathit{roll}_I : \CONTOURS \times \MACHS \times \MACHS
      \rightarrow \MACHS\). The rollback creates a state that matches the call
      state on those components that are high integrity in its contour, and matches the return state
      on low integrity components:

      \[\mathit{roll}_I(\contour,\mach_c,\mach_r)(\component) =
      \begin{cases}
        \mach_c(\component) & \contour(\component) = (\HIGHINT,\_) \\
        \mach_r(\component) & \contour(\component) = (\LOWINT,\_) \\
      \end{cases}\]

      Intuitively, the result of a rollback is an idealized state that reflects changes to accessible
      components, but ignores the callee's interference with the caller's data. Then the observable
      behavior of the resulting trace is the standard to which we expect any real trace to conform.

      A system enjoys observable integrity with regard to \(\callmap\) as follows:

      For any initial state \(\mach_0\), call state \(\mach_c\), and policy state \(\pol_c\) such that
      \(\mpstate{\mach_0}{\mathit{pinit}(\mach_0)} \hookrightarrow ... \mpstate{\mach_c}{\pol_c,\_} ...\)
      and \(\callmap(\mach_c) = n\), if \(\mpstate{\mach_c}{\pol_c} \hookrightarrow ...
        \mpstate{\mach_r}{\pol_r,\_} | \ret{\mach_c}\):

          \begin{itemize}
            \item Let \(\contour = \mathit{Cof}(\mach_c,n)\)
            \item Let \(\mach_r' = \mathit{roll}_I(\contour,\mach_c,\mach_r)\), \(\mpstate{\mach_r}{\pol_r}
              \hookrightarrow \MPT\), and \(\mach_r' \hookrightarrow \machT'\)
            \item \(\MPT \sim_P \machT'\)
          \end{itemize}
    
    \paragraph{Eager Confidentiality}

      A system enjoys eager stack confidentiality with regard to \(\callmap\) if, for any initial state
      \(\mach_0\) and call state \(\mach_c\) such that
      \(\mpstate{\mach_0}{\mathit{pinit}(\mach_0)} \hookrightarrow ... \mpstate{\mach_c}{\pol_c,\_} ...\) and
      \(\callmap(\mach_c) = n\):

      \begin{itemize}
        \item Let \(\contour = \mathit{Cof}(\mach_c,n)\)
        \item For all \(\mach_c'\) such that \(\mach_c \approx_\contour \mach_c'\):
        \item If \(\mpstate{\mach_c}{\pol_c} \hookrightarrow_P \MPT (\mach_r,p_r,\_) | \ret{\mach_c}\)
          \begin{itemize}
            \item Then there is some \(\mach_r'\) where
              
              \(\mach_c' \hookrightarrow \machT' (\mach_r',_) | \ret{\mach_c'}\)
            \item \(\mathit{Obs}(\MPT) \simeq \mathit{Obs}(\machT')\)
            \item And for all \(\component\), if \(\mach_c(\component) \not = \mach_r(\component)\) or
              \(\mach_c'(\component) \not = \mach_r'(\component)\), \(\mach_r(\component) = \mach_r'(\component)\)
          \end{itemize}
        \item Else, for \(\MPT\) and \(\machT'\) such that \(\mpstate{\mach_c}{\pol_c} \hookrightarrow \MPT\)
          and \(\mach_c' \hookrightarrow \machT'\), \(\MPT \sim_P \machT'\)
      \end{itemize}

    \paragraph{Observable Confidentiality}

      Observable confidentiality is concerned with whether high confidentiality data influences
      observable both behavior during and after a call. We define variants given the generalized notion
      of contours:
      \[\mach \approx_\contour \mach' \triangleq \forall \component .
        \contour(\component) = \LOWSEC \rightarrow \mach(\component) = \mach'(\component)\]

      Data that is high confidentiality within a callee due either to being in its caller's frame
      will become low confidentiality if the callee returns. We face a subtle distinction: 
      a callee treats the contents of its caller's frame as secrets, and they should not influence
      its behavior. Nor should the callee be able to influence behavior after its return based
      on the caller's secrets, for instance by copying them to registers that will later be output.
      But the caller itself may still adjust its behavior based on its own secrets. So our property
      relies on another rollback function to preserve changes made by the caller but remove other
      differences between variant states.

      In this case, a confidentiality rollback takes a contour and four states: two call states
      that will be variants, and two return states. It is not symmetric, regarding \(\mach_c\) and \(\mach_r\)
      as the ``real'' call and return, \(\mach_c'\) as a variant and \(\mach_r'\) as a return that follows it.
      \[\begin{split}
        & \mathit{roll}_C(\contour,\mach_c,\mach_c',\mach_r,\mach_r')(\component) = \\
        & \begin{cases}
          \mach_c(\component) & \text{if } C(\component) = \HIGHSEC \text{ and } \\
                  & \mach_c(\component) = \mach_r(\component) \text{ and }
                    \mach_c'(\component) = \mach_r'(\component) \\
          \mach_r'(\component) & \text{else} \\
        \end{cases}
      \end{split}\]

      A system enjoys observable confidentiality with respect to \(\callmap\) as follows:

      For any initial state \(\mach_0\), call state \(\mach_c\), and policy state \(\pol_c\) such that
      \(\mpstate{\mach_0}{\mathit{pinit}(\mach_0)} \hookrightarrow ... (\mach_c,\pol_c,\_) ...\) and
      \(\callmap(\mach_c) = n\):

      \begin{itemize}
        \item Let \(\contour = \mathit{Cof}(\mach_c,n)\)
        \item For all \(\mach_c'\) such that \(\mach_c \approx_C \mach_c'\):
        \item If \(\mpstate{\mach_c}{p_c} \hookrightarrow \MPT_1 (\mach_r,p_r,\_) | \ret{\mach_c}\)
          \begin{itemize}
            \item Then there is some \(\mach_r'\) where

              \(\mach_c' \hookrightarrow \machT_1' (\mach_r',\_) | \ret{\mach_c'}\)
            \item Let \(\mach_r'' = \mathit{roll}_C(\contour,\mach_c,\mach_r,\mach_c',\mach_r')\),

              \(\mpstate{\mach_r}{p_r} \hookrightarrow \MPT_2\), and \(\mach_r'' \hookrightarrow \machT_2'\)
            \item \(\MPT_1 \cdot \MPT_2 \sim_P \MPT_1' \cdot \MPT_2'\)
          \end{itemize}
        \item Else, for \(\MPT\) and \(\machT'\) such that \(\mpstate{\mach_c}{\pol_c} \hookrightarrow \MPT\)
          and \(\mach_c' \hookrightarrow \machT'\), \(\MPT \sim_P \machT'\)
      \end{itemize}


\end{document}
