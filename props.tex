\documentclass[conference]{IEEEtran}
\newif\ifdraft \drafttrue
\newif\iftext \textfalse
\newif\iflater \latertrue
\newif\ifaftersubmission \aftersubmissionfalse

% !!! PLEASE DON'T CHANGE THESE !!! INSTEAD DEFINE YOUR OWN texdirectives.tex !!!
\makeatletter \@input{texdirectives} \makeatother

%\IEEEoverridecommandlockouts
% The preceding line is only needed to identify funding in the first footnote. If that is unneeded, please comment it out.
\usepackage{cite}
\usepackage{amsmath,amssymb,amsfonts}
\usepackage{algorithmic}
\usepackage{hyperref}
\usepackage{graphicx}
\usepackage{textcomp}
\usepackage[capitalize]{cleveref}
\usepackage[inline]{enumitem}

\usepackage{xcolor}
\newcommand{\bcp}[1]{\ifdraft\textcolor{violet}{{[BCP:~#1]}}\fi}
\newcommand{\leo}[1]{\ifdraft\textcolor{teal}{{[LEO:~#1]}}\fi}
\newcommand{\apt}[1]{\ifdraft\textcolor{blue}{{[APT:~#1]}}\fi}
\newcommand{\rb}[1]{\ifdraft\textcolor{orange}{{[RB:~#1]}}\fi}
\newcommand{\sna}[1]{\ifdraft\textcolor{green}{{[SNA:~#1]}}\fi}
\newcommand{\COQ}[1]{\ifdraft\textcolor{red}{{[COQ DIFFERENCE:~#1]}}\fi}

\usepackage{listings}


\usepackage{xspace}
\newcommand{\cn}{\ifdraft\textsuperscript{\textcolor{blue}{[citation needed]}}\xspace\fi}

\makeatletter
\begingroup
\lccode`\A=`\-
\lccode`\N=`\N
\lccode`\V=`\V
\lowercase{\endgroup\def\memory@noval{ANoValue-}}
\long\def\memory@fiBgb\fi#1#2{\fi}
\long\def\memory@fiTBb\fi#1#2#3{\fi#2}
\newcommand\memory@ifnovalF[1]%>>=
  {%
    \ifx\memory@noval#1%
      \memory@fiBgb
    \fi
    \@firstofone
  }%=<<
\newcommand\memory@ifnovalTF[1]%>>=
  {%
    \ifx\memory@noval#1%
      \memory@fiTBb
    \fi
    \@secondoftwo
  }%=<<
\newcommand\memory@Oarg[2]%>>=
  {%
    \@ifnextchar[{\memory@Oarg@{#2}}{#2{#1}}%
  }%=<<
\long\def\memory@Oarg@#1[#2]%>>=
  {%
    #1{#2}%
  }%=<<
\newcommand*\memory@oarg%>>=
  {%
    \memory@Oarg\memory@noval
  }%=<<
\newcommand*\memory@ifcoloropt%>>=
  {%
    \@ifnextchar[\memory@ifcoloropt@true\memory@ifcoloropt@false
  }%=<<
\long\def\memory@ifcoloropt@true#1\memory@noval#2#3%>>=
  {%
    #2%
  }%=<<
\long\def\memory@ifcoloropt@false#1\memory@noval#2#3%>>=
  {%
    #3%
  }%=<<
\newlength\memory@width
\newlength\memory@height
\setlength\memory@width{23pt}
\setlength\memory@height{14pt}
\newcount\memory@num
\newcommand*\memory@blocks[2]%>>=
  {%
    \memory@num#1\relax
    \fboxsep-\fboxrule
    \memory@ifcoloropt#2\memory@noval
      {\def\memory@color{\textcolor#2}}
      {\def\memory@color{\textcolor{#2}}}%
    \loop
    \ifnum\memory@num>0
      \fbox{\memory@color{\rule{\memory@width}{\memory@height}}}%
      \kern-\fboxrule
      \advance\memory@num\m@ne
    \repeat
  }%=<<
% memory:
%  [#1]: width
%   #2 : count
%  [#3]: height
%   #4 : colour
%  [#5]: label
\newcommand*\memory%>>=
  {%
    \begingroup
    \memory@oarg\memory@a
  }%=<<
\newcommand*\memory@a[2]%>>=
  {%
    % #1 width
    % #2 count
    \memory@ifnovalF{#1}{\memory@width#1\relax}%
    \memory@Oarg\memory@height{\memory@b{#2}}%
  }%=<<
\newcommand*\memory@b[3]%>>=
  {%
    % #1 count
    % #2 height
    % #3 colour
    \memory@ifnovalF{#2}{\memory@height#2\relax}%
    \memory@oarg{\memory@c{#1}{#3}}%
  }%=<<
\newcommand*\memory@c[3]%>>=
  {%
    % #1 count
    % #2 colour
    % #3 label
    \memory@ifnovalTF{#3}
      {\ensuremath{\memory@blocks{#1}{#2}}}
      {\ensuremath{\underbrace{\memory@blocks{#1}{#2}}_{\text{#3}}}}%
    \endgroup
  }%=<<
\makeatother

\newcommand{\judgment}[2]{
  {\centering
  \vspace{\abovedisplayskip}
  \begin{tabular}{c}
    #1 \\
    \hline
    #2
  \end{tabular}
   \vspace{\abovedisplayskip}\par}}

\newcommand{\judgmentbr}[4]{
  {\centering
  \vspace{\abovedisplayskip}
  \begin{tabular}{c}
    #1 \\
    #2 \\
    #3 \\
    \hline
    #4
  \end{tabular}
   \vspace{\abovedisplayskip}\par}}


\newcommand{\judgmenttwo}[3]{
  {\centering
  \vspace{\abovedisplayskip}
  \begin{tabular}{c c}
    #1 & #2 \\
    \hline
    \multicolumn{2}{c}{#3}
  \end{tabular}
  \vspace{\abovedisplayskip}\par}}

\newcommand{\judgmentthree}[4]{
  {\centering
  \vspace{\abovedisplayskip}
  \begin{tabular}{c c c}
    #1 & #2 & #3 \\
    \hline
    \multicolumn{3}{c}{#4}
  \end{tabular}
  \vspace{\abovedisplayskip}\par}}

% Notational conventions
\newcommand{\HIGHSEC}{\textsc{HC}}
\newcommand{\LOWSEC}{\textsc{LC}}
\newcommand{\HIGHINT}{\textsc{HI}}
\newcommand{\LOWINT}{\textsc{LI}}
\newcommand{\IDS}{{\mathcal{I}}}
\newcommand{\ID}{I}
\newcommand{\ME}{\textsc{S}}
\newcommand{\NOTME}{\textsc{O}}
\newcommand{\TRANS}{\ensuremath{-}}
\newcommand{\JAL}{\ensuremath{\mathit{JAL}}}
\newcommand{\ACCYES}{\ensuremath{A}}
\newcommand{\ACCNO}{\ensuremath{I}}
\newcommand{\ACCCODE}{\ensuremath{K}}
\newcommand{\CRCALL}{\ensuremath{\mathit{CALL}}}
\newcommand{\CRRET}{\ensuremath{\mathit{RETURN}}}
\newcommand{\CRBOT}{\ensuremath{\bot}}
\newcommand{\VIS}{\textsc{vis}}
\newcommand{\HID}{\textsc{hid}}
\newcommand{\word}{w}
\newcommand{\addr}{a}
\newcommand{\WORDS}{{\mathcal W}}
\newcommand{\reg}{r}
\newcommand{\REGS}{{\mathcal R}}
\newcommand{\mach}{m}
\newcommand{\machT}{M}
\newcommand{\MACHS}{{\mathcal M}}
\newcommand{\MPT}{\mathit{MP}}
\newcommand{\obs}{o}
\newcommand{\obsT}{O}
\newcommand{\OBSS}{\mathit{Obs}}
\newcommand{\PC}[1]{\PCname(#1)}
\newcommand{\PCname}{\textsc{pc}}
\newcommand{\SP}{\textsc{sp}}
\newcommand{\pol}{p}
\newcommand{\POLS}{\mathcal{P}}
\newcommand{\pinit}{pinit}
\newcommand{\prop}{S}
\newcommand{\contour}{C}
\newcommand{\CONTOURS}{{\mathcal C}}
\newcommand{\component}{k}
\newcommand{\COMPONENTS}{{\mathcal K}}
\newcommand{\trace}{T}
\newcommand{\observer}{O}
\newcommand{\stateobs}{\sigma}
\newcommand{\seq}[1]{\overline{#1}}
\newcommand{\SEQ}[1]{\overline{#1}}
\newcommand{\dstk}[1]{{#1}.\mbox{\it stack}}
\newcommand{\dpcd}[1]{{#1}.\mbox{\it PCdepth}}
\newcommand{\ddep}[2]{{#1}.\mbox{\it depth}({#2})}
\newcommand{\dinit}{\mbox{\it Dinit}}
\newcommand{\empstack}{\mbox{\it empty}}
\newcommand{\access}[2]{\mbox{\it accessible}_{#1}({#2})}
\newcommand{\norm}[1]{\lvert{#1}\rvert}
\newcommand{\MPS}{\mathit{MPState}}
\newcommand{\mpstate}[2]{(#1,#2)}
\newcommand{\mpostate}[3]{(#1,#2,#3)}
\newcommand{\mpstatename}{mp}
\newcommand{\callmap}{cm}
\newcommand{\CALLMAPS}{\mathit{CallMap}}
\newcommand{\ret}[1]{\mathit{justret}\ #1}
\newcommand{\nextPC}{next}
\newcommand{\base}{b}
\newcommand{\stepsto}{\Longrightarrow}
\newcommand{\stepstounder}[1]{\stackrel{\mbox{\tiny{$#1$}}}{\Longrightarrow}}
\newcommand{\stepstounderfull}{\stepstounder{\textsc{RISCV}}}
\newcommand{\manystepsto}{\stepsto^\star}
\newcommand{\obstrace}{\mathit{obstrace}}
\newcommand{\funid}{f}
\newcommand{\FUNIDS}{\mathcal{F}}
\newcommand{\retmap}{\mathit{rm}}
\newcommand{\RETMAPS}{\mathit{RetMap}}
\newcommand{\codemap}{\mathit{fm}}
\newcommand{\CODEMAPS}{\mathit{FuncMap}}
\newcommand{\entmap}{\mathit{em}}
\newcommand{\ENTMAPS}{\mathit{EntryMap}}
\newcommand{\PUT}{\mathit{Until}}
\newcommand{\Trace}{T}
\newcommand{\traceelem}{a}
\newcommand{\TRACEELEMS}{A}
\newcommand{\head}{\mathit{head}}
\newcommand{\last}{\mathit{last}}

\newcommand{\stepstoobs}[1]{\xrightarrow{#1}}
\newcommand{\polstep}{\rightharpoonup}
\newcommand{\stepstopol}[1]{\overset{#1}{\rightharpoonup}}
%\newcommand{\stepstopol}[1]{\overset{#1}{\rightharpoonup}_P}

\newcommand{\stepplus}{\Rightarrow}
\newcommand{\stepkappa}{\Rightarrow_\kappa}
\newcommand{\induced}[2]{(#1, #2)^*}
\newcommand{\flows}{\sqsubseteq}
\newcommand{\flowsstrict}{\sqsubset}
\newcommand{\initmach}{\MACHS_{\mathit{init}}}
\newcommand{\initcontour}{\CONTOURS_{\mathit{init}}}
\newcommand{\closure}[1]{\textit{Close}#1}
\newcommand{\variant}[2]{\textit{Vars}(#1, #2)}
\newcommand{\isinf}{\mathit{inf}}

\newcommand{\Last}[1]{\mathit{Last}(#1)}

\newcommand{\HALT}{\textsc{HALT}}

\newcommand{\underscore}{\mbox{\_}}

\newcommand{\propdef}[1]{\text{\sc #1}}

\newcommand{\TRACE}[1]{\mathit{Trace}~(#1)}
\newcommand{\MTRACE}{\TRACE{\MACHS}}
\newcommand{\MOTRACE}{\TRACE{\MACHS \times \OBSS}}
\newcommand{\MPOTRACE}{\TRACE{\MACHS \times \POLS \times \OBSS}}


\usepackage{geometry}
\usepackage{amssymb,amsmath}

\newcommand{\MP}{\mathit{MP}}

\begin{document}

  \section{Preliminaries}

\apt{important tex note: adopt/adapt macros from macros.tex to make it easier to switch notations if desire later.}

    Values and addresses are words:
    \[a,w \in \mathcal{W}\]

    Components are words (addresses) and registers, including special registers:
    \[r \in \mathcal{R}\]
    \[k \in \mathcal{K} ::= \mathcal{W} | \mathcal{R} | \mathtt{PC} | \mathtt{SP}\]

\apt{Use set notation consistently rather than mixing in grammar notation ($::=$ and $|$).}

    A machine state maps components to values:
    \[m \in \mathcal{M} ::= \mathcal{K} \rightarrow \mathcal{W}\]

    Our properties are defined in terms of observations, which may be silent or a word of data:
    \[o \in \mathit{Obs} ::= \tau | w \in \mathcal{W}\]
    
    Execution is by a step function \(m \xrightarrow{o} m' : m \in \mathcal{M} \rightarrow
    o \in \mathit{Obs} \times m' \in \mathcal{M}\).\apt{Very awkward: don't repeat notation conventions in type.}

    A policy tracks a mix of static and dynamic enforcement.\apt{??} A policy consists of
    a set of policy states, \(p \in \mathcal{P}\), a policy initializer
    \(pinit : \mathcal{M} \rightharpoonup \mathcal{P}\), and a policy-parameterized step function:
    \[\begin{split}
      m, p \xrightarrow{o}_P m', p' : m \in \mathcal{M} \times p \in \mathcal{P} & \rightharpoonup \\
      o \in \mathit{Obs} \times m' \in \mathcal{M} \times p' \in \mathcal{P} &
    \end{split}\]
    \apt{I would suggest parenthesizing the pairs $(m,p)$ unless that looks too busy. (This sort of experiment is exactly why macros are a good idea.)}

    Axiomatically, if \(m,p \xrightarrow{o}_P m',p'\), then \(m \xrightarrow{o} m'\).
    \apt{Why not follow coq development by defining policy step function $\mathcal{M} \times \mathcal{P} \rightarrow \mathcal{P}$ separately and then \emph{defining}
      combined function $\mathcal{M} \times \mathcal{P} \rightarrow \mathcal{M} \times \mathcal{P}$.}
    The initializer is a partial function from machine states to policy
    states, reflecting that a policy may apply static constraints on the initial state. The parameterized
    step relation is likewise partial to allow for fail-stop behavior.

  \paragraph{Machine, Policy, and Observation Traces}

    A machine trace \(M\) is a sequence of machine states \(m_0 m_1 m_2 \dots\) such that each
    pair of adjacent states are related by \(\longrightarrow\) with some observation. An
    {\it mptrace} is similarly a sequence of machine state-policy state pairs,
    \((m_0,\mathit{pinit}(m_0))(m_1,p_1) \dots\) \apt{why does it need to start with $\mathit{pinit}$?}
    such that adjacent pairs of pairs are related
    by \(\longrightarrow_P\). Steps being deterministic\apt{awkward}, we can relate an initial state to its trace
    via \(m \hookrightarrow M : m \in \mathcal{M} \rightarrow M \in \mathit{trace}\ \mathcal{M}\):

    \[\frac{m_0 \xrightarrow{o} m_1 \hspace{.5cm} m_1 \hookrightarrow M}
           {m_0 \hookrightarrow m_0 M}\]

    \apt{Note that this is a coinductive relation.}           

    Since the step function is total, if \(m \hookrightarrow M\), then \(M\) must be infinite.
    We will also wish to take a prefix of the trace from an initial state up to when some condition
    holds\apt{``up through the state where some first condition first holds'' (?)}.
    Let \(f\) be a predicate on machine states. Then we define \(m \hookrightarrow M | f\)
    (read ``\(M\) is the prefix from \(m\) until \(f\)'') where \(f\) may bind\apt{wrong word, and this phrase is redundant anyhow} states in \(M\):
    \[\frac{m_0 \xrightarrow{o} m_1 \hspace{.5cm} \neg f\ m_1 \hspace{.5cm} m_1 \hookrightarrow M | f }
           {m_0 \hookrightarrow m_0 M | f}\]
    \[\frac{m_0 \xrightarrow{o} m_1 \hspace{.5cm} f\ m_1}
           {m_0 \hookrightarrow m_0 m_1}\]

           \apt{What if $f$ holds on $m_0$ ? }

    We similarly define \(\hookrightarrow_P\), writing \(m_0,\mathit{pinit}(m_0) \longrightarrow \bot\) when
    there is no \(m_1,p_1\) such that \(m_0,\mathit{pinit}(m_0) \xrightarrow{o} m_1,p_1\) for any \(o\).
    \apt{What does $\mathit{pinit}$ have to do with it?. In any case, suggest using $\nrightarrow$ or similar.}
    \[\frac{m_0,p_0 \xrightarrow{o} m_1,p_1 \hspace{.5cm} m_1,p_1 \hookrightarrow_P \MP}
           {m_0,p_0 \hookrightarrow_P m_0,p_0 \MP}\]
    \[\frac{m_0,p_0 \longrightarrow \bot}
           {m_0,p_0 \hookrightarrow_P m_0,p_0}\]

    With a policy we may see \(m,p \hookrightarrow \MP\) where \(\MP\) is finite due to a policy
    failure. We will identify the set of infinite traces \apt{over what? Don't think set is the right notion here.}
    as \(\top\), so \(\MP \in \top\) will
    indicate an infinite trace and \(\MP \not \in \top\) a finite one.
    We define a similar notion of taking the prefix until a condition:

    \[\frac{m_0,p_0 \xrightarrow{o} m_1,p_1 \hspace{.5cm} \neg f\ m_1 \hspace{.5cm} m_1,p_1
              \hookrightarrow_P M | f }
           {m_0,p_0 \hookrightarrow_P m_0,p_0 M | f}\]
    \[\frac{m_0,p_0 \xrightarrow{o} m_1,p_1 \hspace{.5cm} f\ m_1}
           {m_0,p_0 \hookrightarrow_P (m_0,p_0) (m_1,p_1)}\]

    When we've taken a prefix and want to combine it with another trace by replacing its final
    state and continuing execution, we use the join operator on any type of trace \(T_1 \cdot T_2 :
    \mathit{trace}\ X \rightarrow \mathit{trace}\ X \rightharpoonup \mathit{trace}\ X\), defined
    on \(T_1 \not \in \top\) and any \(T_2\):

    \apt{If we're defining $\mathit{trace}$ generically, why not do so earlier to avoid multiple definitions of $\hookrightarrow$ etc.?}
    \[T_1 \cdot T_2 \triangleq
      \begin{cases}
        t T_2 & \text{if } T_1 = t \\
        t T_1' \cdot T_2 & \text{if } T_1 = t T_1' \\
      \end{cases}\]

      \apt{But actually, the coq definition is defined for all $T_1$ (even if that isn't very useful). And isn't it important to let $T_2$ be empty?}
    An observation trace \(O\) is the trace of observations from a machine or {\it mptrace},
    as defined by \(\mathit{Obs}\):
    \[\frac{M = mM' \hspace{.5cm} m \xrightarrow{o} m'}
           {\mathit{Obs}(M) = o\mathit{Obs}(M')}\]
    \[\frac{M = m}
           {\mathit{Obs}(M) = \tau}\]
    \[\frac{\MP = (m,p)\MP' \hspace{.5cm} m,p \xrightarrow{o} m',p'}
           {\mathit{Obs}(\MP) = o \mathit{Obs}(\MP')}\]
    \[\frac{\MP = m,p}
           {\mathit{Obs}(M) = \tau}\]

    An observational prefix \(O_1 \lesssim O_2\) means that \(O_1\) is a prefix of \(O_2\) up to the deletion
    of \(\tau\) observations:
    \[\begin{split}
      O_1 & \lesssim O_2 \triangleq \forall o . o \in O_1 \Rightarrow o = \tau \lor \\
      & \exists O_1' . O_1 = \tau O_1' \land O_1' \lesssim O_2 \lor \\
      & \exists O_2' .  O_2 = \tau O_2' \land O_1 \lesssim O_2' \lor \\
      & \exists O_1' O_2' . O_1 = w O_1' \land O_2 = w O_2' \land O_1' \lesssim O_2' \\
    \end{split}\]
    
    \apt{This is not quite the same as the coq definition. Is it really equivalent?}

    Equivalence of observation traces is defined as traces prefixing each other:
    \[O_1 \simeq O_2 \triangleq O_1 \lesssim O_2 \land O_2 \lesssim O_1\]

    \apt{Ditto.}
    
    We define policy-sensitive observational equivalence, which relates a machine trace and
    an {\it mptrace} whose observable behavior match, up to a possible halt due to policy failure.

    \[\frac{\MP \not \in \top \hspace{.5cm} \mathit{Obs}(\MP) \lesssim \mathit{Obs}(M)}
           {\MP \sim_P M}\]
    
    \[\frac{\MP \in \top \hspace{.5cm} \mathit{Obs}(\MP) \simeq \mathit{Obs}(M)}
           {\MP \sim_P M}\]
    
           \apt{Does this notion appear in Coq? It seems potentially useful, but it seems to require
             knowing whether we are in the finite or infinite case.}

  \paragraph{Calling Conventions and Contours}

      A subroutine call with \(n\) arguments follows the following convention:

      \begin{itemize}
        \item The stack pointer is increased by \(n+1\)
        \item A {\tt Jal} occurs
        \item The return address is stored at \(\mathtt{SP} - 1\)
      \end{itemize}

      \apt{But the \emph{properties} don't need these notions!}

      The final state of a call sequence \apt{what's that ??} is a {\it call state}. If \(M\) is a call sequence with
      we map it \apt{??} to its call state, number of arguments, and the PC and SP values of its eventual
      return by the function \(\mathit{call} : \mathit{trace}\ \mathcal{M} \rightharpoonup \mathcal{M}
      \times \mathbb{N} \times \mathcal{W} \times \mathcal{W}\). We can also track a call sequence in the
      policy state, defining \(\mathit{call}_P : \mathcal{M} \times \mathcal{P} \rightharpoonup \mathbb{N}
      \times \mathcal{W} \times \mathcal{W}\) such that:

      \[\frac{\begin{split}
        m_0,\mathit{pinit}(m_0) & \hookrightarrow_P \dots m,p \dots \\ % \hspace{.5cm}
        m_0 & \hookrightarrow \dots M \dots \\
        \mathit{call}(M) & = m, n, a_{pc}, a_{sp}\end{split}}
             {\mathit{call}_P(m,p) = n, a_{pc}, a_{sp}}\]

      A state is considered to have returned after a call if the stack pointer is the same as before the call
      (\(a_{sp}\)) and the program counter is the next address after (\(a_{pc}\)):
      \[\mathit{ret}\ m\ a_{pc}\ a_{sp} \triangleq m(\mathtt{SP}) = a_{sp} \land m(\mathtt{PC}) = a_{pc}\]

  \section{Observable Stack Safety}

    {\it Observable stack safety} is a property of a transition system \(\longrightarrow_p\),
    which is decomposed into three subproperties: {\it observable stack integrity},
    {\it observable stack confidentiality}, and {\it observable safe initialization}. 

  %  We unify these properties with a structure
  %  called a {\it contour} that maps components to levels of confidentiality and integrity.
  %  Specifically, we have confidentiality labels \(\mathit{labelC} ::= \{\mathtt{HC}, \mathtt{LC}\}\)
  %  and integrity labels \(\mathit{labelI} ::= \{\mathtt{HI}, \mathtt{LI}\}\), and a
  %  type \(\mathit{contour}\)

    \paragraph{Observable Safe Initialization}

      In the initial state of a program, nothing has yet been allocated on the stack, and
      there is no guarantee of the initial value of memory in the stack. So there is no
      need to protect a caller's stack data, but reading uninitialized data is unsafe.

      Let \(C : \mathcal{K} \rightarrow \{\mathtt{HC},\mathtt{LC}\}\) map all addresses
      to \(\mathtt{HC}\) (high confidentiality) and all registers to \(\mathtt{LC}\)
      (low confidentiality). In the style of non-interference, a pair of states \(m\)
      and \(m'\) are variants with regard to \(C\), \(m \approx_C m'\), if:
      \[\forall k . C(k) = \mathtt{LC} \rightarrow m(k) = m'(k)\]
      
      A system enjoys {\it observable safe initialization} if for any initial states \(m_0\)
      and \(m_0'\) such that \(m_0 \approx_C m_0'\):

      \begin{itemize}
        \item Let \(m_0,\mathit{pinit}(m_0) \hookrightarrow \MP\) and \(m_0' \hookrightarrow M'\)
        \item \(\MP \sim M'\)
      \end{itemize}

    \paragraph{Contours}

      Our remaining properties concern the behavior of function calls and require a dynamic treatment
      of confidentiality and integrity. We generalize \(C\) above to a {\it contour} mapping each component
      to a label:

      \[\mathit{label} ::= \{\mathtt{HC},\mathtt{LC}\} \times \{\mathtt{HI},\mathtt{LI}\}\]

      \[C \in \mathcal{C} ::= \mathcal{K} \rightarrow \mathit{label}\]

      When \(m,p\) is a call state \(\mathit{call}_P(m,p) = n,\_,\_\) for some \(n\),
      we can construct a contour based on the size of \(m\)'s stack frame, as follows:
      \[\mathit{Cof}(m,n)(k) =
      \begin{cases}
        (\mathtt{HC},\mathtt{HI}) & \text{if } k \leq m(\mathtt{SP}) - n \\
        (\mathtt{HC},\mathtt{LI}) & \text{if } k \geq m(\mathtt{SP}) \\
        (\mathtt{LC},\mathtt{HI}) & \text{if } k = m(\mathtt{SP}) - 1 \\
        (\mathtt{LC},\mathtt{LI}) & \text{else} \\
      \end{cases}\]

      Above the stack pointer, memory is high integrity in an extension of safe initialization;
      from the perspective of the callee, that memory is not initialized. Below the old stack
      pointer, memory is high integrity and high confidentiality. The top of the callee's frame
      is its return address, which it must be able to read in order to return but which is
      high integrity [does this make sense?] Registers and the callee's frame are low integrity
      and low confidentiality. These categories inform our integrity and confidentiality properties.

    \paragraph{Observable Integrity}

      Observable integrity is concerned with the writes a call makes to high integrity components,
      and the effects those writes have on the observable behavior of its caller
      {\it after its return}. Given a contour, a call state, and a returned state, we define
      a ``rollback'' function \(\mathit{roll}_I : \mathcal{C} \times \mathcal{M} \times \mathcal{M}
      \rightarrow \mathcal{M}\). The rollback creates a state that matches the call
      state on those components that are high integrity in \(C\), and matches the return state
      on low integrity components:

      \[\mathit{roll}_I(C,m_c,m_r)(k) =
      \begin{cases}
        m_c(k) & C(k) = (\mathtt{HI},\_) \\
        m_r(k) & C(k) = (\mathtt{LI},\_) \\
      \end{cases}\]

      Intuitively, the result of a rollback is an idealized state that reflects changes to accessible
      components, but ignores the callee's interference with the caller's data. Then the observable
      behavior of the resulting trace is the standard to which we expect any real trace to conform.

      A system enjoys observable integrity as follows:

      For any initial state \(m_0\), call state \(m_c\), and policy state \(p_c\) such that
      \((m_0,\mathit{pinit}(m_0)) \hookrightarrow ... (m_c,p_c) ...\)
      and \(\mathit{call}_P(m_c,p_c) = n,a_{pc},a_{sp}\), if 
      \((m_c,p_c) \hookrightarrow ... (m_r,p_r) | \mathit{ret}\ m_r\ a_{pc}\ a_{sp}\):

          \begin{itemize}
            \item Let \(C = \mathit{Cof}(m_c,n)\)
            \item Let \(m_r' = \mathit{roll}_I(C,m_c,m_r)\), \((m_r,p_r) \hookrightarrow \MP\), and
              \(m_r' \hookrightarrow M'\)
            \item \(\MP \sim_P M'\)
          \end{itemize}

    \paragraph{Eager Integrity}
 
      An eager enforcement mechanism implements a stronger property, which we term {\it eager integrity}.
      Eager integrity simplifies observable integrity by requiring that high integrity data of \(m_r\) be
      identical to that of \(m_c\). Formally, a system enjoys eager integrity if for any initial state
      \(m_0\), call state \(m_c\), and policy state \(p_c\) such that \(\mathit{call}(m_c,p_c) =
      n, a_{pc}, a_{sp}\), and some \(m_r\) and \(p_r\) such that
      \(m_0,\mathit{pinit}(m_0) \hookrightarrow ... (m_c,p_c) ... (m_r,p_r) | \mathit{ret}\ m_r\ a_{pc}\ a_{sp}\):

      \[\forall k . C(k) = \mathtt{HI} \rightarrow m_c(k) = m_r(k)\]

      This implies lazy integrity, as the rollback function will produce \(m_r' = m_r\).
 
    \paragraph{Observable Confidentiality}

      Observable confidentiality is concerned with whether high confidentiality data influences
      observable behavior during and after a call. We define variants given the generalized notion
      of contours:
      \[m \approx_C m' \triangleq \forall k . C(k) = \mathtt{LC} \rightarrow m(k) = m'(k)\]

      Data that is high confidentiality within a callee due either to being in its caller's frame
      will become low confidentiality if the callee returns. We face a subtle distinction: 
      a callee treats the contents of its caller's frame as secrets, and they should not influence
      its behavior. Nor should the callee be able to influence behavior after its return based
      on the caller's secrets, for instance by copying them to registers that will later be output.
      But the caller itself may still adjust its behavior based on its own secrets. So our property
      relies on another rollback function to preserve changes made by the caller but remove other
      differences between variant states.

      In this case, a confidentiality rollback takes a contour and four states: two call states
      that will be variants, and two return states. It is not symmetric, regarding \(m_c\) and \(m_r\)
      as the ``real'' call and return, \(m_c'\) as a variant and \(m_r'\) as a return that follows it.
      \[\begin{split}
        & \mathit{roll}_C(C,m_c,m_c',m_r,m_r')(k) = \\
        & \begin{cases}
          m_c(k) & \text{if } C(k) = \mathtt{HC} \text{ and } \\
                  & m_c(k) = m_r(k) \text{ and } m_c'(k) = m_r'(k) \\
          m_r'(k) & \text{else} \\
        \end{cases}
      \end{split}\]

      A system enjoys observable confidentiality as follows:

      For any initial state \(m_0\), call state \(m_c\), and policy state \(p_c\) such that
      \(m_0,\mathit{pinit}(m_0) \hookrightarrow ... (m_c,p_c) ...\) and
      \(\mathit{call}(m_c,p_c) = n, a_{pc}, a_{sp}\):

      \begin{itemize}
        \item Let \(C = \mathit{Cof}(m_c,n)\)
        \item For all \(m_c'\) such that \(m_c \approx_C m_c'\):
        \item If \((m_c,p_c) \hookrightarrow \MP_1 (m_r,p_r) | \mathit{ret}\ m_r\ a_{pc}\ a_{sp}\)
          \begin{itemize}
            \item Then there is some \(m_r'\) where

              \(m_c' \hookrightarrow M_1' m_r' | \mathit{ret}\ m_r'\ a_{pc}\ a_{sp}\)
            \item Let \(m_r'' = \mathit{roll}_C(C,m_c,m_r,m_c',m_r')\),

              \((m_r,p_r) \hookrightarrow \MP_2\), and \(m_r'' \hookrightarrow M_2'\)
            \item \(\MP_1 \cdot \MP_2 \sim_P M_1' \cdot M_2'\)
          \end{itemize}
        \item Else, for \(\MP\) and \(M'\) such that \(m_c,p_c \hookrightarrow \MP\) and \(m_c' \hookrightarrow M'\),
          \(\MP \sim_P M'\)
      \end{itemize}

    \paragraph{Eager Confidentiality}

      Once again an eager policy enforces a stronger property that only concerns the behavior of
      the original trace and its variant within each call. A system enjoys {\it eager stack
      confidentiality} if, for any initial state \(m_0\) and call state \(m_c\) such that
      \(m_0,\mathit{pinit}(m_0) \hookrightarrow ... (m_c,p_c) ...\) and
      \(\mathit{call}(m_c,p_c) = n, a_{pc}, a_{sp}\):

      \begin{itemize}
        \item Let \(C = \mathit{Cof}(m_c,n)\)
        \item For all \(m_c'\) such that \(m_c \approx_C m_c'\):
        \item If \((m_c,p_c) \hookrightarrow_P \MP (m_r,p_r) | \mathit{ret}\ m_r\ a_{pc}\ a_{sp}\)
          \begin{itemize}
            \item Then there is some \(m_r'\) where
              
              \(m_c' \hookrightarrow M m_r' | \mathit{ret}\ m_c'\ m_r'\)
            \item \(\mathit{Obs}(\MP) \simeq \mathit{Obs}(M)\)
            \item And for all \(k\), if \(m_c(k) \not = m_r(k)\) or \(m_c'(k) \not = m_r'(k)\),
              \(m_r(k) = m_r'(k)\)
          \end{itemize}
        \item Else, for \(\MP\) and \(M'\) such that \(m_c,p_c \hookrightarrow \MP\) and \(m_c' \hookrightarrow M'\),
          \(\MP \sim_P M'\)
      \end{itemize}

      Eager confidentiality implies observable confidentiality, because if \(m_r\) and \(m_r'\) agree
      on all \(k\) that changed during the call, then an \(m_r'' = \mathit{roll}_C(C,m_c,m_r,m_c',m_r')\)
      will be identical to \(m_r\), and future observations will be the same up to a potential policy
      stop.
 
\end{document}
