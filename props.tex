\documentclass{article}

\usepackage{geometry}
\usepackage{amssymb,amsmath}

\begin{document}

  \section{Preliminaries}

    Values and addresses are words:
    \[a,w \in \mathcal{W}\]

    Components are words (addresses) and registers, including special registers:
    \[r \in \mathcal{R}\]
    \[k \in \mathcal{K} ::= \mathcal{W} | \mathcal{R} | \mathtt{PC} | \mathtt{SP}\]

    A machine state maps components to values:
    \[m \in \mathcal{M} ::= \mathcal{K} \rightarrow \mathcal{W}\]

    Our properties are defined in terms of observations, which may be silent or a word of data:
    \[o \in \mathit{Obs} ::= \tau | w \in \mathcal{W}\]
    
    Execution is by a step function \(m \xrightarrow{o} m' : m \in \mathcal{M} \rightarrow
    o \in \mathit{Obs} \times m' \in \mathcal{M}\).

    A policy tracks a mix of static and dynamic enforcement. A policy consists of
    a set of policy states, \(p \in \mathcal{P}\), a policy initializer
    \(pinit : \mathcal{M} \rightharpoonup \mathcal{P}\), and a policy-parameterized step function:
    \[m, p \xrightarrow{o}_P m', p' : m \in \mathcal{M} \times p \in \mathcal{P} \rightharpoonup
    o \in \mathit{Obs} \times m' \in \mathcal{M} \times p' \in \mathcal{P}\]

    Axiomatically, if \(m,p \xrightarrow{o}_P m',p'\), then \(m \xrightarrow{o} m'\).
    The initializer is a partial function from machine states to policy
    states, reflecting that a policy may apply static constraints on the initial state. The parameterized
    step relation is likewise partial to allow for fail-stop behavior.

  \paragraph{Machine, Policy, and Observation Traces}

    A machine trace \(M\) is a sequence of machine states \(m_0 m_1 m_2 \dots\) such that each
    pair of adjacent states are related by \(\longrightarrow\) with some observation. An
    {\it mptrace} is similarly a sequence of machine state-policy state pairs,
    \((m_0,\mathit{pinit}(m_0))(m_1,p_1) \dots\) such that adjacent pairs of pairs are related
    by \(\longrightarrow_P\). Steps being deterministic, we can relate an initial state to its trace
    via \(\cdot \hookrightarrow \cdot : \mathcal{M} \rightarrow \mathit{trace}\ \mathcal{M}\):

    \[\frac{m_0 \xrightarrow{o} m_1 \hspace{.5cm} m_1 \hookrightarrow M}
           {m_0 \hookrightarrow m_0 M}\]

    Since the step function is total, if \(m \hookrightarrow M\), then \(M\) must be infinite.
    We will also wish to take a prefix of the trace from an initial state up to when some condition
    holds. Let \(f\) be a predicate on machine states. Then we define \(m \hookrightarrow M | f\)
    (read ``\(M\) is the prefix from \(m\) until \(f\)'') by:

    \begin{minipage}[b]{.45\textwidth}
      \[\frac{m_0 \xrightarrow{o} m_1 \hspace{.5cm} \neg f\ m_1 \hspace{.5cm} m_1 \hookrightarrow M | f }
             {m_0 \hookrightarrow m_0 M | f}\]
    \end{minipage}
    \begin{minipage}[b]{.45\textwidth}
      \[\frac{m_0 \xrightarrow{o} m_1 \hspace{.5cm} f\ m_1}
             {m_0 \hookrightarrow m_0 m_1}\]
    \end{minipage}
    \vspace{\belowdisplayskip}

    We similarly define \(\hookrightarrow_P\), writing \(m_0,\mathit{pinit}(m_0) \longrightarrow \bot\) when
    there is no \(m_1,p_1\) such that \(m_0,\mathit{pinit}(m_0) \xrightarrow{o} m_1,p_1\) for any \(o\).
    
    \begin{minipage}[b]{.45\textwidth}
      \[\frac{m_0,p_0 \xrightarrow{o} m_1,p_1 \hspace{.5cm} m_1,p_1 \hookrightarrow_P MP}
             {m_0,p_0 \hookrightarrow_P m_0,p_0 MP}\]
    \end{minipage}
    \begin{minipage}[b]{.45\textwidth}
      \[\frac{m_0,p_0 \longrightarrow \bot}
             {m_0,p_0 \hookrightarrow_P m_0,p_0}\]
    \end{minipage}
    \vspace{\belowdisplayskip}

    With a policy we may see \(m,p \hookrightarrow MP\) where \(MP\) is finite due to a policy
    failure. We will identify the set of infinite traces as \(\top\), so \(MP \in \top\) will
    indicate an infinite trace and \(MP \not \in \top\) a finite one.
    We define a similar notion of taking the prefix until a condition:

    \begin{minipage}[b]{.45\textwidth}
      \[\frac{m_0,p_0 \xrightarrow{o} m_1,p_1 \hspace{.5cm} \neg f\ m_1 \hspace{.5cm} m_1,p_1
                \hookrightarrow_P M | f }
             {m_0,p_0 \hookrightarrow_P m_0,p_0 M | f}\]
    \end{minipage}
    \begin{minipage}[b]{.45\textwidth}
      \[\frac{m_0,p_0 \xrightarrow{o} m_1,p_1 \hspace{.5cm} f\ m_1}
             {m_0,p_0 \hookrightarrow_P (m_0,p_0) (m_1,p_1)}\]
    \end{minipage}
    \vspace{\belowdisplayskip}

    An observation trace \(O\) is the trace of observations from a machine or {\it mptrace},
    as defined by \(\mathit{ObsOf}\):
    
    \begin{minipage}[b]{.45\textwidth}
      \[\frac{M = mM' \hspace{.5cm} m \xrightarrow{o} m' \hspace{.5cm} \mathit{ObsOf}(M') = O}
             {\mathit{ObsOf}(M) = oO}\]
    \end{minipage}
    \begin{minipage}[b]{.45\textwidth}
      \[\frac{M = m}
             {\mathit{ObsOf}(M) = \tau}\]
    \end{minipage}
    \vspace{\belowdisplayskip}

    \begin{minipage}[b]{.45\textwidth}
      \[\frac{MP = (m,p)MP' \hspace{.5cm} m,p \xrightarrow{o} m',p' \hspace{.5cm} \mathit{ObsOf}(MP') = O}
             {\mathit{ObsOf}(MP) = oO}\]
    \end{minipage}
    \begin{minipage}[b]{.45\textwidth}
      \[\frac{MP = m,p}
             {\mathit{ObsOf}(M) = \tau}\]
    \end{minipage}
    \vspace{\belowdisplayskip}

    We abbreviate the observation trace of the execution trace from a state as:
    \[m \rightsquigarrow O \triangleq m \hookrightarrow M \land \mathit{ObsOf}(M) = O\]
    \[m,p \rightsquigarrow O \triangleq m,p \hookrightarrow_P MP \land \mathit{ObsOf}(MP) = O\]

    An observational prefix \(O_1 \lesssim O_2\) means that \(O_1\) is a prefix of \(O_2\) up to the deletion
    of \(\tau\) observations:

    \[O_1 \sim O_2 \triangleq
      \begin{cases}
        O_1' \sim O_2 & \text{if } O_1 = \tau O_1' \\
        O_1 \sim O_2' & \text{if } O_2 = \tau O_2' \\
        O_1' \sim O_2' & \text{if } O_1 = w O_1' \text{ and } O_2 = w O_2' \\
        \text{True} & \text{if for all } o \text{ in } O_1, o = \tau \\
      \end{cases}\]

    Observational equivalence is defined as a symmetric prefix:
    \[O_1 \simeq O_2 \triangleq O_1 \lesssim O_2 \land O_2 \lesssim O_1\]

  \paragraph{Calling Conventions and Contours}

      A subroutine call with \(n\) arguments follows the following convention:

      \begin{itemize}
        \item The stack pointer is increased by \(n+1\)
        \item A {\tt Jal} occurs
        \item The return address is stored at \(\mathtt{SP} - 1\)
      \end{itemize}

      The final state of a call sequence is a {\it call state}. If \(M\) is a call sequence with
      we map it to its call state, number of arguments, and the PC and SP values of its eventual
      return by the function \(\mathit{call} : \mathit{trace}\ \mathcal{M} \rightharpoonup \mathcal{M}
      \times \mathbb{N} \times \mathcal{W} \times \mathcal{W}\). We can also track a call sequence in the
      policy state, defining \(\mathit{call}_P : \mathcal{M} \times \mathcal{P} \rightharpoonup \mathbb{N}
      \times \mathcal{W} \times \mathcal{W}\) such that:

      \[\frac{m_0,\mathit{pinit}(m_0) \hookrightarrow_P \dots m,p \dots \hspace{.5cm}
              m_0 \hookrightarrow \dots M \dots \hspace{.5cm} \mathit{call}(M) = m, n, a_{pc}, a_{sp}}
             {\mathit{call}_P(m,p) = n, a_{pc}, a_{sp}}\]

      A state is considered to have returned after a call if the stack pointer is the same as before the call
      (\(a_{sp}\)) and the program counter is the next address after (\(a_{pc}\)):
      \[\mathit{returned}\ m\ a_{pc}\ a_{sp} \triangleq m'(\mathtt{SP}) = a_{sp} \land m(\mathtt{PC}) = a_{pc}\]

  \section{Observable Stack Safety}

    {\it Observable stack safety} is a property of a transition system \(\longrightarrow_p\),
    which is decomposed into three subproperties: {\it observable stack integrity},
    {\it observable stack confidentiality}, and {\it observable safe initialization}. 

  %  We unify these properties with a structure
  %  called a {\it contour} that maps components to levels of confidentiality and integrity.
  %  Specifically, we have confidentiality labels \(\mathit{labelC} ::= \{\mathtt{HC}, \mathtt{LC}\}\)
  %  and integrity labels \(\mathit{labelI} ::= \{\mathtt{HI}, \mathtt{LI}\}\), and a
  %  type \(\mathit{contour}\)

    \paragraph{Observable Safe Initialization}

      In the initial state of a program, nothing has yet been allocated on the stack, and
      there is no guarantee of the initial value of memory in the stack. So there is no
      need to protect a caller's stack data, but reading uninitialized data is unsafe.

      Let \(C : \mathcal{K} \rightarrow \{\mathtt{HC},\mathtt{LC}\}\) map all addresses
      to \(\mathtt{HC}\) (high confidentiality) and all registers to \(\mathtt{LC}\)
      (low confidentiality). In the style of non-interference, a pair of states \(m\)
      and \(m'\) are variants with regard to \(C\), \(m \approx_C m'\), if:
      \[\forall k . C(k) = \mathtt{LC} \rightarrow m(k) = m'(k)\]
      
      A system enjoys {\it observable safe initialization} if for any initial states \(m_0\)
      and \(m_0'\) such that \(m_0 \approx_C m_0'\):

      \begin{itemize}
        \item Let \(m_0,\mathit{pinit}(m_0) \rightsquigarrow O\) and \(m_0' \rightsquigarrow O'\)
        \item If \(m_0,\mathit{pinit}(m_0) \hookrightarrow_P MP \in \top\) then \(O \simeq O'\)
        \item Otherwise \(O \lesssim O'\)
      \end{itemize}

    \paragraph{Contours}

      Our remaining properties concern the behavior of function calls and require a dynamic treatment
      of confidentiality and integrity. We generalize \(C\) above to a {\it contour} mapping each component
      to a label:

      \[\mathit{label} ::= \{\mathtt{HC},\mathtt{LC}\} \times \{\mathtt{HI},\mathtt{LI}\}\]

      \[C \in \mathcal{C} ::= \mathcal{K} \rightarrow \mathit{label}\]

      When \(m,p\) is a call state \(\mathit{call}_P(m,p) = n,\_,\_\) for some \(n\),
      we can construct a contour based on the size of \(m\)'s stack frame, as follows:
      \[Cof(m,n)(k) =
      \begin{cases}
        (\mathtt{HC},\mathtt{HI}) & \text{if } k \in \mathcal{W}
                                    \text{ and } k \leq m(\mathtt{SP}) - n \\
        (\mathtt{HC},\mathtt{LI}) & \text{if } k \in \mathcal{W}
                                    \text{ and } k \geq m(\mathtt{SP}) \\
        (\mathtt{LC},\mathtt{HI}) & \text{if } k \in \mathcal{W}
                                    \text{ and } k = m(\mathtt{SP}) - 1 \\
        (\mathtt{LC},\mathtt{LI}) & \text{else} \\
      \end{cases}\]

      Above the stack pointer, memory is high integrity in an extension of safe initialization;
      from the perspective of the callee, that memory is not initialized. Below the old stack
      pointer, memory is high integrity and high confidentiality. The top of the callee's frame
      is its return address, which it must be able to read in order to return but which is
      high integrity [does this make sense?] Registers and the callee's frame are low integrity
      and low confidentiality. These categories inform our integrity and confidentiality properties.

    \paragraph{Observable Integrity}

      Observable integrity is concerned with the writes a call makes to high integrity components,
      and the effects those writes have on the observable behavior of its caller
      {\it after its return}. Given a contour, a call state, and a returned state, we define
      a ``rollback'' function \(\mathit{roll}_I : \mathcal{C} \times \mathcal{M} \times \mathcal{M}
      \rightarrow \mathcal{M}\). The rollback creates a state that matches the call
      state on those components that are high integrity in \(C\), and matches the return state
      on low integrity components:

      \[\mathit{roll}_I(C,m_c,m_r)(k) =
      \begin{cases}
        m_c(k) & C(k) = (\mathtt{HI},\_) \\
        m_r(k) & C(k) = (\mathtt{LI},\_) \\
      \end{cases}\]

      Intuitively, the result of a rollback is an idealized state that reflects changes to accessible
      components, but ignores the callee's interference with the caller's data. Then the observable
      behavior of the resulting trace is the standard to which we expect any real trace to conform.

      A system enjoys observable integrity if for any initial state \(m_0\), call state \(m_c\),
      and policy state \(p_c\) such that \(m_0,\mathit{pinit}(m_0) \hookrightarrow \dots (m_c,p_c) \dots\)
      and \(\mathit{call}_P(m_c,p_c) = n,a_{pc},a_{sp}\):

      \begin{itemize}
        \item If \((m_c,p_c) \hookrightarrow \dots (m_r,p_r) | \mathit{returned}\ m_r\ a_{pc}\ a_{sp}\):
          \begin{itemize}
            \item Let \(C = \mathit{Cof}(m_c,n)\)
            \item Let \(m_r' = \mathit{roll}_I(C,m_c,m_r)\), \((m_r,p_r) \rightsquigarrow O\), and
              \(m_r' \rightsquigarrow O'\)
            \item If \(m_r,p_r \hookrightarrow MP \in \top\) then \(O \simeq O'\)
            \item Otherwise \(O \lesssim O'\)
          \end{itemize}
      \end{itemize}

    \paragraph{Eager Integrity}
 
      An eager enforcement mechanism implements a stronger property, which we term {\it eager integrity}.
      Eager integrity simplifies observable integrity by requiring that high integrity data of \(m_r\) be
      identical to that of \(m_c\). Formally, a system enjoys eager integrity if for any initial state
      \(m_0\), call state \(m_c\), and policy state \(p_c\) such that \(\mathit{call}(m_c,p_c) =
      n, a_{pc}, a_{sp}\), and some \(m_r\) and \(p_r\) such that
      \(m_0,\mathit{pinit}(m_0) \hookrightarrow \dots (m_c,p_c) \dots (m_r,p_r) | \mathit{returned}\ m_r\ a_{pc}\ a_{sp}\):

      \[\forall k . C(k) = \mathtt{HI} \rightarrow m_c(k) = m_r(k)\]

      This implies lazy integrity, as the rollback function will produce \(m_r' = m_r\).
 
    \paragraph{Observable Confidentiality}

      Observable confidentiality is concerned with whether high confidentiality data influences
      observable behavior during and after a call. We define variants given the generalized notion
      of contours:
      \[m \approx_C m' \triangleq \forall k . C(k) = \mathtt{LC} \rightarrow m(k) = m'(k)\]

      Data that is high confidentiality within a callee due either to being in its caller's frame
      will become low confidentiality if the callee returns. We face a subtle distinction: 
      a callee treats the contents of its caller's frame as secrets, and they should not influence
      its behavior. Nor should the callee be able to influence behavior after its return based
      on the caller's secrets, for instance by copying them to registers that will later be output.
      But the caller itself may still adjust its behavior based on its own secrets. So our property
      relies on another rollback function to preserve changes made by the caller but remove other
      differences between variant states.

      In this case, a confidentiality rollback takes a contour and four states: two call states
      that will be variants, and two return states. It is not symmetric, regarding \(m_c\) and \(m_r\)
      as the ``real'' call and return, \(m_c'\) as a variant and \(m_r'\) as a return that follows it.

      \[\mathit{roll}_C(C,m_c,m_c',m_r,m_r')(k) =
        \begin{cases}
          m_c(k) & \text{if } C(k) = \mathtt{HC} \text{ and } m_c(k) = m_r(k) \text{ and } m_c'(k) = m_r'(k) \\
          m_r'(k) & \text{else} \\
        \end{cases}\]

      A system enjoys observable confidentiality if for any initial state \(m_0\), call state \(m_c\),
      and policy state \(p_c\) such that \(m_0,\mathit{pinit}(m_0) \hookrightarrow \dots (m_c,p_c) \dots\) and
      \(\mathit{call}(m_c,p_c) = n, a_{pc}, a_{sp}\):

      \begin{itemize}
        \item Let \(C = \mathit{Cof}(m_c,n)\)
        \item For all \(m_c'\) such that \(m_c \approx_C m_c'\):
        \item If \((m_c,p_c) \hookrightarrow MP (m_r,p_r) | \mathit{returned}\ m_r\ a_{pc}\ a_{sp}\)
          \begin{itemize}
            \item Then there is some \(m_r'\) where
              \(m_c' \hookrightarrow M m_r' | \mathit{returned}\ m_r'\ a_{pc}\ a_{sp}\)
            \item Let \(m_r'' = \mathit{roll}_C(C,m_c,m_r,m_c',m_r')\),
              \((m_r,p_r) \rightsquigarrow O\), and \(m_r'' \rightsquigarrow O'\)
            \item If \(m_r,p_r \hookrightarrow_P MP \in \top\) then \(O \simeq \mathit{ObsOf}(M) +\!\!\!+ O'\)
            \item Otherwise \(O \lesssim \mathit{ObsOf}(M) +\!\!\!+ O'\)
          \end{itemize}
        \item Else, let \(m_c,p_c \rightsquigarrow O\) and \(m_c' \rightsquigarrow O'\)
          \begin{itemize}
            \item If \(m_c,p_c \hookrightarrow_P MP \in \top\) then \(O \simeq O'\)
            \item Otherwise \(O \lesssim O'\)
          \end{itemize}
      \end{itemize}

    \paragraph{Eager Confidentiality}

      Once again an eager policy enforces a stronger property that only concerns the behavior of
      the original trace and its variant within each call. A system enjoys {\it eager stack
      confidentiality} if, for any initial state \(m_0\) and call state \(m_c\) such that
      \(m_0,\mathit{pinit}(m_0) \hookrightarrow \dots (m_c,p_c) \dots\) and
      \(\mathit{call}(m_c,p_c) = n, a_{pc}, a_{sp}\):

      \begin{itemize}
        \item Let \(C = \mathit{Cof}(m_c,n)\)
        \item For all \(m_c'\) such that \(m_c \approx_C m_c'\):
        \item If \((m_c,p_c) \hookrightarrow_P MP (m_r,p_r) | \mathit{returned}\ m_r\ a_{pc}\ a_{sp}\)
          \begin{itemize}
            \item Then there is some \(m_r'\) where
              \(m_c' \hookrightarrow M m_r' | \mathit{returned}\ m_c'\ m_r'\)
            \item \(\mathit{ObsOf}(MP) \simeq \mathit{ObsOf}(M)\)
            \item And for all \(k\), if \(m_c(k) \not = m_r(k)\) or \(m_c'(k) \not = m_r'(k)\),
              \(m_r(k) = m_r'(k)\)
          \end{itemize}
        \item Else, let \(m_c,p_c \rightsquigarrow O\) and \(m_c' \rightsquigarrow O'\)
          \begin{itemize}
            \item If \(m_c,p_c \hookrightarrow_P MP \in \top\) then \(O \simeq O'\)
            \item Otherwise \(O \lesssim O'\)
          \end{itemize}
      \end{itemize}

      Eager confidentiality implies observable confidentiality, because if \(m_r\) and \(m_r'\) agree
      on all \(k\) that changed during the call, then an \(m_r'' = \mathit{roll}_C(C,m_c,m_r,m_c',m_r')\)
      will be identical to \(m_r\), and future observations will be the same up to a potential policy
      stop.
 
\end{document}
