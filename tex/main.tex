%% For double-blind review submission, w/o CCS and ACM Reference (max submission space)
\documentclass[acmsmall,review,anonymous]{acmart}\settopmatter{printfolios=true,printccs=false,printacmref=false}
%% For double-blind review submission, w/ CCS and ACM Reference
%\documentclass[acmsmall,review,anonymous]{acmart}\settopmatter{printfolios=true}
%% For single-blind review submission, w/o CCS and ACM Reference (max submission space)
%\documentclass[acmsmall,review]{acmart}\settopmatter{printfolios=true,printccs=false,printacmref=false}
%% For single-blind review submission, w/ CCS and ACM Reference
%\documentclass[acmsmall,review]{acmart}\settopmatter{printfolios=true}
%% For final camera-ready submission, w/ required CCS and ACM Reference
%\documentclass[acmsmall]{acmart}\settopmatter{}

%% Journal information
%% Supplied to authors by publisher for camera-ready submission;
%% use defaults for review submission.
\acmJournal{PACMPL}
\acmVolume{1}
\acmNumber{SPLASH} % CONF = POPL or ICFP or OOPSLA
\acmArticle{1}
\acmYear{2021}
\acmMonth{1}
\acmDOI{} % \acmDOI{10.1145/nnnnnnn.nnnnnnn}
\startPage{1}

%% Copyright information
%% Supplied to authors (based on authors' rights management selection;
%% see authors.acm.org) by publisher for camera-ready submission;
%% use 'none' for review submission.
\setcopyright{none}
%\setcopyright{acmcopyright}
%\setcopyright{acmlicensed}
%\setcopyright{rightsretained}
%\copyrightyear{2018}           %% If different from \acmYear

%% Bibliography style
\bibliographystyle{ACM-Reference-Format}
%% Citation style
%% Note: author/year citations are required for papers published as an
%% issue of PACMPL.
\citestyle{acmauthoryear}   %% For author/year citations


%%%%%%%%%%%%%%%%%%%%%%%%%%%%%%%%%%%%%%%%%%%%%%%%%%%%%%%%%%%%%%%%%%%%%%
%% Note: Authors migrating a paper from PACMPL format to traditional
%% SIGPLAN proceedings format must update the '\documentclass' and
%% topmatter commands above; see 'acmart-sigplanproc-template.tex'.
%%%%%%%%%%%%%%%%%%%%%%%%%%%%%%%%%%%%%%%%%%%%%%%%%%%%%%%%%%%%%%%%%%%%%%

%% Some recommended packages.
\usepackage{booktabs}   %% For formal tables:
                        %% http://ctan.org/pkg/booktabs
\usepackage{subcaption} %% For complex figures with subfigures/subcaptions
                        %% http://ctan.org/pkg/subcaption

\newif\ifdraft \drafttrue
\newif\iftext \textfalse
\newif\iflater \latertrue
\newif\ifaftersubmission \aftersubmissionfalse

% !!! PLEASE DON'T CHANGE THESE !!! INSTEAD DEFINE YOUR OWN texdirectives.tex !!!
\makeatletter \@input{texdirectives} \makeatother

%\IEEEoverridecommandlockouts
% The preceding line is only needed to identify funding in the first footnote. If that is unneeded, please comment it out.
\usepackage{cite}
\usepackage{amsmath,amssymb,amsfonts}
\usepackage{algorithmic}
\usepackage{hyperref}
\usepackage{graphicx}
\usepackage{textcomp}
\usepackage[capitalize]{cleveref}
\usepackage[inline]{enumitem}

\usepackage{xcolor}
\newcommand{\bcp}[1]{\ifdraft\textcolor{violet}{{[BCP:~#1]}}\fi}
\newcommand{\leo}[1]{\ifdraft\textcolor{teal}{{[LEO:~#1]}}\fi}
\newcommand{\apt}[1]{\ifdraft\textcolor{blue}{{[APT:~#1]}}\fi}
\newcommand{\rb}[1]{\ifdraft\textcolor{orange}{{[RB:~#1]}}\fi}
\newcommand{\sna}[1]{\ifdraft\textcolor{green}{{[SNA:~#1]}}\fi}
\newcommand{\COQ}[1]{\ifdraft\textcolor{red}{{[COQ DIFFERENCE:~#1]}}\fi}

\usepackage{listings}


\usepackage{xspace}
\newcommand{\cn}{\ifdraft\textsuperscript{\textcolor{blue}{[citation needed]}}\xspace\fi}

\makeatletter
\begingroup
\lccode`\A=`\-
\lccode`\N=`\N
\lccode`\V=`\V
\lowercase{\endgroup\def\memory@noval{ANoValue-}}
\long\def\memory@fiBgb\fi#1#2{\fi}
\long\def\memory@fiTBb\fi#1#2#3{\fi#2}
\newcommand\memory@ifnovalF[1]%>>=
  {%
    \ifx\memory@noval#1%
      \memory@fiBgb
    \fi
    \@firstofone
  }%=<<
\newcommand\memory@ifnovalTF[1]%>>=
  {%
    \ifx\memory@noval#1%
      \memory@fiTBb
    \fi
    \@secondoftwo
  }%=<<
\newcommand\memory@Oarg[2]%>>=
  {%
    \@ifnextchar[{\memory@Oarg@{#2}}{#2{#1}}%
  }%=<<
\long\def\memory@Oarg@#1[#2]%>>=
  {%
    #1{#2}%
  }%=<<
\newcommand*\memory@oarg%>>=
  {%
    \memory@Oarg\memory@noval
  }%=<<
\newcommand*\memory@ifcoloropt%>>=
  {%
    \@ifnextchar[\memory@ifcoloropt@true\memory@ifcoloropt@false
  }%=<<
\long\def\memory@ifcoloropt@true#1\memory@noval#2#3%>>=
  {%
    #2%
  }%=<<
\long\def\memory@ifcoloropt@false#1\memory@noval#2#3%>>=
  {%
    #3%
  }%=<<
\newlength\memory@width
\newlength\memory@height
\setlength\memory@width{23pt}
\setlength\memory@height{14pt}
\newcount\memory@num
\newcommand*\memory@blocks[2]%>>=
  {%
    \memory@num#1\relax
    \fboxsep-\fboxrule
    \memory@ifcoloropt#2\memory@noval
      {\def\memory@color{\textcolor#2}}
      {\def\memory@color{\textcolor{#2}}}%
    \loop
    \ifnum\memory@num>0
      \fbox{\memory@color{\rule{\memory@width}{\memory@height}}}%
      \kern-\fboxrule
      \advance\memory@num\m@ne
    \repeat
  }%=<<
% memory:
%  [#1]: width
%   #2 : count
%  [#3]: height
%   #4 : colour
%  [#5]: label
\newcommand*\memory%>>=
  {%
    \begingroup
    \memory@oarg\memory@a
  }%=<<
\newcommand*\memory@a[2]%>>=
  {%
    % #1 width
    % #2 count
    \memory@ifnovalF{#1}{\memory@width#1\relax}%
    \memory@Oarg\memory@height{\memory@b{#2}}%
  }%=<<
\newcommand*\memory@b[3]%>>=
  {%
    % #1 count
    % #2 height
    % #3 colour
    \memory@ifnovalF{#2}{\memory@height#2\relax}%
    \memory@oarg{\memory@c{#1}{#3}}%
  }%=<<
\newcommand*\memory@c[3]%>>=
  {%
    % #1 count
    % #2 colour
    % #3 label
    \memory@ifnovalTF{#3}
      {\ensuremath{\memory@blocks{#1}{#2}}}
      {\ensuremath{\underbrace{\memory@blocks{#1}{#2}}_{\text{#3}}}}%
    \endgroup
  }%=<<
\makeatother

\newcommand{\judgment}[2]{
  {\centering
  \vspace{\abovedisplayskip}
  \begin{tabular}{c}
    #1 \\
    \hline
    #2
  \end{tabular}
   \vspace{\abovedisplayskip}\par}}

\newcommand{\judgmentbr}[4]{
  {\centering
  \vspace{\abovedisplayskip}
  \begin{tabular}{c}
    #1 \\
    #2 \\
    #3 \\
    \hline
    #4
  \end{tabular}
   \vspace{\abovedisplayskip}\par}}


\newcommand{\judgmenttwo}[3]{
  {\centering
  \vspace{\abovedisplayskip}
  \begin{tabular}{c c}
    #1 & #2 \\
    \hline
    \multicolumn{2}{c}{#3}
  \end{tabular}
  \vspace{\abovedisplayskip}\par}}

\newcommand{\judgmentthree}[4]{
  {\centering
  \vspace{\abovedisplayskip}
  \begin{tabular}{c c c}
    #1 & #2 & #3 \\
    \hline
    \multicolumn{3}{c}{#4}
  \end{tabular}
  \vspace{\abovedisplayskip}\par}}

% Notational conventions
\newcommand{\HIGHSEC}{\textsc{HC}}
\newcommand{\LOWSEC}{\textsc{LC}}
\newcommand{\HIGHINT}{\textsc{HI}}
\newcommand{\LOWINT}{\textsc{LI}}
\newcommand{\IDS}{{\mathcal{I}}}
\newcommand{\ID}{I}
\newcommand{\ME}{\textsc{S}}
\newcommand{\NOTME}{\textsc{O}}
\newcommand{\TRANS}{\ensuremath{-}}
\newcommand{\JAL}{\ensuremath{\mathit{JAL}}}
\newcommand{\ACCYES}{\ensuremath{A}}
\newcommand{\ACCNO}{\ensuremath{I}}
\newcommand{\ACCCODE}{\ensuremath{K}}
\newcommand{\CRCALL}{\ensuremath{\mathit{CALL}}}
\newcommand{\CRRET}{\ensuremath{\mathit{RETURN}}}
\newcommand{\CRBOT}{\ensuremath{\bot}}
\newcommand{\VIS}{\textsc{vis}}
\newcommand{\HID}{\textsc{hid}}
\newcommand{\word}{w}
\newcommand{\addr}{a}
\newcommand{\WORDS}{{\mathcal W}}
\newcommand{\reg}{r}
\newcommand{\REGS}{{\mathcal R}}
\newcommand{\mach}{m}
\newcommand{\machT}{M}
\newcommand{\MACHS}{{\mathcal M}}
\newcommand{\MPT}{\mathit{MP}}
\newcommand{\obs}{o}
\newcommand{\obsT}{O}
\newcommand{\OBSS}{\mathit{Obs}}
\newcommand{\PC}[1]{\PCname(#1)}
\newcommand{\PCname}{\textsc{pc}}
\newcommand{\SP}{\textsc{sp}}
\newcommand{\pol}{p}
\newcommand{\POLS}{\mathcal{P}}
\newcommand{\pinit}{pinit}
\newcommand{\prop}{S}
\newcommand{\contour}{C}
\newcommand{\CONTOURS}{{\mathcal C}}
\newcommand{\component}{k}
\newcommand{\COMPONENTS}{{\mathcal K}}
\newcommand{\trace}{T}
\newcommand{\observer}{O}
\newcommand{\stateobs}{\sigma}
\newcommand{\seq}[1]{\overline{#1}}
\newcommand{\SEQ}[1]{\overline{#1}}
\newcommand{\dstk}[1]{{#1}.\mbox{\it stack}}
\newcommand{\dpcd}[1]{{#1}.\mbox{\it PCdepth}}
\newcommand{\ddep}[2]{{#1}.\mbox{\it depth}({#2})}
\newcommand{\dinit}{\mbox{\it Dinit}}
\newcommand{\empstack}{\mbox{\it empty}}
\newcommand{\access}[2]{\mbox{\it accessible}_{#1}({#2})}
\newcommand{\norm}[1]{\lvert{#1}\rvert}
\newcommand{\MPS}{\mathit{MPState}}
\newcommand{\mpstate}[2]{(#1,#2)}
\newcommand{\mpostate}[3]{(#1,#2,#3)}
\newcommand{\mpstatename}{mp}
\newcommand{\callmap}{cm}
\newcommand{\CALLMAPS}{\mathit{CallMap}}
\newcommand{\ret}[1]{\mathit{justret}\ #1}
\newcommand{\nextPC}{next}
\newcommand{\base}{b}
\newcommand{\stepsto}{\Longrightarrow}
\newcommand{\stepstounder}[1]{\stackrel{\mbox{\tiny{$#1$}}}{\Longrightarrow}}
\newcommand{\stepstounderfull}{\stepstounder{\textsc{RISCV}}}
\newcommand{\manystepsto}{\stepsto^\star}
\newcommand{\obstrace}{\mathit{obstrace}}
\newcommand{\funid}{f}
\newcommand{\FUNIDS}{\mathcal{F}}
\newcommand{\retmap}{\mathit{rm}}
\newcommand{\RETMAPS}{\mathit{RetMap}}
\newcommand{\codemap}{\mathit{fm}}
\newcommand{\CODEMAPS}{\mathit{FuncMap}}
\newcommand{\entmap}{\mathit{em}}
\newcommand{\ENTMAPS}{\mathit{EntryMap}}
\newcommand{\PUT}{\mathit{Until}}
\newcommand{\Trace}{T}
\newcommand{\traceelem}{a}
\newcommand{\TRACEELEMS}{A}
\newcommand{\head}{\mathit{head}}
\newcommand{\last}{\mathit{last}}

\newcommand{\stepstoobs}[1]{\xrightarrow{#1}}
\newcommand{\polstep}{\rightharpoonup}
\newcommand{\stepstopol}[1]{\overset{#1}{\rightharpoonup}}
%\newcommand{\stepstopol}[1]{\overset{#1}{\rightharpoonup}_P}

\newcommand{\stepplus}{\Rightarrow}
\newcommand{\stepkappa}{\Rightarrow_\kappa}
\newcommand{\induced}[2]{(#1, #2)^*}
\newcommand{\flows}{\sqsubseteq}
\newcommand{\flowsstrict}{\sqsubset}
\newcommand{\initmach}{\MACHS_{\mathit{init}}}
\newcommand{\initcontour}{\CONTOURS_{\mathit{init}}}
\newcommand{\closure}[1]{\textit{Close}#1}
\newcommand{\variant}[2]{\textit{Vars}(#1, #2)}
\newcommand{\isinf}{\mathit{inf}}

\newcommand{\Last}[1]{\mathit{Last}(#1)}

\newcommand{\HALT}{\textsc{HALT}}

\newcommand{\underscore}{\mbox{\_}}

\newcommand{\propdef}[1]{\text{\sc #1}}

\newcommand{\TRACE}[1]{\mathit{Trace}~(#1)}
\newcommand{\MTRACE}{\TRACE{\MACHS}}
\newcommand{\MOTRACE}{\TRACE{\MACHS \times \OBSS}}
\newcommand{\MPOTRACE}{\TRACE{\MACHS \times \POLS \times \OBSS}}

\newcommand{\high}{cyan}
\newcommand{\low}{green}
\newcommand{\varied}{magenta}

\colorlet{taintH}{\high!40}
\colorlet{taintL}{\low!40}

\newcommand{\vardata}[2]{\bf {\color{\high} #1}/{\color{\varied} #2}}

\newcommand{\integrityexample}{
% At Call to H
% Memory Layout
\begin{center}
\MemoryLabel{12em}{2em}{100}
\MemoryLabel{47em}{2em}{\SP}
  \(\mach_e\) :
  \memory{1}{\emptyoutc}
  \memory{6}{\high}[{\makebox[0pt]{\(\HIGHINT\)}}]%
  \memory{4}{\low}[{\makebox[0pt]{\(\LOWINT\)}}]%
~$\cdots$
\MemoryLabel{-46.5em}{0.75em}{42}
\MemoryLabel{-42.5em}{0.75em}{0}
\MemoryLabel{-38.5em}{0.75em}{0}
\MemoryLabel{-34.5em}{0.75em}{0}
\MemoryLabel{-30.5em}{0.75em}{8}
\MemoryLabel{-26.5em}{0.75em}{}
\MemoryLabel{-22.5em}{0.75em}{17}
\MemoryLabel{-18.5em}{0.75em}{0}
\\
\end{center}
% Starting Register File
\summary{40}{0}{\thepcctr}{108}{17}{?}
\setcounter{pcctr}{40}
\vspace*{0.2em}
% H - Body
\[
\begin{array}{c|c|c|c}
  \row{\addi ~ \rsp ~ \rsp ~ 1}{\rsp \leftarrow \rsp + 1}{\rsp \leftarrow 109}
      {Allocate for return address}
  \row{\sw ~ \rsp ~ \rra ~ \negate 1}{[\rsp - 1] \leftarrow \rra}{[108] \leftarrow 27}
      {Store return address}
  \row{\lw ~ r_4 ~ \rsp ~ \negate 3}{r_4 \leftarrow [\rsp - 3]}{r_4 \leftarrow 17}
      {Load the argument {\tt v} to $r_4$}
  \row{\addi ~ r_4 ~ r_4 ~ 1}{r_4 \leftarrow r_4 + 1}{r_4 \leftarrow 18}
      {Increment $r_4$ by 1}
  \row{\sw ~ \rsp ~ r_4 ~ \negate 5}{[\rsp - 5] \leftarrow r_4}{[103] \leftarrow 18}
      {Store $r_4$ to {\tt z} {\color{red} (oops!)}}
  \row{\lw ~ \rra ~ \rsp ~ \negate 1}{\rra \leftarrow [\rsp - 1]}{\rra \leftarrow 27}
      {Load return address}
  \row{\addi ~ \rsp ~ \rsp ~ \negate 1}{\rsp \leftarrow \rsp - 1}{\rsp \leftarrow 108}
      {Restore the stack pointer of {\tt f}}
  \row{\jalr ~ \rra ~ \rra ~ 0}{\PCname \leftarrow \rra}{\PCname \leftarrow 27}
      {Return to {\tt f} ($\jalr$)}
  \end{array}
\]
\begin{center}
  \vspace{\abovedisplayskip}
  \hspace{.3em}
  \(\mach_r :\)
  \memory{1}{\emptyoutc}
  \memory{3}{\high}%
  \memory{1}{taintH}%
  \memory{2}{\high}%
  \memory{2}{\low}%
  \memory{1}{taintL}%
  \memory{1}{\low}%
~$\cdots$
\MemoryLabel{-46.5em}{0.75em}{42}
\MemoryLabel{-42.5em}{0.75em}{0}
\MemoryLabel{-38.5em}{0.75em}{0}
\MemoryLabel{-34.5em}{0.75em}{18}
\MemoryLabel{-30.5em}{0.75em}{8}
\MemoryLabel{-26.5em}{0.75em}{}
\MemoryLabel{-22.5em}{0.75em}{17}
\MemoryLabel{-18.5em}{0.75em}{0}
\MemoryLabel{-15em}{0.75em}{27}
\\
\end{center}
}

\newcommand{\integritylazyexample}{
\begin{center}
\MemoryLabel{12em}{2em}{100}
\MemoryLabel{47em}{2em}{\SP}
  \(\mach_r\) :
  \memory{1}{\emptyoutc}%
  \memory{3}{\high}%
  \memory{1}{taintH}%
  \memory{2}{\high}%
  \memory{4}{\low}%
~$\cdots$
\MemoryLabel{-46.5em}{0.75em}{42}
\MemoryLabel{-42.5em}{0.75em}{0}
\MemoryLabel{-38.5em}{0.75em}{0}
\MemoryLabel{-34.5em}{0.75em}{18}
\MemoryLabel{-30.5em}{0.75em}{8}
\MemoryLabel{-26.5em}{0.75em}{}
\MemoryLabel{-22.5em}{0.75em}{17}
\MemoryLabel{-18.5em}{0.75em}{0}
\MemoryLabel{-14.5em}{0.75em}{27}
\\
\end{center}
%
\begin{center}
  \ \ 
  \(\mach_r'\) :
  \memory{1}{\emptyoutc}%
  \memory{3}{\high}%
  \memory{1}{taintH}%
  \memory{2}{\high}%
  \memory{4}{\low}%
~$\cdots$
\MemoryLabel{-46.5em}{0.75em}{42}
\MemoryLabel{-42.5em}{0.75em}{0}
\MemoryLabel{-38.5em}{0.75em}{0}
\MemoryLabel{-34.5em}{0.75em}{0}
\MemoryLabel{-30.5em}{0.75em}{8}
\MemoryLabel{-26.5em}{0.75em}{}
\MemoryLabel{-22.5em}{0.75em}{17}
\MemoryLabel{-18.5em}{0.75em}{0}
\MemoryLabel{-14.5em}{0.75em}{27}
\\
\end{center}
% 
% Starting Register File
\summary{27}{0}{\thepcctr}{108}{18}{?}
%\vspace*{0.2em}
\setcounter{pcctr}{27}
% H - Body
 \vspace*{0.2em}
\[
\begin{array}{c|c|c|c}
  \row{\lw ~ r_4 ~ \rsp ~ 0}{r_4 \leftarrow [\rsp]}{r_4 \leftarrow 0}
      {Load return value to $r_4$}
  \row{\sw ~ \rsp ~ r_4 ~ \negate 3}{[\rsp - 3] \leftarrow r_4}{[105] \leftarrow 0}
      {Set {\tt w} to return value}
  \row{\addi ~ \rsp ~ \rsp ~ \negate 2}{\rsp \leftarrow \rsp - 2}{\rsp \leftarrow 106}
      {Deallocate {\tt g}'s call frame}
  \row{\lw ~ r_5 ~ \rsp ~ \negate 4}{r_5 \leftarrow [\rsp - 4]}
      {r_5 \leftarrow \vardata{18}{0}}{Load argument {\tt z} $r_5$}
  \row{\sw ~ \rout ~ r_5 ~ 0}{[\rout] \leftarrow r_5}{[99] \leftarrow \vardata{18}{0}}
      {Print {\tt z}}      
\end{array}
\]
\[\pi_o(\MPT) \not \eqsim \pi_o(\machT')\]
}

\newcommand{\confidentialityendexample}{
% At Call to H
% Memory Layout
\begin{center}
\MemoryLabel{12em}{2em}{100}
\MemoryLabel{47em}{2em}{\SP}
  \(\mach_e\) :
  \memory{1}{\emptyoutc}
  \memory{6}{\high}[{\makebox[0pt]{\(\HIGHSEC\)}}]%
  \memory{2}{\low}[{\makebox[0pt]{\(\LOWSEC\)}}]%
  \memory{2}{\high}[{\makebox[0pt]{\(\HIGHSEC\)}}]
~$\cdots$
\MemoryLabel{-46.5em}{0.75em}{42}
\MemoryLabel{-42.5em}{0.75em}{0}
\MemoryLabel{-38.5em}{0.75em}{0}
\MemoryLabel{-34.5em}{0.75em}{0}
\MemoryLabel{-30.5em}{0.75em}{8}
\MemoryLabel{-26.5em}{0.75em}{}
\MemoryLabel{-22.5em}{0.75em}{17}
\MemoryLabel{-18.5em}{0.75em}{0}
\\
\end{center}
% Variant
\begin{center}
  \ \ \ \ 
  \(\mach_e'\) :
  \memory{1}{\emptyoutc}%
  \memory{6}{\varied}[{\makebox[0pt]{Varied}}]%
  \memory{2}{\low}%
\memory{2}{\varied}[{\makebox[0pt]{Varied}}]
~$\cdots$
\MemoryLabel{-46.5em}{0.75em}{$-5$}
\MemoryLabel{-42.5em}{0.75em}{$-5$}
\MemoryLabel{-38.5em}{0.75em}{$-5$}
\MemoryLabel{-34.5em}{0.75em}{$-5$}
\MemoryLabel{-31em}{0.75em}{$-5$}
\MemoryLabel{-27em}{0.75em}{$-5$}
\MemoryLabel{-22.5em}{0.75em}{$17$}
\MemoryLabel{-18.5em}{0.75em}{$0$}
\MemoryLabel{-14.5em}{0.75em}{$-5$}
%\MemoryLabel{-11.5em}{0.75em}{$-5$}
\\
\end{center}

% Starting Register File
\summary{40}{0}{\thepcctr}{108}{17}{?}
\setcounter{pcctr}{40}
\vspace*{0.2em}
% H - Body
\[
\begin{array}{c|c|c|c}
  \row{\addi ~ \rsp ~ \rsp ~ 1}{\rsp \leftarrow \rsp + 1}{\rsp \leftarrow 109}
      {Allocate for return address}
  \row{\sw ~ \rsp ~ \rra ~ \negate 1}{[\rsp - 1] \leftarrow \rra}{[108] \leftarrow 27}
      {Store return address}
  \row{\lw ~ r_4 ~ 100}{r_4 \leftarrow 100}{r_4 \leftarrow \vardata{42}{-5}}
      {Load {\tt main}'s secret {\tt x} to $r_4$ {\color{red} (oops!)}}
  \row{\addi ~ r_4 ~ r_4 ~ 1}{r_4 \leftarrow r_4 + 1}{r_4 \leftarrow \vardata{43}{-4}}
      {Increment $r_4$ by 1}
  \row{\sw ~ \rsp ~ r_4 ~ \negate 2}{[\rsp - 2] \leftarrow r_4}{[107] \leftarrow \vardata{43}{-4}}
      {Store $r_4$ as the return value}
  \row{\lw ~ \rra ~ \rsp ~ \negate 1}{\rra \leftarrow [\rsp - 1]}{\rra \leftarrow 27}
      {Load return address}
  \row{\addi ~ \rsp ~ \rsp ~ \negate 1}{\rsp \leftarrow \rsp - 1}{\rsp \leftarrow 108}
      {Restore the stack pointer of {\tt f}}
  \row{\jalr ~ \rra ~ \rra ~ 0}{\PCname \leftarrow \rra}{\PCname \leftarrow 27}
      {Return to {\tt f} ($\jalr$) }
  \end{array}
\]

\begin{center}
  \hspace{.4em}
  \(\mach_r\) :
  \memory{1}{\emptyoutc}%
  \memory{6}{\high}%
  \memory{1}{\low}%
  \memory{1}{taintL}%
\memory{2}{\high}%[{\makebox[0pt]{Uninitialized}}]
~$\cdots$
\MemoryLabel{-46.5em}{0.75em}{42}
\MemoryLabel{-42.5em}{0.75em}{0}
\MemoryLabel{-38.5em}{0.75em}{0}
\MemoryLabel{-34.5em}{0.75em}{0}
\MemoryLabel{-30.5em}{0.75em}{8}
\MemoryLabel{-26.5em}{0.75em}{}
\MemoryLabel{-22.5em}{0.75em}{17}
\MemoryLabel{-18.5em}{0.75em}{43}
\MemoryLabel{-14.5em}{0.75em}{27}
\\
\end{center}

%
\begin{center}
  \hspace{.4em}
  \(\mach_r'\) :
  \memory{1}{\emptyoutc}%
  \memory{6}{\varied}%
  \memory{1}{\low}%
  \memory{1}{taintL}%
\memory{2}{\varied}%[{\makebox[0pt]{Uninitialized}}]
~$\cdots$
\MemoryLabel{-46.5em}{0.75em}{$-5$}
\MemoryLabel{-42.5em}{0.75em}{$-5$}
\MemoryLabel{-38.5em}{0.75em}{$-5$}
\MemoryLabel{-34.5em}{0.75em}{$-5$}
\MemoryLabel{-31em}{0.75em}{$-5$}
\MemoryLabel{-27em}{0.75em}{$-5$}
\MemoryLabel{-22.5em}{0.75em}{$17$}
\MemoryLabel{-19em}{0.75em}{$-4$}
\MemoryLabel{-14.5em}{0.75em}{$27$}
%\MemoryLabel{-11.5em}{0.75em}{-5}
\\
\end{center}
%
}

\newcommand{\confidentialitylazyexample}{
\begin{center}
  \MemoryLabel{12em}{2em}{100}
  \MemoryLabel{47em}{2em}{\SP}
  \(\mach_r\) :
  \memory{1}{\emptyoutc}%
  \memory{6}{\high}%
  \memory{1}{\low}%
  \memory{1}{taintL}%
  \memory{1}{taintH}%
  \memory{1}{\high}%[{\makebox[0pt]{Uninitialized}}]
~$\cdots$
\MemoryLabel{-46.5em}{0.75em}{42}
\MemoryLabel{-42.5em}{0.75em}{0}
\MemoryLabel{-38.5em}{0.75em}{0}
\MemoryLabel{-34.5em}{0.75em}{0}
\MemoryLabel{-30.5em}{0.75em}{8}
\MemoryLabel{-26.5em}{0.75em}{}
\MemoryLabel{-22.5em}{0.75em}{17}
\MemoryLabel{-18.5em}{0.75em}{43}
\MemoryLabel{-14.5em}{0.75em}{27}
\\
\end{center}

\begin{center}
  \ \ \ 
  \(\mach_r'\) :
  \memory{1}{\emptyoutc}%
  \memory{6}{\varied}%
  \memory{1}{\low}%
  \memory{1}{taintL}%
  \memory{1}{taintH}%
\memory{1}{\varied}%[{\makebox[0pt]{Uninitialized}}]
~$\cdots$
\MemoryLabel{-46.5em}{0.75em}{$-5$}
\MemoryLabel{-42.5em}{0.75em}{$-5$}
\MemoryLabel{-38.5em}{0.75em}{$-5$}
\MemoryLabel{-34.5em}{0.75em}{$-5$}
\MemoryLabel{-31em}{0.75em}{$-5$}
\MemoryLabel{-27em}{0.75em}{$-5$}
\MemoryLabel{-22.5em}{0.75em}{$17$}
\MemoryLabel{-18.5em}{0.75em}{$-4$}
\MemoryLabel{-14.5em}{0.75em}{$27$}
\MemoryLabel{-10.5em}{0.75em}{$-5$}
\\
\end{center}
%
%
\begin{center}
  \ \ \ 
  \(\mach_r''\) :
  \memory{1}{\emptyoutc}%
  \memory{6}{\varied}%
  \memory{1}{\low}%
  \memory{1}{taintL}%
  \memory{1}{taintH}%
\memory{1}{\varied}%[{\makebox[0pt]{Uninitialized}}]
~$\cdots$
\MemoryLabel{-46.5em}{0.75em}{$42$}
\MemoryLabel{-42.5em}{0.75em}{$0$}
\MemoryLabel{-38.5em}{0.75em}{$0$}
\MemoryLabel{-34.5em}{0.75em}{$0$}
\MemoryLabel{-31em}{0.75em}{$-5$}
\MemoryLabel{-27em}{0.75em}{$-5$}
\MemoryLabel{-22.5em}{0.75em}{$17$}
\MemoryLabel{-18.5em}{0.75em}{$-4$}
\MemoryLabel{-14.5em}{0.75em}{$27$}
\MemoryLabel{-10.5em}{0.75em}{$-5$}
\\
\end{center}
%
\summary{27}{0}{\thepcctr}{108}{18}{?}
%\vspace*{0.2em}
\setcounter{pcctr}{27}
% H - Body
\vspace*{0.2em}
\[
\begin{array}{c|c|c|c}
  \row{\lw ~ r_4 ~ \rsp ~ 0}{r_4 \leftarrow [\rsp]}{r_4 \leftarrow \vardata{43}{-4}}
      {Load return value to $r_4$}
  \row{\sw ~ \rsp ~ r_4 ~ \negate 3}{[\rsp - 3] \leftarrow r_4}{[105] \leftarrow \vardata{43}{-4}}
      {Set {\tt w} to returned value}
  \row{\addi ~ \rsp ~ \rsp ~ \negate 2}{\rsp \leftarrow \rsp - 2}{\rsp \leftarrow 106}
      {Deallocate {\tt g}'s call frame}
  \row{\lw ~ r_5 ~ \rsp ~ \negate 4}{r_5 \leftarrow [\rsp - 4]}{r_5 \leftarrow 0}
      {Load argument {\tt z} $r_5$}
  \row{\sw ~ \rout ~ r_5 ~ 0}{[\rout] \leftarrow r_5}{[99] \leftarrow 0}
      {Print {\tt z}}
  \row{\add ~ r_4 ~ r_4 ~ r_5}{r_4 \leftarrow r_4 + r_5}{r_4 \leftarrow \vardata{43}{-4}}
      {Add $r_4$ and $r_5$}
  \row{\sw ~ \rsp ~ r_4 ~ \negate 3}{[\rsp - 3] \leftarrow r_4}{[103] \leftarrow \vardata{43}{-4}}
      {Store $r_4$ as ther return of {\tt f}}
  \row{\lw ~ \rra ~ \rsp ~ \negate 2}{\rra \leftarrow [\rsp - 2]}{\rra \leftarrow 8}
      {Load return address}
  \row{\addi ~ \rsp ~ \rsp ~ \negate 2}{\rsp \leftarrow \rsp - 2}{\rsp \leftarrow 104}
      {Deallocate {\tt f}'s local state}
  \row{\jalr ~ \rra ~ \rra ~ 0}{\PCname \leftarrow \rra}{\PCname \leftarrow 8}{Return to {\tt main} ($\jalr$)}
\end{array}
\]
\setcounter{pcctr}{8}
\[
\begin{array}{c|c|c|c}
  \row{\lw ~ \rsp ~ r_4 ~ \negate 1}{r_4 \leftarrow [\rsp - 1]}{r_4 \leftarrow \vardata{43}{-4}}
      {Load return value to $r_4$}
  \row{\sw ~ \rsp ~ r_4 ~ \negate 3}{[\rsp - 3] \leftarrow r_4}{[101] \leftarrow \vardata{43}{-4}}
      {Set {\tt y} to the result}
  \row{\addi ~ \rsp ~ \rsp ~ \negate 4}{\rsp \leftarrow \rsp - 2}{\rsp \leftarrow 102}
      {Deallocate {\tt f}'s call frame}
  \row{\lw ~ \rsp ~ r_5 ~ \negate 2}{r_5 \leftarrow [\rsp - 2]}{r_5 \leftarrow 42}
      {Load {\tt x} to $r_5$}
  \row{\add ~ r_4 ~ r_4 ~ r_5}{r_4 \leftarrow r_4 + r_5}{r_4 \leftarrow \vardata{85}{38}}
      {Add {\tt x} and {\tt y}}
  \row{\sw ~ \rout ~ r_4 ~ 0}{[\rout] \leftarrow r_4}{[99] \leftarrow \vardata{85}{38}}
      {Print {\tt x + y}}
\end{array}
\]
\[\pi_o(\MPT) \not \eqsim \pi_o(\machT)\]
}


\newcommand*{\add}{\textsc{add}}
\newcommand*{\addi}{\textsc{addi}}
\newcommand*{\sw}{\textsc{sw}}
\newcommand*{\lw}{\textsc{lw}}
\newcommand*{\jal}{\textsc{jal}}
\newcommand*{\jalr}{\textsc{jalr}}
%\newcommand*{\rsp}{\textsc{sp}}
\newcommand*{\rra}{\textsc{ra}}
\newcommand*{\rout}{\textsc{out}}

\newcommand*{\tagInstr}{\textsc{instr}}
\newcommand*{\tagCall}{\textsc{call}}
\newcommand*{\tagHa}{\textsc{h1}}
\newcommand*{\tagHb}{\textsc{h2}}
\newcommand*{\tagRa}{\textsc{r1}}
\newcommand*{\tagRb}{\textsc{r2}}
\newcommand*{\tagRc}{\textsc{r3}}
\newcommand*{\tagNoDepth}{\textsc{unused}}
\newcommand*{\tagStackDepth}[1]{\textsc{stack} ~ #1}
\newcommand*{\tagPCDepth}[1]{\textsc{pc} ~ #1}
\newcommand*{\tagSP}{\textsc{sp}}

\newcommand{\negate}{\textrm{-}}

\colorlet{lgray}{gray!40}

\newcommand{\mainsealc}{cyan}
\newcommand{\fsealc}{green}
\newcommand{\unsealc}{lgray}
\newcommand{\emptyoutc}{white} % or gray for consistency with unusedc ?
\newcommand{\fulloutc}{white}

\begin{figure}

\begin{subfigure}{.4\textwidth}
\begin{verbatim}
int main() {
0:     int x = 42, y = 0;
1[C]:  y = f(10);
2:     print (x + y);
}

int f(int z) {
10:     print z;
11[R]:  return (z+z);
}
\end{verbatim}

\caption{A Simple Call}
\label{fig:simple-program}
\end{subfigure}
\begin{subfigure}{.4\textwidth}  
\begin{center}
\begin{tabular}{l l}
{\tt 0} &
\memoryaddrs{4em}
\memory{2}{\unsealc}[{\makebox[0pt]{Unsealed}}]
~$\cdots$
\\
{\tt 1} &
\memoryaddrs{12em}
\memory{3}{\unsealc}
~$\cdots$
\MemoryLabel{-14em}{0.75em}{42}
\MemoryLabel{-10em}{0.75em}{0}
\vspace{.5em}
\\
{\tt 10} &
\memoryaddrs{17em}
\memory{2}{\mainsealc}[{\makebox[0pt]{Sealed(0)}}]%
\memory{2}{\unsealc}
~$\cdots$
\MemoryLabel{-10em}{0.75em}{2}
\\
{\tt 2} &
\memoryaddrs{12em}
\memory{3}{\unsealc}
~$\cdots$
\MemoryLabel{-14em}{0.75em}{42}
\MemoryLabel{-10em}{0.75em}{0}
\MemoryLabel{-7em}{0.75em}{2}
\\
\end{tabular}
\end{center}

\vspace{\abovedisplayskip}

\caption{Trace of Stack Domains}

\label{fig:simple-trace}
\end{subfigure}
\end{figure}

This section will elucidate the intuition behind the properties through a series
of examples. Existing stack safety enforcement work mostly refers directly to examples
of bad behavior that it prevents; we will insert our model between the enforcement and
the examples, arguing that:

\begin{itemize}
\item Examples that appear to violate stack safety, violate at least one of our properties
\item We do not find desireable behaviors that appear stack-safe yet violate our properties
\item Our properties are testably implemented by stack safety micro-policies from the literature,
  and the micro-policies are not significantly more conservative (i.e., policy failstops tend to
  correspond to imminent property violations)
\end{itemize}

\Cref{fig:simple-program} shows a simple example, written in C-like pseudo-code.
It consists of a {\tt main} function that performs a call to a function {\tt f},
which prints its argument, performs arithmetic, and returns a result.
\Cref{fig:simple-trace} shows the stack as it is divided into domains throughout
the program execution. We use high-level code for compactness, but reason about
a standard compilation of each function. Line numbers correspond to the code
addresses just before the code that implements the line, and annotations
{\tt C} and {\tt R} indicate that an instruction is annotated as a call or
return, respectively.

We also assume that the machine is word- (not byte-) addressed and that the
machine communicates with the outside world through a single memory-mapped
output port whose address is elided (so a {\tt print} statement will be
compiled to a store to this location), and that the stack grows upwards.

In the initial memory layout the entire stack is currently {\em unsealed},
meaning that it is eligible to be written to. At the call from {\tt main} to
{\tt f}, {\tt main}'s locals (as designated by the stack pointer) are
{\em sealed} and labeled with the depth of their owner, in this case 0.
Sealing is the fundamental integrity contract between the caller and callee:
{\tt main} expects that {\tt x} and {\tt y} will be unchanged when it gets control back.
This gives us our simple, intuitive statement of {\em stepwise integrity}:
if a component is sealed before a step, it will be unchanged afterward. Stepwise
refers to the fact that the property must hold on every step. Later we will
loosen this condition in the interest of extensionality -- the caller only
cares about changes it can see after the return, not changes during execution.

Suppose we insert an instruction after line 10, that stores 5 to the stack pointer - 1.
Then we would be changing {\tt y}, which is sealed, in violation of stepwise integrity.

\begin{figure}

\begin{subfigure}{.33\textwidth}
\begin{verbatim}
int main() {
0:     int x = 42, y = 0;
1[C]:  y = f();
2:     print (x + y);
}

int f() {
10:     int z = *($SP-1);
11[R]:  return (z+z);
}
\end{verbatim}
\label{fig:conf1-program}
\end{subfigure}
\begin{subfigure}{.59\textwidth}
\centering
\begin{tabular}{l l l}
  {\tt 0} &
  \multicolumn{2}{c}{
    \memoryaddrs{4em}
    \memory{3}{\unsealc}
    ~$\cdots$
    \MemoryLabel{-14em}{0.75em}{a}
    \MemoryLabel{-10em}{0.75em}{b}
    \MemoryLabel{-7em}{0.75em}{c}
    \vspace{.5em}
  } \\
  {\tt 1} &
  \multicolumn{2}{c}{
    \memoryaddrs{12em}
    \memory{3}{\unsealc}
    ~$\cdots$
    \MemoryLabel{-14em}{0.75em}{42}
    \MemoryLabel{-10em}{0.75em}{0}
    \MemoryLabel{-7em}{0.75em}{c}
    \vspace{.5em}
  } \\
  {\tt 10} &
  \memoryaddrs{20em}
  \memory{2}{\mainsealc}
  \memory{2}{\unsealc}
  ~$\cdots$
  \MemoryLabel{-19em}{0.75em}{42}
  \MemoryLabel{-14em}{0.75em}{0}
  \MemoryLabel{-10em}{0.75em}{2}
  \MemoryLabel{-7em}{0.75em}{\bf d}
 &
  \memoryaddrs{20em}
  \memory{2}{\mainsealc}
  \memory{2}{\unsealc}
  ~$\cdots$
  \MemoryLabel{-19em}{0.75em}{a'}
  \MemoryLabel{-14em}{0.75em}{b'}
  \MemoryLabel{-10em}{0.75em}{2}
  \MemoryLabel{-7em}{0.75em}{\bf d'}
  \\
  {\tt 11} &
  \memoryaddrs{20em}
  \memory{2}{\mainsealc}
  \memory{2}{\unsealc}
  ~$\cdots$
  \MemoryLabel{-19em}{0.75em}{42}
  \MemoryLabel{-14em}{0.75em}{0}
  \MemoryLabel{-10em}{0.75em}{2}
  \MemoryLabel{-7em}{0.75em}{\bf 0}
  &
  \memoryaddrs{20em}
  \memory{2}{\mainsealc}
  \memory{2}{\unsealc}
  ~$\cdots$
  \MemoryLabel{-19em}{0.75em}{a'}
  \MemoryLabel{-14em}{0.75em}{b'}
  \MemoryLabel{-10em}{0.75em}{2}
  \MemoryLabel{-7em}{0.75em}{\bf b'}
  \\
\end{tabular}

\vspace{\abovedisplayskip}

\label{fig:conf1-trace}
\end{subfigure}
\caption{A Confidentiality Violation}
\end{figure}

Meanwhile confidentiality intuitively states that the callee, {\tt f}, is unaware
of the context of its call (with the exception, of course, of its arguments).
We state this in a style similar to {\em noninterference}, a well-known concept in theoretical
security. Noninterference describes knowledge: it states that for some component(s)
that we do not know, we can substitute any values whatsoever without changing the results
of our computation. If there exists some value for which we see different results, then
in fact we can learn something about the components by observing those results, violating
confidentiality.

In example \ref{fig:conf1} we see that the callee, {\tt f}, violates confidentiality
by reading {\tt main}'s variable {\tt y}, which was not passed to it. The trace
in figure \ref{fig:conf1-trace} shows how noninterference works. First, let \(a\),
\(b\), and \(c\) be the unknown initial values at their locations. On entry to {\tt f}
we again have 42 and 0 in {\tt x} and {\tt y}, and {\tt z} contains \(c\). We consider a
{\em variant state} that replaces the values of components that should be secret with
arbitrary values \(a'\), \(b'\), and \(c'\). After a step, we've copied 0 and \(c'\),
respectively, into {\tt z}. Since {\tt z} changed value, for confidentiality to hold,
its value should be the same in all variant states. But when \(c' \not = 0\), the
states changes differently, which violates confidentiality. It would also be a violation
to print the secret value.

This form of confidentiality we will term {\em lockstep confidentiality}, defined informally
as: for any state at the entry to a function, for any variant over the values in the stack,
the primary state and its variant will step in lockstep until they return. Any components
whose values update and any printed values will be identical between the primary and variant
traces.

\begin{figure}

\begin{subfigure}{.32\textwidth}
\begin{verbatim}
int main() {
0[C]:  int t = f();
1[R]:  return;
}

int f() {
10:     int x, y = 5, z = 6;
11:     print y+z;
12:     $SP-=2;
13[C]:  x = g();
14[R]:  return x;
}

int g() {
20:     int z;
21[R]:  return (z+z);
}
\end{verbatim}
\end{subfigure}
\begin{subfigure}{.5\textwidth}  
\begin{center}
\begin{tabular}{l l l l}
{\tt 13} &
\memoryaddrs{12em}
\memory{1}{\mainsealc}
\memory{3}{\unsealc}
~$\cdots$
\MemoryLabel{-18em}{0.75em}{a}
\MemoryLabel{-14em}{0.75em}{b}
\MemoryLabel{-10em}{0.75em}{5}
\MemoryLabel{-7em}{0.75em}{6}
\vspace{.5em}
& &
\\
{\tt 20} &
\memoryaddrs{12em}
\memory{1}{\mainsealc}
\memory{1}{\fsealc}
\memory{2}{\unsealc}
~$\cdots$
\MemoryLabel{-18em}{0.75em}{a}
\MemoryLabel{-14em}{0.75em}{b}
\MemoryLabel{-10em}{0.75em}{5}
\MemoryLabel{-7em}{0.75em}{6}
\vspace{.5em} &
\memoryaddrs{12em}
\memory{1}{\mainsealc}
\memory{1}{\fsealc}
\memory{2}{\unsealc}
~$\cdots$
\MemoryLabel{-18em}{0.75em}{a'}
\MemoryLabel{-14em}{0.75em}{b'}
\MemoryLabel{-10em}{0.75em}{c'}
\MemoryLabel{-7em}{0.75em}{d'}
\vspace{.5em}
&
\\
{\tt 21} &
\memoryaddrs{12em}
\memory{1}{\mainsealc}
\memory{1}{\fsealc}
\memory{2}{\unsealc}
~$\cdots$
\MemoryLabel{-18em}{0.75em}{a}
\MemoryLabel{-14em}{0.75em}{b}
\MemoryLabel{-10em}{0.75em}{5}
\MemoryLabel{-7em}{0.75em}{6}
\vspace{.5em} &
\memoryaddrs{12em}
\memory{1}{\mainsealc}
\memory{1}{\fsealc}
\memory{2}{\unsealc}
~$\cdots$
\MemoryLabel{-18em}{0.75em}{a'}
\MemoryLabel{-14em}{0.75em}{b'}
\MemoryLabel{-10em}{0.75em}{c'}
\MemoryLabel{-7em}{0.75em}{d'}
\vspace{.5em}
&
\\

\end{tabular}
\end{center}

\vspace{\abovedisplayskip}

\end{subfigure}

\caption{Another Confidentiality Violation}
\label{fig:conf2}
\end{figure}

Two important details of confidentiality are reflected in figure \ref{fig:conf2}.
First of all, it has multiple calls, and while we focus on the stack at the call
from line 13 to line 20, it should be clear that we also need to consider a similar
variant at the call from 0 to 10. Confidentiality is in this sense recursive: each
nested must obey confidentiality in its own right.

More importantly, careful readers will have noticed that the informal definition
of stack confidentiality refers to a variation of the entire stack, not just the
sealed portion. In fact, we support temporal stack safety, in which values left behind
by previous activations are also inaccessible. In this case, we connect temporal safety
firmly to the caller-callee attacker model -- {\tt f} stored some data in the variables
{\tt y} and {\tt z}, used it, then deallocated them by moving the stack pointer.
After the stack pointer is written over {\tt y}, the value of {\tt z} is still
left behind. Then {\tt g} can read it just by allocating and accessing an uninitialized
variable!

\paragraph*{Awkward Examples}


\newcommand{\powerset}{\raisebox{.15\baselineskip}{\Large\ensuremath{\wp}}}
\newcommand*{\rsp}{\textsc{sp}}

\newcommand{\TAGS}{\mathcal{T}}
\newcommand{\tagname}{t}
\newcommand{\uP}{{\mu P}}

\begin{document}

%% Title information
\title{Stack Safety is a Security Property}         %% [Short Title] is optional;
                                        %% when present, will be used in
                                        %% header instead of Full Title.
%\titlenote{with title note}             %% \titlenote is optional;
%                                        %% can be repeated if necessary;
%                                        %% contents suppressed with 'anonymous'
%\subtitle{Subtitle}                     %% \subtitle is optional
%\subtitlenote{with subtitle note}       %% \subtitlenote is optional;
%                                        %% can be repeated if necessary;
%                                        %% contents suppressed with 'anonymous'


%% Author information
%% Contents and number of authors suppressed with 'anonymous'.
%% Each author should be introduced by \author, followed by
%% \authornote (optional), \orcid (optional), \affiliation, and
%% \email.
%% An author may have multiple affiliations and/or emails; repeat the
%% appropriate command.
%% Many elements are not rendered, but should be provided for metadata
%% extraction tools.

%% Author with single affiliation.
%\author{First1 Last1}
%\authornote{with author1 note}          %% \authornote is optional;
%                                        %% can be repeated if necessary
%\orcid{nnnn-nnnn-nnnn-nnnn}             %% \orcid is optional
%\affiliation{
%  \position{Position1}
%  \department{Department1}              %% \department is recommended
%  \institution{Institution1}            %% \institution is required
%  \streetaddress{Street1 Address1}
%  \city{City1}
%  \state{State1}
%  \postcode{Post-Code1}
%  \country{Country1}                    %% \country is recommended
%}
%\email{first1.last1@inst1.edu}          %% \email is recommended
%
%%% Author with two affiliations and emails.
%\author{First2 Last2}
%\authornote{with author2 note}          %% \authornote is optional;
%                                        %% can be repeated if necessary
%\orcid{nnnn-nnnn-nnnn-nnnn}             %% \orcid is optional
%\affiliation{
%  \position{Position2a}
%  \department{Department2a}             %% \department is recommended
%  \institution{Institution2a}           %% \institution is required
%  \streetaddress{Street2a Address2a}
%  \city{City2a}
%  \state{State2a}
%  \postcode{Post-Code2a}
%  \country{Country2a}                   %% \country is recommended
%}
%\email{first2.last2@inst2a.com}         %% \email is recommended
%\affiliation{
%  \position{Position2b}
%  \department{Department2b}             %% \department is recommended
%  \institution{Institution2b}           %% \institution is required
%  \streetaddress{Street3b Address2b}
%  \city{City2b}
%  \state{State2b}
%  \postcode{Post-Code2b}
%  \country{Country2b}                   %% \country is recommended
%}
%\email{first2.last2@inst2b.org}         %% \email is recommended


%% Abstract
%% Note: \begin{abstract}...\end{abstract} environment must come
%% before \maketitle command
\begin{abstract}
%\apt{maybe: ``Stack safety is a testable security property''?}\bcp{That
%  would sacrifice a lot of oomph for a little bit of precision, IMO.  Prefer
%  to leave it out of the title but hit it strongly in the abstract.}%
What does ``stack safety'' mean? The phrase is associated with a
variety of compiler,
run-time, and hardware mechanisms for guarding against corruption of stack
memory.  But these mechanisms typically lack precise specifications,
relying instead on informal descriptions and examples of bad
behaviors that they prevent.

We propose a formal characterization
of stack safety, formulated with concepts from language-based security: a
combination of an integrity property (``the private
state in each caller's stack frame is held invariant by the callee'')
a confidentiality property (``the callee's behavior is insensitive to the
caller's private state''), and a well-bracketedness property (``each
callee returns control to its immediate caller'').
% These properties are sensible even in the presence of
% abnormal control flow.\apt{Is this important enough to mention here?}
% BCP: I think not in the abstract, but maybe in the intro.
% We further apply\apt{what does this mean? we didn't
%  invent it} a straightforward notion of
% ``well-bracketed control flow'' that restricts control flow to the
% typical pattern of matched calls and returns.
%
We use these properties to validate the stack-safety ``micro-policies''
proposed by~\citet{DBLP:conf/sp/RoesslerD18}.  Specifically, we check (with
property-based random testing) that Roessler and Dehon's ``eager''
micro-policy, which catches violations as early as possible, enforces a
simple ``stepwise'' variant of our properties and correctly detects several
broken variants, and that (a repaired version of) their more performant
``lazy'' micro-policy corresponds to a slightly weaker and more extensional
variant of our properties.
%
% \BCP{I still don't think the next sentence really nails it.  And (sigh) I am
%   still not really clear yet on whether our definition applies / might apply
%   to other enforcement mechanisms or not.  Saying that our attacker model is
%   stronger does not help---that should make it easier for us to describe
%   their protections, not harder.  So I don't understand what we're saying
%   here.}%
% %
% While we do not rule out other enforcement mechanisms, our
% definitions are strongly informed by micropolicy systems, especially
% our attacker model, which is stronger than other enforcement mechanisms
% can enable.

% For the sake of concreteness, most of our presentation assumes a standard RISC
% machine architecture with a basic subroutine calling convention (parameters
% passed in registers) and a micropolicy-like enforcement mechanism. We
% also show how to accomodate stack-allocated parameter passing and
% coroutines.

% \bcp{this is also a
%   good place to admit that our definitions are tuned for micropolicies, if
%   we can find a succinct way to say it.}
% \sna{See above.}
% \BCP{I added it to the last paragraph (which is pretty close to the preceding
%   paragraph :-)  We should follow this up with some discussion in the (new)
%   Assumptions section about how different enforcement mechanisms (us,
%   StkTokens, maybe others) adopt different perspectives on what the stack is
% for and how it is organized and so lead to somewhat different notions of
% low-level safety.}

% We argue \apt{do we?}\bcp{I guess we will not have time to.  Propose
%   deleting this para.} that our properties are suited to a range of other
% enforcement mechanisms. The strict stepwise properties can describe
% mechanisms that catch errors immediately, as is typical. While the
% observational properties are closely linked to lazy micro-policies they are
% weak enough to be a lower bound that captures stack safety for any
% reasonable enforcement mechanism.

% \apt{This is a good story, but does it describe the paper we are able to produce this week?
%   We were rejected before exactly because we claimed to have
%   a general criterion, but did not in fact demonstrate its generality (which is not
%   obvious). What has changed?}\bcp{I agree with these concerns. Deleted the
%   para about the general story.  Does the rest now avoid making (or
%   implying) claims that we can't defend?}
% \bcp{Also, I think we should save
%   WBCF for the intro---I can't find a way to discuss it here that is brief
%   and not confusing/distracting.}

  % \iftrue\bcp{should we mention stktokens here?}
  % \leo{Not sure. We almost never cite stuff in the abstract, do we?}
  % \bcp{Some people say not to, but I think their argument is weak: I've
  %   added Nick/Andre above.  But I'm not sure we want to get into stkTokens
  %   in the abstract.}
  % \fi
\end{abstract}


%% 2012 ACM Computing Classification System (CSS) concepts
%% Generate at 'http://dl.acm.org/ccs/ccs.cfm'.
\begin{CCSXML}
<ccs2012>
<concept>
<concept_id>10011007.10011006.10011008</concept_id>
<concept_desc>Software and its engineering~General programming languages</concept_desc>
<concept_significance>500</concept_significance>
</concept>
<concept>
<concept_id>10003456.10003457.10003521.10003525</concept_id>
<concept_desc>Social and professional topics~History of programming languages</concept_desc>
<concept_significance>300</concept_significance>
</concept>
</ccs2012>
\end{CCSXML}

%\ccsdesc[500]{Software and its engineering~General programming languages}
%\ccsdesc[300]{Social and professional topics~History of programming languages}
%% End of generated code


%% Keywords
%% comma separated list
\ifcameraready
\keywords{Stack Safety, Micro-Policies}  %% \keywords are mandatory in final camera-ready submission
\fi


%% \maketitle
%% Note: \maketitle command must come after title commands, author
%% commands, abstract environment, Computing Classification System
%% environment and commands, and keywords command.
\maketitle

\section{Introduction}

\newcommand*{\MemoryLabel}[3]{\raisebox{#2}{\makebox(0,0){\hspace{#1}#3}}}
\newcommand{\memoryaddrs}[2][]
  {
    \MemoryLabel{#2}{2em}{$\downarrow$\SP #1}
  }

%\begin{figure}
%  \begin{minipage}{\textwidth}
%    \begin{center}
%\MemoryLabel{10em}{1.5em}{40}
%\MemoryLabel{16em}{1.5em}{60}
%\MemoryLabel{22em}{1.5em}{80}
%$\cdots$
%\memory{4}{gray}%
%\memory{5}{green}[{\makebox[0pt]{Caller}}]%
%\memory{5}{yellow}[{\makebox[0pt]{Args}}]%
%\memory{8}{red}[{\makebox[0pt]{Callee}}]%
%\memory{4}{\unusedc}%
%~$\cdots$\\
%  \end{center}
%  \end{minipage}\\
%~\\
%~\\
%  \begin{minipage}{0.45\textwidth}
%    \begin{center}
%\begin{verbatim}
%jal ...
%lw r1 r0 40
%\end{verbatim}
%    \end{center}
%  \end{minipage}~
%  \begin{minipage}{0.45\textwidth}
%    \begin{center}
%\begin{verbatim}
%jal ...
%sw r0 r0 40
%\end{verbatim}
%    \end{center}
%  \end{minipage}
%  \caption{Examples of Unsafe Programs\bcp{The numbers don't line up with
%      the boxes; the program fragments need comments explaining what each
%      line does.}}
%  \label{fig:stackunsafety}
%\end{figure}
%
%\leo{I hate the writing here, but I wanted to make a start.}

%\bcp{We don't actually say very clearly anywhere that we are interested in
%  low-level enforcement mechanisms for machine-code programs (because this
%  is where stack attacks occur, and because at higher levels of abstraction
%  the stack is a built-in concept, not something that can be attacked even
%  in principle).  This could cause confusion.\apt{not so worried}}
%\leo{I agree. I've clarified the writing in the top of the running example
%  a bit to make sure though}

The call stack is a perennial target for low-level attacks, leading to
consequences
from leakage or corruption of private stack data to
control-flow hijacking. To deter or detect such attacks, a profusion of
software and hardware protections have been proposed,
%
including stack canaries~\citep{Cowan+98},
bounds checking~\citep{NagarakatteZMZ09,NagarakatteZMZ10,DeviettiBMZ08},
split stacks~\citep{Kuznetsov+14},
shadow stacks~\citep{Dang+15,Shanbhogue+19},
capabilities~\citep{Woodruff+14,Chisnall+15,Skorstengaard+19,Skorstengaard+19b,Tsampas+19,Georges+21},
and hardware tagging~\citep{DBLP:conf/sp/RoesslerD18}. \ifaftersubmission\apt{Mostly from
  nick; there could be more}\bcp{Yes, going back to MIT days---we should
  include several more of these, if only to give readers the impression that
this is a well-studied mechanism (so formalizing its protections is
useful).}
\fi
%
The protections offered by such mechanisms are commonly described in terms
of concrete examples of attacks that they can prevent---corruption
of return addresses, buffer overflows, use of uninitialized variables,
etc.---leaving a more abstract characterization to the reader's intuition.
But these mechanisms can be quite intricate,
and they often require trading off protection and efficiency.
Thus, it would be useful to have a precise, generic, and formal
specification for stack
safety, both as a basis for comparing the security claims of different
enforcement techniques and for validating more rigorously that these claims
are met by particular implementations.

%\bcp{Does use of unitialized variables lead to stack safety attacks?\apt{sure, via buffer overflow}}
%with the notion usually being
%defined negatively: through examples of stack-based exploits
%\rb{usually though not always; as noted below there are other
%  approaches}\bcp{Yes, we have to be very careful about this}.
%\leo{Examples at a high level}: such as -- does not overwrite return addresses, or ...
The most abstract characterization stack safety that we are aware of is
recent work by \citet{Skorstengaard+19b,Skorstengaard+19}, which defines stack safety as the conjunction of two
properties:
{\em local state encapsulation (LSE)} and {\em well-bracketed control flow (WBCF)}.
Informally, LSE says that the contents of
a caller's stack frame are not read or written while its callees (and their callees, etc.) are
executing, and WBCF says that callees always
return to the instruction following their call (if they return at all).
Skorstengaard et al.{} formalize these properties by defining an idealized
machine in which they hold {\em by construction};
they then show that an enforcement
mechanism (in their case, hardware capabilities) guarantees stack safety, by
proving that an ordinary machine with enforcement simulates the idealized
machine.
%an enforcement mechanism (in their case, hardware
%capabilities)
%can then be shown to guarantee stack safety by exhibiting a fully abstract
%embedding function
%from the idealized machine to an ordinary machine equipped with this
%enforcement mechanism.  \bcp{Many OOPSLA readers/reviewers will not
%  understand that sentence.}
%%
%% While this work represents an important step in the right direction, we believe
%% there is still room to craft an operational characterization
%% of stack safety. \apt{Perhaps our rhetoric should be more micro-policy-specific here?}
%% In other words, we seek formal criteria that can be applied
%% to a mechanism for protecting stack-based low-level programs to judge
%% whether or not it actually ``guarantees stack safety.''
%% \rb{The remainder of this paragraph may be diving too quickly, would a reader be
%% confused by the references to these types of properties, yet to be introduced?}
%% We begin with a strong
%% local state encapsulation property that closely models the behavior of eager
%% micro-policies from \citet{DBLP:conf/sp/RoesslerD18}. This ``eager'' property
%% is too strong to characterize stack safety; reasonable enforcement mechanisms
%% exist that violate it. One such mechanism is a ``lazy'' tag policy, similar to
%% the one in \citet{DBLP:conf/sp/RoesslerD18}. We weaken local state encapsulation
%% to support this technique and argue that the new property, defined relative to an
%% arbitrary notion of observation. This property is still strong enough to prevent
%% the bad behaviors we associate with unsafe stack behavior, but weak enough that
%% other reasonable enforcement mechanisms can enforce it.

In the present work, we also aim to protect both stack data (like
LSE) and
control flow (like WBCF).  But our formulations of these properties are
rather different from those of Skorstengaard et al.  Instead of
defining stack safety implicitly,
as ``what a certain idealized machine does,'' we define it
\emph{explicitly} using the technical framework of language-based
security\ifaftersubmission~\citep{??}\bcp{Maybe the Sabelfeld and Myers
  survey?}\fi.
Our core properties are \emph{stepwise stack integrity},
\emph{stepwise stack confidentiality}, and \emph{return to caller}.
%
A key technical novelty in the definitions of confidentiality and integrity,
compared to standard formulations from the security literature, is that they
are ``nested'': {\em each} caller is guaranteed protection from its
immediate callees (which, in turn, need protection from their immediate
callees, etc.).

% This approach is novel in that the
% confidentiality\bcp{only?} property applies to many nested sub-traces within a single execution,
% as opposed to traces of entire executions.
% {\em Integrity} says that each caller's local data is protected from
% modification by callees,
% motivated by the reasoning principle that local data persists across a
% call.

For the confidentiality and integrity properties, we consider both {\em
  stepwise} and {\em extensional} variants.  The simpler {stepwise} variants
formalize the intuition that a caller's local data is {\em never} read or
modified during a call. The extensional variants allow a callee to read from
from and write to their caller's stack frame, as long as nothing that they
read affects their observable behavior and nothing they write affects the
observable behavior of the caller after they return.  \bcp{Please check: is
  that accurate?}

% {\em Confidentiality} says that a caller's
% data is not leaked into its callee, and  that the callee cannot see uninitialized data.
% As a reasoning principle, confidentiality guarantees that the callee's
% behavior is determined only by a predictable interface (function arguments and
% register contents) that does not include the caller's
% local state or the uninitialized portion of the stack. This is formalized as a
% {\em noninterference}-style property that, rather than comparing pairs of whole program traces,
% compares a pair of sub-traces for each call, within other calls\bcp{?}. Once
% again
% {\em stepwise} describes the granularity of enforcement\bcp{?}: at every step of a call,
% changes in the state are independent of the caller's data.

% Together, these properties capture roughly the same intuition as LSE, but
% they are somewhat stronger.
% First, we model returns as returns to a valid return target\bcp{that will
%   make no sense to the reader at this point}; an invalid return
% is treated as a mere jump within the callee, and does not grant access to the caller's
% data. So, even without WBCF, our stack safety properties rule out many control-flow attacks.
% The attacker in such an instance is permitted to ``return'' so long as it avoids
% violating the integrity and confidentiality of the caller.\bcp{ditto} Second, confidentiality
% contains\bcp{?} temporal protections that LSE does not necessarily address\bcp{?}. The intuition
% for LSE is that it protects the caller's stack frame, not necessarily memory that was
% previously allocated but now outside of the caller's frame. (Cheri-capability-based
% implementations of LSE {\em do} protect previously allocated data, typically by clearing
% it, but because it is a potential vector for leaking capabilities into the caller's frame,
% not as a goal of the property.\bcp{Oof.  We need to do better than that!})

\bcp{I trimmed away a ton of text here that I thought was getting too much
  into the weeds for the introduction.  However, in the process I lost the
  explanation of how our properties are a bit stronger than LSE.  One way to
put it back would be simply to add to the contributions a claim that we are
a bit stronger than them, with a pointer to where we discuss this in detail.}

To demonstrate the utility of our formal characterization, we use these
properties to validate and improve an existing enforcement mechanism, the
{\em stack-safety micro-policies} of~\citet{DBLP:conf/sp/RoesslerD18}.  We
use QuickChick~\citep{Denes:VSL2014,Pierce:SF4}, a property-based testing
tool for the Coq proof assistant, to generate many random programs and check
that Roessler and Dehon's micro-policies correctly abort the ones that
attempt to violate one of our properties; furthermore, we introduce several
intentionally broken micro-policies and check that the testing framework is
able to generate counterexamples that violate our properties but are not
halted by the incorrect enforcement variants.  (Showing that the testing
framework detects intentional bugs in enforcement mechanisms increases our
confidence that it can also catch unknown ones, strengthening our confidence
in the enforcement mechanism itself.\ifaftersubmission\bcp{Maybe that
  sentence belongs in the testing section?}\fi)

We test against their {\em Depth Isolation} micro-policy, in which memory cells within each stack
frame are tagged with the identity of the function activation
that owns the frame, and access to those locations is then gated
according to the activation that is currently executing. This is simple and corresponds
to intuitive notions of stack safety, and we argue that by describing its behavior
we approach a reasonable stack-safety property that other micro-policies can be
validated against.

% Randomized testing is a relatively low-cost way to validate a potential enforcement
% mechanism; it finds flaws in the enforcement fast enough to be integrated into the workflow
% of enforcement development \citep{TestingNI:ICFP}. To demonstrate that the testing adequately
% covers potentially erroneous programs, we test intentionally buggy policies alongside
% the real one.

A second challenge is that enforcement mechanisms may weaken their
protections\bcp{``weaken their protections'' doesn't convey what we mean
  very well}
in the name of efficiency, while arguing that they still capture stack safety in a meaningful sense.
\citet{DBLP:conf/sp/RoesslerD18} explore alternative micro-policies, with different
ways of identifying functions and deciding exactly which accesses are allowed.
These vary in their efficiency and in the spatial and temporal precision of their
security guarantees.
The Depth Isolation policy prevents a callee from writing outside its frame
at all, which is simple and intuitively correct, but it must manually clear the tags
on the frame on each return, which is very expensive. Conversely, their
\emph{Lazy Tagging and Clearing} policy improves
performance by assigning locations to the active function when they are written, which
permits  ``illicit'' writes to inaccessible stack frames so long as the
written locations are not subsequently read by their ``real'' owners. This violates our
straightforward properties, but ntuitively,
this form of laziness is justified if any dangerous writes will
ultimately be detected before they can affect program behavior.

\bcp{This discussion is getting too long for the intro.  Could it be trimmed.}
What does ``affecting program behavior'' really mean? This micro-policy
calls for a second characterization of stack safety in terms of the externally
observable behavior of a stack-safe system. Such a characterization will have the
additional benefit of being more {\em extensional}, concerned primarily with the
external behavior of the system. In contract, our stepwise properties are {\em intensional},
focusing on internal details that may not actually matter to overall security.

\bcp{Ditto, even moreso.}
To formalize this intuition, we develop an \emph{observational integrity} property which
characterizes enforcement mechanisms that detect violations between the point when
they occur and the point when they make an observable impact. Intuitively this property
can be phrased as ``no visible reads after dangerous writes.''  We prove (in Coq) that stepwise
integrity implies observational integrity. \sna{TODO: check this.}

We additionally sketch an associated {\em observational confidentiality} property
makes confidentiality extensional in the same way, in order to form an
{\em observational stack safety} that is highly extensional.
We argue that, despite its somewhat greater complexity,
it may be a better specification for
stack safety enforcement mechanisms in general.

%\apt{more to say?}\bcp{Yes---partly to underscore our claim that this is a
%  significant contribution, and partly to give the reader some context for
%  why this is part of our story.  E.g., ``One nice thing we can do with our
%  spec is to use it to evaluate the claims of existing or future proposals
%  for enforcement mechanisms.  One relatively inexpensive way to carry out
%  such an evaluation is to see whether property-based testing against this
%  spec can distinguish correct implementations of the mechanism from buggy
%  ones...''}
As it turns out, \citeauthor{DBLP:conf/sp/RoesslerD18}'s Lazy Tagging and
Clearing micro-policy enforces observational integrity, but not confidentiality,
whether stepwise or observational.
But formalizing the ideal property allows us to see how it falls short -- a specific
violation of the temporal aspect of confidentiality occurs in corner cases where callees
can leak data between one another.
We propose  a variant of Lazy Tagging and Clearing that we believe \emph{does} enforce
confidentiality, albeit at some performance cost.  \bcp{Did Nick and Andre
  know about this deficiency?  Did they mention it in their paper?  (If
  neither, we should make a bigger deal out of the fact that we discovered it.)}

Our initial description of these properties assumes a single simple stack with no sharing
between callers and callees: all parameters and return values are passed in registers.  Later,
we show how to refine it to allow passing of scalar stack data, granting a
callee access to the caller's data. We do not model more sophisticated sharing,
such as taking the address of a variable and treating it like a pointer.

Finally, we define integrity and confidentiality properties for a coroutine system featuring
a static layout, in which each coroutine's stack is constrained to a fixed region of memory.
Though simplistic, this extension is a step toward concurrency and demonstrates that
our approach is flexible enough to handle more sophisticated styles of
control flow.

%% \paragraph*{The Use of Stack Safety}

%% How does the programmer rely on stack safety? The stack makes implicit promises about
%% the behavior of function calls. First, that when a call is finished, the caller can continue
%% from where it left off with its local state unchanged. And second, that the call's
%% behavior is determined only by a predictable interface that does not include the caller's
%% local state.

%% %The most natural example is the first
%% %in figure \ref{fig:examples1}, in which {\tt main} calls stores 42 to a local variable {\tt x},
%% %calls {\tt f}, and then prints {\tt x}. Here the programmer should be able to expect that
%% %{\tt main} will print 42, but in C such an assumption may be wrong, if {\tt f} should
%% %overwrite {\tt main}'s stack data.

%% Both expectations imply predictable control flow, termed {\em well-bracketed
%% control flow} (WBCF) by \citeauthor{Skorstengaard+19}. Calls always jump to valid entry
%% points and return to their call sites. Then the data portion of the first expectation is for
%% {\em integrity}: that a caller's local data is protected from modification by its callee.
%% This is vital for local reasoning about the behavior of the caller. The security implications
%% are significant: without integrity, a caller that branches on a local variable could be
%% manipulated by its callee into unexpected behavior.

%% Alongside integrity, we have {\em confidentiality}, in which a caller's data is protected
%% from leakage by its callee. As a reasoning principle, confidentiality tells us that to
%% predict the behavior of a call we need only consider a subset of the machine state.
%% And in security terms, we can guarantee the secrecy of local data that we haven't passed
%% even when calling a badly behaved function. Together, integrity and confidentiality
%% comprise \citeauthor{Skorstengaard+19}'s {\em local state encapsulation} (LSE) property.

%% It must be reiterated that LSE is of little use without WBCF: without WBCF, a callee
%% could subvert LSE by ``returning'' to arbitrary code; after the return LSE should permit
%% accessing the caller's state, but the execution would remain under the callee's control.

%% To phrase LSE and WBCF as security properties requires an explicit attacker model:
%% a clear understanding of who is being protected and from what. Most existing
%% stack protection mechanisms envisage scenarios where a callee is
%% influenced by external inputs to corrupt the stack in some way. To
%% avoid getting into the details of exactly how this happens when
%% formalizing our properties, we use an even stronger model: when we
%% state the protections that a function can rely on when it makes a
%% call, we assume that the callee can execute arbitrary instructions
%% until it returns. Thus we have no need to differentiate safe code
%% from adversarial --- all functions, and indeed all activations of the same function,
%% are mutually distrusting.
%% %
%% For this model to make sense, we need to know when a call or a return is taking
%% place. We assume that calls and returns are annotated in the code, so that
%% we can track the point at which control transfers from the protected caller
%% to the attacking callee --- and when that callee in turn becomes a protected
%% caller in its own right.

%% This ``nesting'' of properties is a key challenge in adapting the concepts
%% of language-based security to the stack setting. We address this issue by
%% turning any set of annotations into a dynamic model of {\em domains},
%% in which some stack locations are sealed by the caller that needs to preserve
%% them while the rest are unsealed and available for use by the active, attacking
%% callee.

%% First we introduce this model for a single simple stack with no sharing between callers and
%% callees. Then we expand it to allow passing of stack data, granting a callee one-time
%% access to the caller's data. We do not model more sophisticated sharing, because to do
%% so realistically would require a separate protection scheme, such as a capability
%% model. While this is quite feasible in a tag-based enforcement mechanism, it is properly
%% a separate policy outside the scope of stack safety. (A tag-based approach can easily
%% compose multiple such policies.)

%% Finally, we extend the model to a coroutine system featuring finitely-bounded stacks.
%% Though simplistic, this extension is a step toward concurrency and demonstrates that
%% the model is flexible enough to handle more sophisticated styles of control-flow.

%% After formally defining stack safety, we show that the definition is
%% enforceable using an existing tag-based enforcement
%% mechanism~\citep{DBLP:conf/sp/RoesslerD18}. We first consider
%% \citeauthor{DBLP:conf/sp/RoesslerD18}'s conservative Depth Isolation policy.
%% This policy testably enforces our {\em eager} properties, in which integrity is respected
%% at every step of a program, and each call respects integrity at each of its internal steps.
%% These properties are very strong, and our policies enforce them; this also makes them very
%% intensional, in that they do not distinguish internal behavior from external.
%% They also have the advantage that they are suitable for efficient testing, as they
%% fail immediately when a violation occurs.

%% The downside of the Depth Isolation policy is that it is rather expensive to implement.
%% In the same work, \citeauthor{DBLP:conf/sp/RoesslerD18} propose a different mechanism
%% that is more efficient by virtue of being lazy in its enforcement.
%% Instead of enforcing the integrity and confidentiality of the stack at the
%% exact moment when a callee returns, this policy signals violations when the
%% caller actually accesses data that the callee has written into its stack
%% frame. \iftrue\bcp{We could also hint at the lazy confidentiality
%%   property in the same way...}\fi
%% The lazy policy does not enforce our eager integrity property, and to our
%% knowledge there is no formal statement of the properties that it ought to enforce.
%% We propose {\em lazy} stack-safety properties, which capture the desired behavior of a
%% policy that detects integrity violations between when they occur and when they make
%% a visible impact. Intuitively this can be phrased as ``no visible reads after dangerous
%% writes.'' Unfortunately \citeauthor{DBLP:conf/sp/RoesslerD18}'s existing
%% ``Lazy Tagging and Clearing'' policy does not quite enforce this property, as we show in
%% testing, but formalizing the ideal property allows us to see how it falls short.
%% We provide a variant of ``Lazy Tagging and Clearing'' that does enforce lazy integrity
%% at some performance cost.
%% %We prove (in Coq) that the eager property implies the lazy one.\apt{do we?
%% %  and is it so interesting anyhow?}
%% We argue that any system purporting to enforce stack safety ought to obey at least lazy
%% stack safety, making it a plausible extensional characterization.

%% %\bcp{I worry that we are being a little
%% %  dishonest here: the stated advantage of the lazy policy is that it is more
%% %efficient, but our variant is {\em not} efficient!}
In summary, we offer the following contributions:

\begin{itemize}
\item
  We adapt formulations of integrity and confidentiality from language-based security
  to formalize a parameterized {\em stepwise stack-safety} property and argue that it
  captures the intuitive concepts of stack safety~(\cref{sec:lse}). Our model extends to
  argument passing on the stack and to a simple coroutine system.
\item
  We use a property-based randomized testing framework to support the claim
  that this stepwise property is actually enforced by the Depth Isolation policy
  of~\citet{DBLP:conf/sp/RoesslerD18}~(\cref{sec:testing}).
\item
  We formalize a weaker but more extensional \emph{observational} variant of the
  stack safety property and argue that it accurately characterizes
  the Lazy Tagging and Clearing policy of~\citet{DBLP:conf/sp/RoesslerD18}
  enhanced with unique tags on each activation~(\cref{sec:lazy}).
\end{itemize}

We begin with a description of our setting (\cref{sec:setup}) and threat
model (\cref{sec:threat}),
then show how our properties apply to examples of stack unsafety
(\cref{sec:running-example}). Then, after some technical preliminaries
(\cref{sec:prelim}), we formalize the properties (\cref{sec:eager}\bcp{??}, \cref{sec:lazy})
and describe testing (\cref{sec:testing})\bcp{The sections are out of order?}.
\Cref{sec:relwork} discusses related work, and
\cref{sec:future} sketches directions for future work.  \bcp{Where is 7?}

\section{Setting}
\label{sec:setup}

Our properties are defined in terms of an abstract machine model that makes very few
assumptions about the concrete machine that instantiates it.  We assume a register
machine with a program counter (\(\PCname\)); other
details of the machine's ISA are unimportant. We assume that we know ahead of time
the layout of memory, including where code lives and which addresses are within
the bounds of the stack. We do not model a heap; programs might choose to use memory
that is neither code nor stack to store global variables, as a heap, etc., but our
properties do not protect it or assume any particular behavior. We also do not
model run-time code generation.

Our stack-safety properties depend
critically on the notions of ``call'' and ``return,'' which are not fully explicit
in machine code%
%
\footnote{\bcp{Some useful text: A major challenge in characterizing stack
    safety formally is that, in a low-level machine without built-in stacks,
    we are at the mercy\bcp{?} of architectural\bcp{?}  choices that do not
    relate to the abstract notion of stack safety. Some implementations may
    have only a stack pointer while others add a frame pointer, in some
    systems return addresses and data live in separate stacks, and so on. We
    abstract over these choices as much as possible.  }}%
%
---the same instruction opcode may be used in some places as part of
a call or return sequence and in other places for other purposes.
%
So we assume we are given a machine language program {\em annotated} with the locations of
instructions representing calls, and we determine when a return has occurred by tracking
control flow and observing when we reach a {\em return target}. For a given state annotated
as a call, in a standard calling convention, another state fulfils its return target if
it has the same stack pointer and the program counter is at the instruction following
the call, but we abstract over different conventions.
%, returns, and function entry points\BCP{I
%  still want to talk about whether returns and entry points need to be
%  annotated.  To me, these annotations do not make intuitive sense: it is
%  the caller's responsibility to know whether the address that it is calling
%is really an entry point, and we can trust it to check this.  Similarly, it
%is not the callee's role to say when it is returning; it is the caller's
%prerogative to decide when it has been returned to!  I.e., from the caller's
%point of view, it has been returned to when control resumes at the
%instruction following the call with the SP reset to where it was at the
%point of the call. (This is the point where it wants to be able to access
%its own local state again and expected to find it unmolested.)}.
Although call annotations would typically come from a compiler together with
the machine code, we do not assume anything about their provenance.

We model the base machine as a step function over machine states, and a machine
enhanced with an enforcement mechanism (in the language of tag-based enforcement,
a {\em policy}) as a partial step function over machine states extended with some
auxiliary state, refining the base machine. When the policy-enhanced machine fails
to step, this is a {\em failstop}, halting the program before it can perform an
action that would violate the security property;
our properties are therefore naturally \emph{termination insensitive.}
This model will be made more precise in \cref{sec:prelim}.

Dividing policy state from machine state helps us reason about
enforcement mechanisms that keep a significant amount of data separate from the
primary execution data to inform security. Obviously tag-based systems do this,
with tags and extra policy state separated from and untouchable by application code.
Another significant example is the shadow stack, where a second stack
mirrors the real one, and a return may only occur if both stacks agree on the
target address. Typically the shadow stack is isolated using a separate protection
mechanism.

The separation is smaller in protection mechanisms that interact more directly
with application code. Some mechanisms might exhibit failstop behavior, but have
no protected policy state; others might never even failstop. Then the policy-enhanced
machine is simply identical to the original. This is the case for software-based
techniques.

\section{Threat Model}
\label{sec:threat}

Intuitively, stack safety means that the stack frame of each caller (and
those of its ancestor callers) is protected
from attempts by a confused or malicious callee (including all its
descendant callees)
to read or write its private data.
That is, the attacker model treats every caller as a defender and its callee as
an attacker, meaning that with nested calls, a single function invocation
may be viewed in both ways.

We adopt a strong threat model in which both caller and callee may execute
arbitrary machine code. Hardware and timing attacks are out of scope.

\paragraph*{Well-formedness and Bad Annotations}
\bcp{What is ``well-formedness''??}

\BCP{I believe we could delete this section---and never mention WBCF at all
  (except to say that we leave it out, compared to STkTokens)---if we
  eliminated return (and entry point) annotations.}
\sna{I'm not sure we can escape questions about annotations, even if we have
  just calls, but I tried to abbreviate it.}

Since we rely on call annotations\bcp{to...}, we must ask: what if the annotations
are wrong? First, suppose a call occurs that is not annotated.
In this case, the answer is simple: as far as the properties are concerned, there has
been no call, only an instruction sequence resembling one, and
therefore no associated change in permissions. Then the caller is guaranteed no
protections.

\bcp{Confusing, and maybe out of date?}Alternatively, an annotated instruction might not obey the calling convention.
In the context of \citeauthor{DBLP:conf/sp/RoesslerD18}'s micro-policies, well-behaved calls
and returns are implemented by fixed sequences of instructions which the tag policy
will force to be executed from start to finish, termed {\em blessed sequences}.
When applying our properties to a micro-policy setting, we will statically rule out some
scenarios with a well-formedness criterion on initial states: all annotated instructions will
be situated at the appropriate instruction of a blessed sequence, tagged according to the policy.
This is all we need for the policy to protect itself: it can then failstop if it would
execute the blessed sequence out of order.

%We model ordinary program state and policy-enforcement state separately to make it
%easier to use well-established concepts from language-based security.
%In particular, confidentiality has an elegant extensional characterization in terms of
%\emph{noninterference}: a program preserves the confidentiality of certain data if
%\emph{varying} that data would not change the program's observable
%behavior~\citep{6234468}.
%In applying this idea to a system that incorporates policy enforcement, only
%ordinary program data should be varied, while any associated policy
%state should be left alone; our model makes this distinction obvious.
%That is, we compare the actual execution of the program with a \emph{hypothetical}
%execution where data has been varied.
%We will also introduce a notion of {\em observational integrity} that relies on the idea of
%\emph{rolling back} program state; again, our model makes it easy to describe
%rollback of ordinary program data while leaving policy state alone.

%% \leo{Policies of interest: null, micro, cherri, software-only}
%% \leo{TODO: Write down carefully}

%This enforcement model directly describes micro-policies~\citep{pump_oakland2015}.
%We believe that it is flexible enough to capture code-altering approaches, such as code
%rewriting or capability-based techniques, though perhaps with less elegance.

%\paragraph*{Micro-Policies}
%
%Micro-Policies are a flexible tag-based, hardware-accelerated
%reference monitoring mechanism; they been applied to stack safety
%enforcement by \citet{DBLP:conf/sp/RoesslerD18}.
%Here the policy state consists of metadata tags (e.g., identifying stack frames)
%attached to each value in memory or registers, and the policy step
%function checks that each machine operation obeys a set of rules on tags
%(e.g., that the current $\PCname$ tag matches the tag on the stack
%location being accessed), and halts the machine if not.  Distinguishing
%ordinary machine state and policy state is very natural in this setting.
%% Here the machine state is extended with tags paired with all components of the
%% system, and a monitor checks for each operation that the tags on the operands
%% obey a set of rules. One could treat tags directly as part of the machine
%% state, and build the rules into the step function. But local state
%% encapsulation properties as we frame them are not intuitive in their
%% interaction with tags or other additional enforcement state, so it is helpful
%% to separate tags for clarity. This also distinguishes a hardware fail-stop from
%% a software fail state.
%% \rb{TODO Similarly to above, should we consider tags as part of the machine or part of the
%%   policy? }


%\iftrue\apt{This still seems wrong to me. The policy part needs somehow to describe everything that isn't varied or rolled back.}\fi
%These approaches are represented as policies that
%accept\bcp{what does it mean for a policy to accept a state??} only initial machine states that are well-formed for the policy (due to
%being produced by the compiler.) The set of policy states is again the unit type
%and the step function constant. Property enforcement comes from the fact that
%only ill-formed states can induce a trace that violates the property.


%% \Paragraph*{Null policy}
%% %
%% The simplest possible policy is the neutral policy that carries no information
%% and does nothing. Its set of policy states is the unit type, and its policy step
%% function is the constant unit function, thereby allowing all calls to the machine
%% step function. Its initialization function is defined on all states. This
%% corresponds to the lifting of our machine model to the policy setting. Machines
%% that fulfill a property by construction will also fulfill that property with the
%% null policy.

%
%\leo{TODO: Add citations of mitigation techniques}
%\rb{And for those techniques, explain how they reflect their motivating
%examples. What does each technique do for those examples? How are they
%connected to this work?}
%
%\paragraph*{Bad Write}
%
%\[
%44 : [\rsp - 5] \leftarrow r_4
%%44 : ~ \sw ~ \rsp ~ r_4 ~ \negate 5
%\]
%
%Instruction 44 currently is
%$ [\rsp - 2] \leftarrow r_4$,
%%$\sw ~ \rsp ~ r_4 ~ \negate 2$,
%with the
%effect of storing the contents of $r_4$ as the result of {\tt g}. If
%instead it wrote to location $\rsp - 5$ then it would be overwriting
%private data (the return address) belonging to {\tt f}.
%
%\paragraph*{Bad Read}
%
%\[
%30 : r_5  \leftarrow [\rsp - 6]
%%30 : ~ \lw ~ r_5 ~ \rsp ~ \negate 6
%\]
%
%Instruction 30 currently is
%$ r_5 \leftarrow [\rsp - 4]$,
%%$\lw ~ r_5 ~ \rsp ~ \negate 4$,
%with the
%effect of loading the argument of {\tt f} to $r_5$. If instead it
%loaded the contents of location $\rsp - 6$, it would obtain access
%to the value of {\tt main}'s local variable {\tt x}. Worse, since the next instruction
%outputs the contents of $r_5$, that value would immediately be made public.
%
%\paragraph*{Bad Control Flow}
%
%\[
%5  : \rra \leftarrow \PCname + 1 ; ~ \PCname \leftarrow r_4
%%5 : \jalr ~ r_4 ~ \rra ~ 0
%\]
%If a {\jalr} instruction was added to {\tt main} with its target being
%the middle of the code block of {\tt g}.\apt{describe result of this?}

\section{Key Ideas by Example}
\label{sec:running-example}
\colorlet{lgray}{gray!40}
\colorlet{lred}{red!40}
\colorlet{lblue}{blue!20}

\newcommand{\mainsealc}{cyan}
\newcommand{\fsealc}{green}
\newcommand{\unsealc}{lgray}
\newcommand{\emptyoutc}{white} % or gray for consistency with unusedc ?
\newcommand{\fulloutc}{white}
\newcommand{\badc}{lred}
\newcommand{\goodc}{lblue}
\newcommand{\retptrc}{black}

\begin{figure}

\begin{subfigure}{.2\textwidth}
{\small
\begin{verbatim}
int main() {
  int x = 42;
  return f()+x;
}

int f() {
  int y;
  *(&y-2) = 0;
  return 5;
}
\end{verbatim}
}
\end{subfigure}
\begin{subfigure}{.5\textwidth}
{\small
\begin{verbatim}
1 main: add $1,%sp      ; allocate frame
2       mov $42,-1(%sp) ; initialize local
3       call f
4       add -1(%sp),%ra ; set return value
5       sub $1,%sp      ; deallocate frame
6       ret

10 f:   add $1,%sp      ; allocate frame
11      mov $0,-3(%sp)  ; violation!
12      mov $5, %ra     ; set return value
13      sub $1,%sp      ; deallocate frame
14      ret
\end{verbatim}
}
\end{subfigure}
\begin{subfigure}{.25\textwidth}
\begin{center}
\begin{tabular}{l l}
{\tt 2} &
\memoryaddrs{8em}
\memory{3}{\unsealc}[Unseal]
~$\cdots$
\vspace{.5em}
\\
{\tt 3} &
\memoryaddrs{8em}
\memory{3}{\unsealc}
~$\cdots$
    \MemoryLabel{-15em}{0.75em}{42}
    \vspace{.5em}
\\
{\tt 11} &
\memoryaddrs{16em}
\memory{1}{\mainsealc}[Seal(0)]%
\memory{1}{\retptrc}[RetPtr]%
\memory{1}{\unsealc}[Unseal]
~$\cdots$
\MemoryLabel{-15em}{0.75em}{42}
\vspace{.5em}
\\
{\tt 12} &
\memoryaddrs{16em}
\memory{1}{\mainsealc}
\memory{1}{\retptrc}
\memory{1}{\unsealc}
~$\cdots$
\MemoryLabel{-15em}{0.75em}{\bf 0}
\vspace{.5em}
\end{tabular}
\end{center}

\vspace{\abovedisplayskip}
\end{subfigure}
\caption{An integrity violation\bcp{Can we align the C-like snippets with
    the corresponding assembly?}  \ifaftersubmission\bcp{I wonder if we
    could make the figures easier to read by putting very light, differently
  colored backgrounds under the different parts}\fi}
\label{fig:int1}
\end{figure}

This section introduces the intuition behind the properties through a series
of examples. Existing stack safety enforcement work mostly refers directly to examples
of bad behavior that it prevents; we will argue that: \bcp{Not sure this
  enumeration is useful.}

\begin{itemize}
\item examples that appear to violate stack safety violate at least one of our properties;
\item we do not find desireable behaviors that appear stack-safe yet violate our properties; and
\item our properties are testably implemented by stack safety micro-policies from the
  literature, and the micro-policies are not significantly more conservative
  (i.e., policy failstops usually correspond to imminent property violations).
\end{itemize}

We show our examples both in C syntax and in corresponding assembly code for a
simple two-address machine. The address of each instruction is shown to its left.
The assembly code reflects a simple compilation model in which
local variables are stored in the stack frame.
For the moment, arguments and return values are passed in registers.
The machine is word-addressed. The stack grows upward in
memory and {\tt \$SP} points to the first unused word above the top of the stack.
The {\tt call} and {\tt ret} instructions push and pop the return address from the stack,
implicitly adjusting the stack pointer.
At the right of each example, we show the layout of stack memory just before
the specified instruction is executed; each box represents a memory word, and
addresses increase to the right.

Unless otherwise noted\bcp{where is it ``otherwise noted''??}, colored boxes
indicate an assignment of the address to a
{\em domain} -- a label reflecting a particular set of access rights. Domains
are \(\unsealed\)\bcp{``unsealed'' would read better} (in the stack and accessible to writes), \(\sealed{\depth}\)
(in the stack and reserved for the caller at depth \(\depth\)), and \(\outside\)
(outside the stack entirely.) For our purposes, \(\outside\) will apply to global
variables. Domains are part of the {\em context} of a state.
We will introduce the remaining context as needed.

Figure~\ref{fig:int1} shows a simple integrity violation. It consists of a {\tt main}
function that performs a call to a function {\tt f}, which overwrites {\tt main}'s
variable {\tt x}. In the initial memory layout the entire stack is {\em unsealed},
meaning that it is eligible to be written to. At the call from {\tt main} to
{\tt f}, {\tt main}'s locals are {\em sealed} with {\tt main}'s depth,
in this case 0. Sealing is the fundamental integrity contract between the caller and callee:
{\tt main} expects that {\tt x} will be unchanged when it gets control back.
This gives us our simple, intuitive statement of {\em stepwise integrity}:
if a component is sealed before a step, it will be unchanged afterward. Stepwise
refers to the fact that the property must hold on every step. %Later we will
%loosen this condition in the interest of extensionality -- the caller only
%cares about changes it can see after the return, not changes during execution.
In this case, after instruction 8, {\tt f} has violated integrity by writing to {\tt x},
so the final return value produced by {\tt main} is wrong.

The black rectangle represents the return address, which is saved by the call
instruction itself, so is not sealed.

\begin{figure}
\begin{subfigure}{.2\textwidth}
{\small
\begin{verbatim}
int main() {
  int x = 42;
  return f()+x;
}

int f() {
  int y,z;
  y = 5;
  z = *(&y-2);
  return y+z;
}
\end{verbatim}
}
\end{subfigure}
\begin{subfigure}{.5\textwidth}
{\small
\begin{verbatim}
1 main: add $1,%sp      ; allocate frame
2       mov $42,-1(%sp) ; initialize local
3       call f
4       add -1(%sp),%ra ; set return value
5       sub $1,%sp      ; deallocate frame
6       ret

10 f:   add $2,%sp      ; allocate frame
11      mov $5,-2(%sp)  ; assign into y
12      mov -4(%sp),-1(%sp)  ; violation!
13      mov -2(%sp),%ra ; calculate and
14      add -1(%sp),%ra ; set return value
15      sub $2,%sp      ; deallocate frame
16      ret
\end{verbatim}
}
\end{subfigure}
\begin{subfigure}{.59\textwidth}
\centering
\begin{tabular}{l l | l}
  {\tt 2} &
  \multicolumn{2}{c}{
    \memoryaddrs{8em}
    \memory{4}{\unsealc}
    ~$\cdots$
%    \MemoryLabel{-18em}{0.75em}{0}
%    \MemoryLabel{-14em}{0.75em}{0}
%    \MemoryLabel{-10em}{0.75em}{0}
%    \MemoryLabel{-7em}{0.75em}{0}
    \vspace{.5em}
  } \\
  {\tt 11} &
  \memoryaddrs{21em}
  \memory{1}{\unsealc}
  \memory{1}{\retptrc}
  \memory{2}{\unsealc}
  ~$\cdots$
  \MemoryLabel{-19em}{0.75em}{42}
%  \MemoryLabel{-10em}{0.75em}{0}
%  \MemoryLabel{-6em}{0.75em}{0}
  &
  \memoryaddrs{21em}
  \memory{1}{\unsealc}
  \memory{1}{\retptrc}
  \memory{2}{\unsealc}
  ~$\cdots$
  \MemoryLabel{-19em}{0.75em}{\(v_0\)}
  \MemoryLabel{-10em}{0.75em}{\(v_1\)}
  \MemoryLabel{-6em}{0.75em}{\(v_2\)}
  \\
  {\tt 12} &
  \memoryaddrs{21em}
  \memory{1}{\unsealc}
  \memory{1}{\retptrc}
  \memory{1}{\goodc}
  \memory{1}{\unsealc}
  ~$\cdots$
  \MemoryLabel{-19em}{0.75em}{42}
  \MemoryLabel{-10em}{0.75em}{5}
 % \MemoryLabel{-6em}{0.75em}{0}
  &
  \memoryaddrs{21em}
  \memory{1}{\unsealc}
  \memory{1}{\retptrc}
  \memory{1}{\goodc}
  \memory{1}{\unsealc}
  ~$\cdots$
  \MemoryLabel{-19em}{0.75em}{\(v_0\)}
  \MemoryLabel{-10em}{0.75em}{5}
  \MemoryLabel{-6em}{0.75em}{\(v_2\)}
  \\
  {\tt 13} &
  \memoryaddrs{21em}
  \memory{1}{\unsealc}
  \memory{1}{\retptrc}
  \memory{1}{\unsealc}
  \memory{1}{\badc}
  ~$\cdots$
  \MemoryLabel{-19em}{0.75em}{42}
  \MemoryLabel{-10em}{0.75em}{5}
  \MemoryLabel{-6em}{0.75em}{42}
  \vspace{.5em}
  &
  \memoryaddrs{21em}
  \memory{1}{\unsealc}
  \memory{1}{\retptrc}
  \memory{1}{\unsealc}
  \memory{1}{\badc}
  ~$\cdots$
  \MemoryLabel{-19em}{0.75em}{\(v_0\)}
  \MemoryLabel{-10em}{0.75em}{5}
  \MemoryLabel{-6em}{0.75em}{\(v_0\)}
  \vspace{.5em}
\end{tabular}

\vspace{\abovedisplayskip}

\end{subfigure}
\caption{A confidentiality violation\apt{Could save space by not repeating C or assemly code for {\tt main}.}}
\label{fig:conf1}
\end{figure}

Meanwhile, confidentiality intuitively states that a callee is unaware\bcp{??}
of the context of its call (except for its arguments and any global
variables).
We state this as a kind of {\em noninterference} property~\citep{??}.
Noninterference describes knowledge: it states that for system components
containing secret values, we can substitute any values whatsoever without changing the results
of our computation. If there exists some value for which we see different results, then
in fact we can learn something about the initial state by observing those results, violating
confidentiality.

In Figure~\ref{fig:conf1} we see that the callee, {\tt f}, violates confidentiality
by reading {\tt main}'s variable {\tt x}. The trace
shows how noninterference applies to this scenario. At instruction 10, we create a {\em variant}
state with arbitrary values\ifaftersubmission\bcp{IMO, it would be easier to
understand if we chose concrete values for the examples}\fi \(v_0\), etc., in place of the values that should
be hidden from {\tt f}. At instruction 12, we note that the callee's first variable, {\tt y}, has
changed to 5 in both the primary trace and the variant. The primary and variant agree on
the value in the changed location, so confidentiality was not violated. But at instruction 13,
the other variable, {\tt z}, has changed. In the primary trace it changed to 42, but
in the variant trace it might have any value \(v_0\). If \(v_0 \neq 42\) then the variant
has behaved differently than the primary, violating confidentiality.

This form of confidentiality we will term {\em stepwise confidentiality}, defined informally
as: for any state at the entry to a function, for any variant over the values in the stack,
the primary state and its variant will step in lockstep until they return. To step in lockstep
means that if both states step -- that is, neither the primary nor the variant failstops --
any {\em system component} (such as memory cell, register, program counter) that changes
in one step has a matching value at the end of the other step. And when we
extend the model with a notion of external output, values that are outputted will need
to match as well.

%\bcp{Remove this para?}We alternatively could imagine a version of stepwise confidentiality in which we explicitly
%track which addresses may be read (i.e., which have been initialized), and at each step
%guarantee that the step alters state identically to a step from a variant state. Then we
%wouldn't need to divide up the execution by calls. But the approach we take is more
%consistent with a typical noninterference approach, in that it varies the memory once
%at the beginning and trace changes through the execution.

\begin{figure}

\begin{subfigure}{.2\textwidth}
  {\small
\begin{verbatim}
int main() {
  int x;
  x = f();
  return g();
}

int f() {
  int y = 5;
  return y;
}

int g() {
  int z;
  if (z == 5)
    return 10;
  else
    return 20;
}
\end{verbatim}
}
\end{subfigure}
\begin{subfigure}{.7\textwidth}
  {\small
\begin{verbatim}
1 main:  add $1,%sp      ; allocate frame
2        call f
3        mov %ra,-1(%sp) ; store x
4        call g
5        sub $1,%sp      ; deallocate frame
6        ret

10 f:    add $1,%sp      ; allocate frame
11       mov $5,-1(%sp)  ; initialize y
12       mov -1(%sp),%ra ; set return value
13       sub $1,%sp      ; deallocate frame
14       ret

20 g:    add $1,%sp      ; allocate frame
21       cmp $5,-1(%sp)  ; compare
22       brneq #25       ; branch if not equal
23       mov $10,%ra     ; set return value
24       jmp #26
25       mov $20,%ra     ; set return value
26       sub $1,%sp      ; deallocate frame
27       ret
}
\end{verbatim}
}
\end{subfigure}
\begin{subfigure}{.65\textwidth}
\begin{center}
\begin{tabular}{l r | l}
  {\tt 2} &
  \multicolumn{2}{c}{
    \memoryaddrs{8em}
    \memory{4}{\unsealc}
    ~$\cdots$
%    \MemoryLabel{-18em}{0.75em}{0}
%    \MemoryLabel{-14em}{0.75em}{0}
%    \MemoryLabel{-10em}{0.75em}{0}
%    \MemoryLabel{-7em}{0.75em}{0}
    \vspace{.5em}
  } \\
  {\tt 11} &
  \memoryaddrs{16em}
  \memory{1}{\unsealc}
  \memory{1}{\retptrc}
  \memory{2}{\unsealc}
  ~$\cdots$
 % \MemoryLabel{-19em}{0.75em}{0}
%  \MemoryLabel{-10em}{0.75em}{0}
%  \MemoryLabel{-6em}{0.75em}{0}
  &
  \memoryaddrs{16em}
  \memory{1}{\unsealc}
  \memory{1}{\retptrc}
  \memory{2}{\unsealc}
  ~$\cdots$
  \MemoryLabel{-19em}{0.75em}{\(v_0\)}
  \MemoryLabel{-10em}{0.75em}{\(v_1\)}
  \MemoryLabel{-6em}{0.75em}{\(v_2\)}
  \\
  {\tt 12} &
  \memoryaddrs{16em}
  \memory{1}{\unsealc}
  \memory{1}{\retptrc}
  \memory{2}{\unsealc}
  ~$\cdots$
 % \MemoryLabel{-19em}{0.75em}{0}
  \MemoryLabel{-10em}{0.75em}{5}
%  \MemoryLabel{-6em}{0.75em}{0}
  &
  \memoryaddrs{16em}
  \memory{1}{\unsealc}
  \memory{1}{\retptrc}
  \memory{2}{\unsealc}
  ~$\cdots$
  \MemoryLabel{-19em}{0.75em}{\(v_0\)}
  \MemoryLabel{-10em}{0.75em}{\(5\)}
  \MemoryLabel{-6em}{0.75em}{\(v_2\)}
  \\
  {\tt 4} &
  \multicolumn{2}{c}{
    \memoryaddrs{8em}
%    \memory{4}{\unsealc}
    \memory{1}{\unsealc}
    \memory{1}{\retptrc}
    \memory{2}{\unsealc}
    ~$\cdots$
    \MemoryLabel{-18em}{0.75em}{5}
    \MemoryLabel{-14em}{0.75em}{2}
    \MemoryLabel{-10em}{0.75em}{5}
%    \MemoryLabel{-7em}{0.75em}{0}
    \vspace{.5em}
  }
  \\
  {\tt 22} &
  \memoryaddrs{16em}
  \memory{1}{\unsealc}
  \memory{1}{\retptrc}
  \memory{2}{\unsealc}
  ~$\cdots$
  \MemoryLabel{-19em}{0.75em}{5}
  \MemoryLabel{-10em}{0.75em}{5}
 % \MemoryLabel{-6em}{0.75em}{0}
  &
  \memoryaddrs{16em}
  \memory{1}{\unsealc}
  \memory{1}{\retptrc}
  \memory{2}{\unsealc}
  ~$\cdots$
  \MemoryLabel{-19em}{0.75em}{\(v_3\)}
  \MemoryLabel{-10em}{0.75em}{\(v_4\)}
  \MemoryLabel{-6em}{0.75em}{\(v_5\)}
  \\ &
  {\bf 23}
  ~$\cdots$
  &
  {\bf 25}
  ~$\cdots$
  \\
\end{tabular}
\end{center}

\vspace{\abovedisplayskip}

\end{subfigure}

\caption{Another confidentiality violation \bcp{Again, let's align the C
    code with the assembly, etc.}}
\label{fig:conf2}
\end{figure}

It may be unclear in figure \ref{fig:conf1} why we vary the initial values of {\tt y}
and {\tt z} as well as that of {\tt x}. This is to detect violations of so-called
temporal stack safety.
In Figure~\ref{fig:conf2}, we watch as {\tt f} leaves behind a value in {\tt y}
that corresponds to {\tt main}'s variable, {\tt x}. Then {\tt g} is able to access
{\tt y} and hence {\tt x} without directly reading the stack pointer. In
this case, {\tt g} branches on that value, which is a different way of
violating lockstep confidentiality. Technically, this violation
occurs because the program counter is also a component of the state that must change
identically between the primary and variant traces. In a case where \(v_4 \not = 5\),
the program counters will differ, breaking the lockstep.\apt{Is there any reason
to complicate things like this by bringing in the PC?}\bcp{+1}


\paragraph*{Control-Flow Attacks}
\begin{figure}
  \centering
  \begin{subfigure}{.3\textwidth}
{\small
\begin{verbatim}
int *stash = 0;

int main() {
  int x = 1;
  f();
  x = -x;
  f();
  return x;
}

void f() {
  int y;
  if (!stash) {
    stash = *(&y-1);
  } else {
    *(&y-1) = stash;
    stash = 0;
  }
  return;
}
\end{verbatim}
}
\end{subfigure}
  \begin{subfigure}{.65\textwidth}
{\small
\begin{verbatim}
   stash: .word 0
1  main:  add $1,%sp        ; allocate frame
2         mov $1,-1(%sp)    ; initialize x
3         call f
4         neg -1(%sp)       ; x = -x
5         call f
6         mov -1(%sp),%ra   ; set return value
7         sub $1,%sp        ; deallocate frame
6         ret

10 f:     add $1,%sp        ; allocate frame
11        cmp stash, $0     ; compare
12        breq #15          ; branch if equal
13        mov -2(%sp),stash ; update stash
14        jmp #17
15        mov stash,-2(%sp) ; retrieve stash
16        mov $0,stash
17        sub $1,%sp        ; deallocate frame
18        ret
\end{verbatim}
}
  \end{subfigure}
  \begin{subfigure}{.4\textwidth}
    \begin{tabular}{l l l}
      {\tt 3} &
      \memoryaddrs{8em}
      \memory{3}{\unsealc}
      ~$\cdots$
      \MemoryLabel{-15em}{0.75em}{1} \\
      {\tt 18} &
      \memoryaddrs{12em}
      \memory{1}{\mainsealc}[Seal(0)]%
      \memory{1}{\unsealc}[RetPtr]%
      \memory{1}{\unsealc}%
      ~$\cdots$
      \MemoryLabel{-15em}{0.75em}{1}
      \MemoryLabel{-11em}{0.75em}{\#4}
      \vspace{.5em} &
      \memory[3em]{1}{\mainsealc}[Target]%
      \MemoryLabel{-3em}{0.75em}{\(\PCname\) = \#4}
      \\
      {\tt 4} &
      \memoryaddrs{8em}
      \memory{3}{\unsealc}
      ~$\cdots$
      \MemoryLabel{-15em}{0.75em}{1} \\
      {\tt 11} &
      \memoryaddrs{16em}
      \memory{1}{\mainsealc}
      \memory{2}{\unsealc}%
      ~$\cdots$
      \MemoryLabel{-15em}{0.75em}{-1}
      \MemoryLabel{-11em}{0.75em}{\#6} &
      \memory[3em]{1}{\mainsealc}
      \MemoryLabel{-3em}{0.75em}{\(\PCname\) = \#6}
      \\
      {\tt 18} &
      \memoryaddrs{12em}
      \memory{1}{\mainsealc}
      \memory{1}{\badc}
      \memory{1}{\unsealc}%
      ~$\cdots$
      \MemoryLabel{-15em}{0.75em}{-1}
      \MemoryLabel{-11em}{0.75em}{\#4} &
      \memory[3em]{1}{\mainsealc}
      \MemoryLabel{-3em}{0.75em}{\(\PCname\) = \#6}
      \\
      {\tt 4} &
      \memoryaddrs{8em}
      \memory{1}{\mainsealc}
      \memory{2}{\unsealc}
      ~$\cdots$
      \MemoryLabel{-15em}{0.75em}{-1} &
      \memory[3em]{1}{\mainsealc}
      \MemoryLabel{-3em}{0.75em}{\(\PCname\) = \#6}
      \\
      {\tt 5} &
      \memoryaddrs{8em}
      \memory{1}{\mainsealc}
      \memory{2}{\unsealc}
      ~$\cdots$
      \MemoryLabel{-15em}{0.75em}{\bf 1} &
      \memory[3em]{1}{\mainsealc}
      \MemoryLabel{-3em}{0.75em}{\(\PCname\) = \#6} \\
    \end{tabular}
  \end{subfigure}
  \caption{A control-flow attack \bcp{wonder if we can cram the bottom part
      onto the same row as the rest.  Might need to make things smaller,
      spill into the margin, etc., but we really need the space.  (Another
      option is to make it a separate figure and flow text around it, but
      this will save less space.)}}
  \label{fig:controlflow}
\end{figure}

%From the classic stack smashing attack to return oriented programming,
One of the most pernicious ways to attack the stack is by hijacking control flow.
Our stack safety properties do not assume that control flow follows a typical structure
of matching sets calls and returns nested within on another.
Rather, they aim to capture what it means to protect data even in the presence of control-flow attacks.
Consider example \ref{fig:controlflow}.
Here we show return addresses explicitly as line numbers, in the form $\#n$.
Function {\tt f} is called twice, at instructions 2 and 4; it should therefore return to instructions
3 and 5, respectively. But the first time, before returning, it stashes its
return pointer away, and the second time it returns to instruction 3 instead of 5; then it ends
up being called a third time, at which point it finally returns to instruction 5.

Reasoning about {\tt main}, we expect that {\tt x} is negated once, so that {\tt main} returns
-1. But because of {\tt f}'s interference, {\tt x} is negated twice, and {\tt main} returns 1.
This has the appearance of an integrity violation.  How do we capture this in our
integrity property?

The key intuition is that to be considered as having returned to {\tt main}, {\tt f} needs
to do more than just execute a return-labeled instruction: it must actually reach the
appropriate {\em return target} in {\tt main}. A return target is a predicate on states,
in this case one that holds on any state whose program counter points to the correct instruction.
We introduce a new piece of context: we keep a stack of return targets, and a callee is
only considered to have returned when it reaches a state that matches the caller's target.
A return target is associated with a set of sealed locations that will become unsealed
when it is reached, here represented by sharing a color with the sealed data.

In the example, on its first return, {\tt f} reaches its return target, because the
\(\PCname\) is 3 after its return. But {\tt f}'s second return is to the wrong instruction
(3 rather than 5) because the stack pointer has been overwritten (shown in light red).
Thus {\tt x} is still sealed and {\tt f}'s attempt to overwrite it at instruction 4
violates the integrity property.

\section{Machines and Traces}
\label{sec:prelim}

We now define our machine model precisely. To make our definition of stack safety
as generic as possible, this section proposes an abstract interface to a
machine model plus some kind of (software- or hardware-enforced) policy monitor.
In~\cref{sec:enforcement}, we instantiate this model with a concrete dynamic
policy enforcement mechanism based on \citet{DBLP:conf/sp/RoesslerD18}.

\subsection{Values and States}

The building blocks of the machine are {\em values} and {\em addresses}.
Both are drawn from some set of {\em words} \(\WORDS\), ranged over by \(\word\) and
\(\addr\), respectively.
%
Our machine states are composed of {\em components} \(\component\),
which are either addresses or register names (\(\reg\)) drawn from some set
\(\REGS\), which is assumed to include two special purpose registers: the
program counter {\PCname} and the stack pointer \(\SP\).
A {\em machine state} is a map from components to values.
%
    \[\component \in \COMPONENTS ::= \WORDS + \REGS \]
%
    \[\mach \in \MACHS ::= \COMPONENTS \rightarrow \WORDS\]
%
We have a total step function between machine states, written \(\mach \stepsto \mach'\).

% BCP: This will intentionally fail when someone adds the real intended
% definition of \WF!
\newcommand{\WF}{\mathit{WF}}

We now formalize the model of enforcement that we introduced in \Cref{sec:threat}.
A {\em policy} consists of a set of policy
states \(\pol \in \POLS\) and a policy step function \((\mach, \pol) \polstep \pol' \in
\MACHS \times \POLS \rightharpoonup \POLS\). This step function is partial;
it is undefined on input configurations that correspond to a policy fault. The policy
can see the machine state, but cannot modify it. The policy also provides a
well-formedness condition, \(\WF (\mpstatename \in \MACHS \times \POLS)\), describing
initial states that are properly configured.
%
A concrete policy based on \citet{DBLP:conf/sp/RoesslerD18} will be described in
\Cref{sec:enforcement}.

We lift the policy step function to operate on such pairs
by combining it with the regular step function for machine states.
%
\judgmenttwo{\(\mach \stepsto \mach'\)}{\((\mach, \pol)
               \polstep \pol'\)}
            {\(\mpstate{\mach}{\pol} \stepstopol
               \mpstate{\mach'}{\pol'}\)}
%
Note that the base machine has a total step function, while the policy-enhanced machine
refines it to a partial function. The extra policy state never changes behavior,
but it can cause
premature termination, representing failstop behavior.

\subsection{Contexts}

Separate from the machine state, we will keep track of additional context
information representing the history of the run. This has no influence on execution, it merely
describes information relevant to the particular property. Different information is needed
for different versions of the model, so the definitions are parameterized
over an arbitrary set of context states, \(\CONTEXTS\), annotations \(\ANNS\),
an initial context state \(\context_0 \in \CONTEXTS\), and a context step function
\(\mach, \context \constep \context' \in \MACHS \times \CONTEXTS
\times \rightarrow \CONTEXTS\). We also assume the existence of a \emph{code map},
\(\codemap \in \WORDS \rightharpoonup \ANNS\)
that relates each code address to its annotation, if any.
We lift contexts into an MP trace in much the same way as we lifted policy states to get
a triple state \(\mpcstatename \in \MPCS = \MACHS \times \POLS \times
\CONTEXTS\).

\judgmenttwo{\(\mpstate{\mach}{\pol} \stepstopol \mpstate{\mach'}{\pol'}\)}
              {\(\mach, \context \constep \context'\)}
              {\(\mpcstate{\mach}{\pol}{\context} \stepstocon
                \mpcstate{\mach'}{\pol'}{\context'}\)}

\subsection{Machine Traces}
\label{sec:traces}

A {\em machine trace} is a nonempty, finite or infinite sequence of elements
of MPC-states, ranged over by \(\MPCT\) and \(\NPCT\).
We use juxtaposition of an element and a trace to represent ``cons.'' (\bcp{Ugh})

The ``trace-of'' operator, written \(\mpcstatename \hookrightarrow \MPCT\),
coinductively relates an initial state with the trace of states
produced by repeated application of \(\stepstocon\)

\begin{center}
\begin{minipage}{.4\textwidth}
\judgmenttwo{\(\mpcstatename \stepstocon \mpcstatename'\)}
            {\(\mpcstatename' \hookrightarrow \MPCT\)}
            {\(\mpcstatename \hookrightarrow \mpcstatename' \MPCT\)}%
\end{minipage}
\begin{minipage}{.4\textwidth}
\judgment%[Default]
         {\(\not\exists \mpcstatename'. \mpcstatename \stepstocon \mpcstatename'\)}
         {\(\mpcstatename \hookrightarrow \mpcstatename\)}
\end{minipage}
\end{center}
%

We project out the machine state of an MPC-state with \(\pi_\mach\), the
policy state with \(\pi_\pol\), and the context with \(\pi_\context\).
We take the first element of a trace with \(\head(\MPCT)\), which is a total
function since traces are non-empty, and the final element (if one exists) with
\(\last(\MPCT)\), which is partial.

\paragraph*{Until}
The operation \(\PUT ~ f ~ \MPCT\) takes a trace \(\MPCT\)
and a predicate on states \(f ~ \mpcstate \in \MPCS\) and gives the prefix of
\(\MPCT\) ending with the first element on which \(f\) holds.

\begin{center}
  \begin{minipage}{.3\textwidth}
    \judgment{\(f ~ \mpcstatename\)}
             {\(\PUT ~ f ~ \mpcstatename\MPCT = \mpcstatename\)}
  \end{minipage}
%
  \begin{minipage}{.3\textwidth}
    \judgment{}
             {\(\PUT ~ f ~ \mpcstatename = \mpcstatename\)}
%
  \end{minipage}
  \begin{minipage}{.3\textwidth}
    \judgmenttwo{\(\neg f ~ \mpcstatename\)}{\(\PUT ~ f ~ \MPCT = \MPCT'\)}
                {\(\PUT ~ f ~ \mpcstatename\MPCT = \mpcstatename\MPCT'\)}
  \end{minipage}
\end{center}
%
If the resulting trace is a strict prefix of the parameter, then the
predicate must hold on its final element. If the predicate always holds, then
\(\PUT\) is the identity function.

We will frequently take a prefix of the trace from an initial state up
through the first state where some condition holds on the machine state.
This can easily be implemented using \(\PUT\), and we provide special notation
for convenience.
%
If \(f\) is a predicate on machine states, we define \(\mpcstatename
\hookrightarrow \MPCT | f\) (read ``\(\MPCT\) is the prefix of
running \(\mpcstatename\) up to \(f\)''):
%
\begin{center}
\judgmenttwo{\(\mpcstatename \hookrightarrow \MPCT'\)}
            {\(\PUT ~ f ~ \mpcstatename ~ \MPCT' = \MPCT\)}
            {\(\mpcstatename \hookrightarrow \MPCT | f\)}
\end{center}

\paragraph*{Context Segments}

individual calls from a trace by taking subtraces in which the stack is
at or below a given depth. If \(f\) is still a predicate on states, then
we define \(\mpcstatename \hookrightarrow \MPCT \langle f \rangle\), pronounced
``\(\mpcstatename\) segmented by \(f\) yields a sub-trace \(\MPCT\).''

\begin{center}
  \judgmenttwo[Now]
              {\(f ~ \mpcstatename\)}
              {\(\mpcstatename \hookrightarrow \MPCT | \neg f\)}
              {\(\mpcstatename \hookrightarrow \MPCT \langle f \rangle\)}
%
  \judgmentthree[Skip]
                {\(f ~ \mpcstatename\)}
                {\(\mpcstatename \hookrightarrow \MPCT | \neg f\)}
                {\(\last(\MPCT) \hookrightarrow \MPCT' \langle f \rangle\)}
                {\(\mpcstatename \hookrightarrow \MPCT' \langle f \rangle\)}
%
  \judgmentthree[Search]
                {\(\neg f ~ \mpcstatename\)}
                {\(\mpcstatename \hookrightarrow \MPCT | f\)}
                {\(\last(\MPCT) \hookrightarrow \MPCT' \langle f \rangle\)}
                {\(\mpcstatename \hookrightarrow \MPCT' \langle f \rangle\)}
\end{center}

\paragraph*{Reachability}

We often want to quantify over all traces from any initial state so that our
properties apply to the system as a whole. We say that a state \(\mpcstatename'\)
is {\em reachable} if for some initial state \(\mpcstatename = \mach,\pol,\context_0\)
where \(\WF ~ \mach,\pol\), \(\mpcstatename \hookrightarrow \MPCT\)
and \(\MPCT\) contains \(\mpcstatename'\). A {\it reachable segment} \(\MPCT \langle f \rangle\) is a subtrace
such that from some similarly well-formed initial state \(\mpcstatename\),
\(\mpcstatename \hookrightarrow \MPCT \langle f \rangle\).

\section{Stack Safety, Formally}
\label{sec:lse}

We are finally ready for our definition of stack safety. We begin
by describing how we model knowledge about the program structure,
then dive into the formal definitions of local state encapsulation and
well-bracketed control flow, which together constitute stack safety\bcp{not
  any more?}.
Initially we will focus on a simple stack model, with a single stack
that doesn't share between caller and callee. All arguments are passed
in registers. Recall that we keep track of which addresses are
annotated as calls, with a ``code map'':
\[\codemap \in \CODEMAPS \subseteq \WORDS\]
\apt{Doesn't match previous definition any more.}

We parameterize over the structure of the stack, particularly the means by
which a caller identifies the data that should be protected during a call. We term
this the {\em sealing convention}, \(\sealcon \subseteq \MACHS \times \COMPONENTS\), which
maps a machine state to a set of addresses. \apt{Is it a relation or a function (seems to
  be the latter)?}
If a state's \(\PCname\) is annotated as a call, the sealing convention tells us
which addresses must be protected until the return. Similarly, we abstract over what
it means to return with a {\em return convention}, \(\retcon \subseteq \MACHS \times \MACHS\),
which relates call states to the states that can be considered valid returns to that call.
The usual return convention is that \(\mach\) and \(\mach'\) agree on the stack pointer,
and \(\mach'\) has its program counter advanced by one instruction from \(\mach\).
But whatever the return convention, at each step we update the context according
to the context-step function, \(\stepsto_C\).\apt{abrupt transition}

Each system component is assigned to a {\em domain}, which may be {\em Outside}
of the stack entirely, {\em Unsealed} and therefore accessible, or {\em Sealed}
at a particular depth. Each caller registers a {\em target}
condition -- a predicate on machine states that must hold for it to consider itself active.
A context is a pair of a ``domain map'' from components to domains, and a
``return target list,'' which is a stack of targets, one for each caller awaiting
return.  \ifaftersubmission\bcp{spacing is funny}\fi
%
\[\stackDom \in \STKDOMS ::= \outside + \unsealed + \sealed{\depth} \hspace*{0.5in} \target \subseteq \MACHS\]
\[(\domMap,\rts) = \context \in \CONTEXTS ::= (\COMPONENTS \rightarrow \STKDOMS)
  \times (\listT (\target)) \]
%
The initial context \(\context_0 = \domMap_0, []\), where \(\domMap_0\) maps
the stack to \(\unsealed\) and all other addresses and all registers to \(\outside\).

The rules for \(\stepsto_C\) are given inductively, and apply in the listed order.
A call annotation tells us to push the return target onto the return stack and
seal all components with the depth of the previous return stack. If we aren't making a call,
we identify a return by looking ahead and checking if we're about to step to a state
that matches the topmost return target. If neither is true, the context remains unchanged.

\judgmentthree[Call]
              {\(\codemap ~ (\mach ~ \PCname) = \callmap\)}
              {\(\components = \sealcon ~ \mach\)}
              {\(\domMap' = \domMap[\components \mapsto \sealed{|\rts|}]\)}
                {\(\mach,(\domMap,\rts) \stepsto_C \domMap',(\retcon ~ \mach)::\rts\)}

\vspace*{-1ex}
\judgmentthree[ReturnFound]
              {\(\mach \stepsto \mach' \in \target\)}
              {\(\components = \{\component | \domMap ~ \component = \sealed{|\rts|}\}\)}
              {\(\domMap' = \domMap[\components \mapsto \unsealed]\)}
              {\(\mach,(\domMap,\target::\rts) \stepsto_C \domMap',\rts\)}

\vspace*{-3.5ex}
\judgment[Default]
         {}
         {\(\_,\context \stepsto_C \context\)}

When we build machine traces with this step function, we always know how deep in the
call stack we are and which components are sealed. We now proceed to state our properties.
The first is {\em stepwise stack integrity}, where ``stepwise'' means that the property is
quantified over each step of execution from any initial state. The criterion for integrity
is simple: sealed components do not change.

\definition A system enjoys {\em stepwise stack integrity} if, for any reachable state
\(\mpcstatename\) such that \(\mpcstatename \stepstocon \mpcstatename'\),
and any component \(\component\), if \(\pi_\context(\mpcstatename) = \domMap,\_\) and
\(\domMap ~ \component = \sealed{\_}\), then \(\pi_\mach(\mpcstatename') ~ \component =
\pi_\mach(\mpcstatename) ~ \component\).

Next we consider {\em stepwise stack confidentiality}. There are a number of preliminary
concepts we must discuss here -- confidentiality quantifies over {\em variants} of the
initial state of a call, and requires that the traces from them move in {\em lockstep}.

\definition Machine states \(\mach\) and \(\nach\) are {\em \(\components\)-variants},
written \(\mach \approx_\components \nach\), if, for
all \(\component \not \in \components\), \(\mach ~ \component = \nach ~ \component\).

\definition Machine state pairs \(\mach,\mach'\) and \(\nach,\nach'\) are said to
\emph{change together}, written \(\mach,\mach' \diamond \nach,\nach'\), if,
for all components \(\component\) such that
\(\mach ~ \component \not = \mach' ~ \component\) or
\(\nach ~ \component \not = \nach' ~ \component\),
\(\mach' ~ \component = \nach' ~ \component\).

\definition A pair of traces \(\machT\) and \(\nachT\) are {\em in lockstep}
if their adjacent pairs of states change together. We will represent this relation with
\(\machT \doteq \nachT\). \bcp{People may not be familiar with the notation}
\sna{It's a new notation, we use this enough that it seems worth picking an
  evocative symbol.}\bcp{Sorry -- I meant the notational convention of representing a
coinductive relation with inference rules in this way, not the symbol...}
We define it as a coinductive relation:

\begin{minipage}{.3\textwidth}
\judgment{}
         {\(\mach \doteq \nach\)}
\end{minipage}
\begin{minipage}{.6\textwidth}
\judgmenttwo{\(\mach, \head(\machT) \diamond \nach, \head(\nachT)\)}
            {\(\machT \doteq \nachT\)}
            {\(\mach \machT \doteq \nach \nachT\)}
\end{minipage}

\definition For any depth \(\depth\), define \(f_\depth ~ \mach ~ (\_, \rts)\) to be
\(|\rts| \geq \depth\). Then each reachable segment \(\MPCT \langle f_\depth \rangle\)
corresponds to a trace of a function from its call to its return.
Given such a segment \(S\), let \(\mach,\_,(\domMap,\_)\)  = \(\head(S)\) and let
\(\components\) be the set of components \(\component\) such that \(\domMap ~ \component = \outside\).
A system enjoys \emph{stepwise stack confidentiality} if, for all \(\MPCT\langle f_\depth\rangle\)
with associated \(\mach\) and \(\components\) as above, and
for all \(\nach\) such that \(\mach \approx_\components \nach\) and
 \(\nach \hookrightarrow \NPCT\),
 \(\MPCT\langle f_\depth\rangle \doteq \NPCT | \neg f_\depth \).\apt{I found the previous
   version hopelessly confusing, but I'm not at all sure that this one is right! SNA please check.}
 \sna{It's right, but I need to rethink how the M<f> notation works...}

\subsection{Stack Safety with Passing}

\newcommand{\mainpassc}{magenta}

\begin{figure}
  \centering
  \begin{subfigure}{.2\textwidth}
{\small
\begin{verbatim}
void main() {
  int x = 5;
  x = f(x);
  return;
}

void f(int a) {
  int y = 2*a;
  a = 0;
  return;
}
\end{verbatim}
}
  \end{subfigure}
  \begin{subfigure}{.6\textwidth}
{\small
\begin{verbatim}
1  main: add $1,%sp          ; allocate frame
2        mov $5,-1(%sp)      ; initalize x
3        add $1,%sp          ; allocate arg
4        mov -2(%sp),-1(%sp) ; set arg
5        call f              ; SHARE: %sp-1
6        sub $1,%sp          ; deallocate arg
7        mov %ra,-1(%sp)     ; store x
8        sub $1,%sp          ; deallocate frame
9        ret

10  f:   add $1,%sp          ; allocate frame
11       mov -3(%sp),-1(%sp) ; y = a
12       mul $2,-1(%sp)      ; y = a*2
13       mov $0,-3(%sp)      ; a = 0
14       sub $1,%sp          ; deallocate frame
15       ret
\end{verbatim}
}
  \end{subfigure}
\begin{subfigure}{.65\textwidth}
\begin{center}
\begin{tabular}{l r | l}
  {\tt 3} &
  \multicolumn{2}{c}{
    \memoryaddrs[(0)]{8em}
    \memory{4}{\unsealc}
    ~$\cdots$
    \MemoryLabel{-19em}{0.75em}{5}
%    \MemoryLabel{-14em}{0.75em}{0}
%    \MemoryLabel{-10em}{0.75em}{0}
%    \MemoryLabel{-7em}{0.75em}{0}
    \vspace{.5em}
  } \\
  {\tt 11} &
  \memoryaddrs[(1)]{21.5em}
  \memory{1}{\mainsealc}[Seal(0)]%
  \memory{1}{\mainpassc}[P(0,1)]%
  \memory{1}{\retptrc}[RetPtr]%
  \memory{1}{\unsealc}
  ~$\cdots$
  \MemoryLabel{-19em}{0.75em}{5}
  \MemoryLabel{-14em}{0.75em}{5}
%  \MemoryLabel{-6em}{0.75em}{0}
  &
  \memoryaddrs[(1)]{21.5em}
  \memory{1}{\mainsealc}
  \memory{1}{\mainpassc}
  \memory{1}{\retptrc}
  \memory{1}{\unsealc}
  ~$\cdots$
  \MemoryLabel{-19em}{0.75em}{\(v_0\)}
  \MemoryLabel{-14em}{0.75em}{5}
  \MemoryLabel{-6em}{0.75em}{\(v_1\)}
  \\
  {\tt 13} &
  \memoryaddrs[(1)]{21.5em}
  \memory{1}{\mainsealc}
  \memory{1}{\mainpassc}
  \memory{1}{\retptrc}
  \memory{1}{\unsealc}
  ~$\cdots$
  \MemoryLabel{-19em}{0.75em}{5}
  \MemoryLabel{-14em}{0.75em}{5}
  \MemoryLabel{-6em}{0.75em}{10}
  &
  \memoryaddrs[(1)]{21.5em}
  \memory{1}{\mainsealc}
  \memory{1}{\mainpassc}
  \memory{1}{\retptrc}
  \memory{1}{\unsealc}
  ~$\cdots$
  \MemoryLabel{-19em}{0.75em}{\(v_0\)}
  \MemoryLabel{-14em}{0.75em}{5}
  \MemoryLabel{-6em}{0.75em}{10}
  \\
  {\tt 14} &
  \memoryaddrs[(1)]{21.5em}
  \memory{1}{\mainsealc}
  \memory{1}{\mainpassc}
  \memory{1}{\retptrc}
  \memory{1}{\unsealc}
  ~$\cdots$
  \MemoryLabel{-19em}{0.75em}{5}
  \MemoryLabel{-14em}{0.75em}{0}
  \MemoryLabel{-6em}{0.75em}{10}
  &
  \memoryaddrs[(1)]{21.5em}
  \memory{1}{\mainsealc}
  \memory{1}{\mainpassc}
  \memory{1}{\retptrc}
  \memory{1}{\unsealc}
  ~$\cdots$
  \MemoryLabel{-19em}{0.75em}{\(v_0\)}
  \MemoryLabel{-14em}{0.75em}{0}
  \MemoryLabel{-6em}{0.75em}{10}
  \\
  \\
  {\tt 8} &
  \multicolumn{2}{c}{
    \memoryaddrs[(0)]{8em}
    \memory{4}{\unsealc}
    ~$\cdots$
    \MemoryLabel{-19em}{0.75em}{10}
    \MemoryLabel{-14em}{0.75em}{0}
    \MemoryLabel{-10em}{0.75em}{\#6}
    \MemoryLabel{-7em}{0.75em}{10}
    \vspace{.5em}
  } \\
\end{tabular}
\end{center}

\vspace{\abovedisplayskip}

\end{subfigure}

\caption{Sharing on the stack \ifaftersubmission\bcp{Not enough space above the bottom part of
    the figure (at least in the latex build that I'm looking at...)}\fi}
\label{fig:passingsimple}
\end{figure}

\begin{figure}
  \centering
  \begin{subfigure}{.2\textwidth}
{\small
\begin{verbatim}
int *stash = 0;

int main() {
  int x, y;
  f(x);
  x = 0;
  f(y);
  return x;
}

void f(int &a) {
  if(!stash)
    stash = &a;
  *stash = 5;
  return;
}
\end{verbatim}
}
  \end{subfigure}
  \begin{subfigure}{.49\textwidth}
{\small
\begin{verbatim}
   stash: .word 0
1  main:  add $2,%sp      ; allocate frame
2         mov %sp,%ra
3         add -2,%ra      ; ra = &x
4         call f          ; SHARE: %sp-2
5         mov $0,-2(%sp)  ; x = 0
6         mov %sp,%ra
7         add -1,%ra      ; ra = &y
8         call f          ; SHARE: %sp-1
9         mov -2(%sp),%ra ; set return value
10        sub $2,%sp      ; deallocate frame
11        ret

20 f:     cmp stash, $0   ; compare
21        breq #13        ; branch if equal
22        mov %ra,stash   ; stash = &a
23        jmp #15
24        mov $5, (stash) ; *stash = 5
25        ret
}
\end{verbatim}
}
  \end{subfigure}
  \begin{subfigure}{.29\textwidth}
    \begin{tabular}{l l}
      {\tt 20} &
      \memoryaddrs[(1)]{17em}
      \memory{1}{\mainpassc}[{\makebox[0pt]{P(0,1)}}]%
      \memory{1}{\mainsealc}[{\makebox[0pt]{Seal(0)}}]%
      \memory{1}{\retptrc}
%      \memory{1}{\unsealc}
      ~$\cdots$ \\
      {\tt 25} &
      \memoryaddrs[(1)]{17em}
      \memory{1}{\mainpassc}
      \memory{1}{\mainsealc}
      \memory{1}{\retptrc}
%      \memory{1}{\unsealc}
      ~$\cdots$
      \MemoryLabel{-18em}{0.75em}{5}
      \\
      {\tt 8} &
      \memoryaddrs[(0)]{12em}
      \memory{3}{\unsealc}
      ~$\cdots$
      \MemoryLabel{-18em}{0.75em}{0}
      \\
      {\tt 20} &
      \memoryaddrs[(1)]{17em}
      \memory{1}{\mainsealc}[{\makebox[0pt]{Seal(0)}}]%
      \memory{1}{\mainpassc}[{\makebox[0pt]{P(0-1)}}]%
      \memory{1}{\retptrc}
%      \memory{1}{\unsealc}
      ~$\cdots$
      \MemoryLabel{-18em}{0.75em}{0}
      \\
      {\tt 25} &
      \memoryaddrs[(1)]{17em}
      \memory{1}{\mainsealc}
      \memory{1}{\mainpassc}
      \memory{1}{\retptrc}
%      \memory{1}{\unsealc}
      ~$\cdots$
      \MemoryLabel{-18em}{0.75em}{\bf 5}
\end{tabular}
\end{subfigure}
\caption{A violation with pass-by-reference\apt{SP(n) depths looked wrong; SNA double-check that I fixed them right.}}
\label{fig:passing}
\end{figure}

To this point we have ignored function arguments, since our simplest model places them in
registers. This is an oversimplification, which we will now correct, aiming at two kinds
of sharing: passing by value on the stack, and passing by reference. First we will explore
some examples at the same level of detail as our earlier ones, and then move on to
extending our formalization.

Example~\ref{fig:passingsimple} shows a sample trace of
a program that passes a value on the stack, without an attack.
We extend our call annotation with an additional parameter identifying a memory location to be
passed, and our diagram displays it in magenta. Our notion of domains is extended as well,
to include \(\passed{\depth,\depth'}\), recording two depths. We note the
current depth of the active function next to the stack pointer in the diagram, for reference.
The argument {\tt a} has
its value copied from {\tt x}, and we note that {\tt f} can read it -- because it is marked
as passed between depths 0 and 1, it is not varied. Then at instruction 13 it is even
modified, and naturally this is permitted as well.

The same annotation mechanism can be used to pass
variables by reference, and here there are important stack safety concerns,
as seen in Figure~\ref{fig:passing}. In this example, written in C++ notation,
{\tt f} takes a single argument by reference.
{\tt main} calls {\tt f} passing a reference to {\tt x}, then clears {\tt x} and calls
{\tt f} again passing a reference to {\tt y}.
We would expect that the first call to {\tt f} might modify
{\tt x}, but not the second. Unfortunately {\tt f} has hidden away our reference and modifies
{\tt x} in the second call. (Incidentally, this is perfectly compilable C++!)

To avoid cluttering the picture at the assembly level, we revert to assuming that the
argument {\tt a} is passed in a register, namely {\tt \%ra}.
The question is how the local variables {\tt x} and {\tt y} themselves
should be labelled in order for the policy to detect an integrity violation.
At the initial call, the address of {\tt x} is marked as passed, and {\tt f} is free to write to it. Then upon
return, {\tt main}'s frame is unsealed. At the second call it is the address of {\tt y} that is marked as passed,
and when {\tt f} instead tries to write to {\tt x}, it violates integrity. For that matter
a future call with no arguments passed by reference would also violate integrity if it wrote
to either variable.

When passing by reference, it is possible to repeatedly pass the same reference further
and further down the stack. Note, however, that this sharing of the reference must be over
an unbroken chain of calls. So it suffices to track two depths for each passed
cell: the original owner, and the highest depth it has been passed to. The policy considers
it to be accessible anywhere between the two.

Our model provides only limited support for handling fully first-class pointers into
stack frames, e.g., pointers obtained by taking the address of a local variable.
Unlike the case for pass-by-reference, there is no easy way to model the intended
sharing behavior of such pointers in terms of stack frame activation patterns.
The best approximation we can make is to treat an address-taken variable in the
same way as a passed argument, with the additional property
that it remains accessible at \emph{all} stack depths above the caller.
This prevents most violations of temporal safety, although stale pointers can still be
used in certain corner cases.
More seriously, callees are permitted to access the location without having
been explicitly given a pointer to it.  Preventing the latter kind of abuse
is the provence of a fine-grained memory safety policy, e.g. based on capabilities
(implemented by tags or otherwise); we argue it is beyond the scope of stack safety
\emph{per se}.

Next we introduce variable passing. We will need to add extra information to our call
annotations to reflect that they may now pass variables on the stack, and that the locations
of those variables may be dynamic (as in the case of pass-by-reference).
A call annotation now includes a predicate relating machine states to the components
that should be passed -- when passing by value, this will typically give the memory cells
at the appropriate offset from the stack pointer, while passing by reference gives the
addresses of the appropriate variables wherever they happen to be.

\[\codemap \in \CODEMAPS ::= \addr \in \WORDS \rightharpoonup \callmap(P \subseteq \MACHS \times \COMPONENTS)\]

And we extend the set of domains to include passed components,
which are labeled with the range of depths of the caller that has passed them.

\[\stackDom \in \STKDOMS ::= \outside + \unsealed + \sealed{\depth} + \passed{\depth,\depth'}\]
\[\context \in \CONTEXTS ::= (\domMap \in \COMPONENTS \rightarrow \STKDOMS)
\times (\rts \in \listT (\target \subseteq \MACHS)) \]

We straightforwardly describe how the sealing convention and passing predicate combine to
update a domain map when pushing it, where \(\depth\) is an integer depth.

\[(\mathit{push} ~ \domMap ~ \sealcon ~ P ~ \mach ~ \depth) ~ \component =
\begin{cases}
  \sealed{\depth}              & \text{if } \domMap ~ \component = \unsealed \text{ and }
                                 \sealcon ~ \mach ~ \component \text{ and } \neg P ~ \mach ~ \component \\
  \passed{\depth,\depth+1}     & \text{if } \domMap ~ \component = \unsealed \text{ and }
                                 \sealcon ~ \mach ~ \component \text{ and } P ~ \mach ~ \component \\
  \passed{\depth',\depth+1}    & \text{if } \domMap ~ \component = \passed{\depth',\depth} \text{ and }
                                 \sealcon ~ \mach ~ \component \text{ and } P ~ \mach ~ \component,
                                 \text{ for any } \depth' \\
  \domMap~\component & \text{else} \\
\end{cases}\]

And to pop we just unseal the components at depth \(\depth\):

\[(\mathit{pop} ~ \domMap ~ \depth) ~ \component =
\begin{cases}
  \unsealed                 & \text{if } \domMap ~ \component = \sealed{\depth} \\
  \unsealed                 & \text{if } \domMap ~ \component = \passed{\depth-1,\depth} \\
  \passed{\depth',\depth-1} & \text{if } \domMap ~ \component = \passed{\depth',\depth}
                              \text{ for any } \depth' < \depth \\
  \domMap~\component        & \text{else} \\
\end{cases}\]

Now we can inductively define the context update function; these again apply in the order
presented.

\judgmenttwo[Call]
            {\(\codemap ~ (\mach ~ \PCname) = \callmap(P)\)}
            {\(\domMap' = \mathit{push} ~ \domMap ~ \sealcon ~ P ~ \mach ~ |\rts|\)}
            {\(\mach,(\domMap,\rts) \stepsto_C \domMap',(\retcon ~ \mach)::\rts\)}

\judgmenttwo[ReturnFound]
            {\(\mach \stepsto \mach' \in \target\)}
            {\(\domMap' = \mathit{pop} ~ \domMap ~ |\rts|\)}
            {\(\mach,(\domMap,\target::\rts) \stepsto_C \domMap',\rts\)}

\judgment[Default]
         {}
         {\(\_,\context \stepsto_C \context\)}

\definition Let \(\mpcstatename\) and \(\mpcstatename'\) be reachable states such that
\(\mpcstatename \stepstocon \mpcstatename'\), and \(\pi_\context(\mpcstatename) = \domMap,\rts\).
Let \(\components\) be the set of components \(\component\) where
\(\domMap ~ \component = \sealed{\_}\) or \(\passed{\depth,\depth'}\) where \(|\rts| < \depth\)
or \(|\rts| > \depth'\). A system enjoys {\em stepwise stack integrity with passing} if for all
such states, \(\pi_\mach(\mpcstatename) ~ \component = \pi_\mach(\mpcstatename') ~ \component\).

\definition Let \(\depth\) be any depth and \(f_\depth ~ \mach ~ (\_, \rts)\) hold for
\(|\rts| \geq \depth\). For any reachable segment \(\MPCT \langle f_\depth \rangle\),
let \(\head(\MPCT) = \mach,\pol,(\domMap,\rts)\). Let
\(\components\) be the set of components \(\component\) such that
\(\domMap ~ \component = \unsealed\) or \(\passed{\depth,\depth'}\) where
\(\depth \leq |\rts| \leq \depth'\) .
Then take any \(\nach\) such that \(\mach \approx_\components \nach\), and
\(\nach \hookrightarrow \NPCT | \neg f_\depth\). A system enjoys
{\em stepwise stack confidentiality with passing} if for all such \(\MPCT\) and \(\NPCT\),
\(\MPCT \doteq \NPCT\).

\subsection{Coroutine Safety}

Our final extension to the model is to add coroutines. We assume a very simple model with
a fixed number of coroutines, each with its own stack.
Stack identifiers are drawn from a set \(\STACKS\), and there is a static partial map
\(\stackof\) from addresses to stack ids. Note that we now explicitly reference the stack
pointer, using the \(\stackof\) of the stack pointer to determine the target of a yield.
We add a second annotation indicating yields.

\[\ann \in \ANNS ::= \callmap(P \subseteq \MACHS \times \COMPONENTS) + \yieldmap\]
\[\codemap \in \CODEMAPS ::= \addr \in \WORDS \rightharpoonup \ann \in \ANNS\]

The first major difference in this model is the domains. Stack domains are now nested inside
top-level domains, and associated with stack identifiers.
Our context now consists of a domain map, a map from stack identifiers to return target stacks,
an additional map of targets for yields, and a stack identifier of the active stack.
Our yield target map \(\yts\) is a map from stack identifiers to targets.
In a standard system an initial yield target map will map each stack identifier to the
set of all states with their stack pointer at the stack base. When yielding from state \(\mach\),
\(\ycon ~ \mach \subseteq \MACHS\) tells us what yielding back looks like.

\[\stackDom \in \STKDOMS ::= \unsealed + \sealed{\depth} + \passed{\depth-\depth'}\]
\[\topDom \in \TOPDOMS ::= \outside + \instack{\stackid}{\stackDom}\]
\[\context \in \CONTEXTS ::= (\domMap \in \COMPONENTS \rightarrow \STKDOMS) \times
(\rtm \in \STACKS \rightarrow \listT ~ \target) \times (\yts \in \STACKS \rightarrow
\target) \times (\stackid \in \STACKS)\]

In our context update rules, calls and returns work the same as in the sharing model,
with domain updates restricted to the top-level domain corresponding to the active stack,
and likewise using the return target stacks. We will omit these rules and focus on the new yield rule.

\judgmentthreebr[Yield]
                {\(\codemap ~ (\mach ~ \PCname) = \yieldmap\)}
                {\(\mach \stepsto \mach'\)}
                {\(\stackid' = \stackof ~ (\mach'~ \rsp)\)}
                {\(\stackid \not = \stackid'\)}
                {\(\yts ~ \stackid'\)}
                {\(\yts' = \yts[\stackid \mapsto \ycon ~ \mach]\)}
                {\(\mach,(\domMap,\rts,\yts,\stackid) \stepsto_C \domMap,\rts,\yts',\stackid'\)}

This calls for some explanation. First, we determine which coroutine we are yielding to by
checking which stack the stack pointer of the next state is pointing to. Then we can only yield
if the next state is a valid yield target for that stack, and we update the yield target for
the current stack, ensuring that when the system yields back we expect the current coroutine
to continue from where it left off. So, it is possible for a coroutine to claim to yield
but not actually change the active stack, with similar effects to when a callee fails
to return properly. Now we can define both stack safety and coroutine safety under this model.

\definition Let \(\mpcstatename\) and \(\mpcstatename'\) be reachable states such that
\(\mpcstatename \stepstocon \mpcstatename'\), and \(\pi_\context(\mpcstatename) = \domMap,\rtm,\yts,\stackid\).
Let \(\components\) be the set of components \(\component\) where
\(\domMap ~ \component = \instack{\stackid}{\sealed{\_}}\) or
\(\instack{\stackid}{\passed{\depth,\depth'}}\)
where \(|\rtm ~ \stackid| < \depth\) or \(|\rtm ~ \stackid| > \depth'\). A system enjoys
{\em stepwise stack integrity} in the coroutine setting if for all
such states, \(\pi_\mach(\mpcstatename) ~ \component = \pi_\mach(\mpcstatename') ~ \component\).

\definition Let \(\mpcstatename\) and \(\mpcstatename'\) be reachable states such that
\(\mpcstatename \stepstocon \mpcstatename'\), and \(\pi_\context(\mpcstatename) = \domMap,\rtm,\yts,\stackid\).
Let \(\components\) be the set of components \(\component\) where
\(\domMap ~ \component = \instack{\stackid'}{\_}\) where \(\stackid \not = \stackid'\).
A system enjoys {\em stepwise coroutine integrity} in the coroutine setting if for all
such states, \(\pi_\mach(\mpcstatename) ~ \component = \pi_\mach(\mpcstatename') ~ \component\).

\definition Let \(\stackid\) be a stack identifier, \(\depth\) be any depth, and
\(f_\depth ~ \mach ~ (\_, \rtm, \_, \stackid)\) hold for \(|\rtm ~ \stackid| \geq \depth\).
For any reachable segment \(\MPCT \langle f_\depth \rangle\),
let \(\head(\MPCT) = \mach,\pol,(\domMap,\rtm,\yts,\stackid)\). Let
\(\components\) be the set of components \(\component\) such that
\(\domMap ~ \component = \instack{\stackid}{\unsealed}\) or
\(\instack{\stackid}{\passed{\depth,\depth'}}\)
where \(\depth \leq |\rtm ~ \stackid| \leq \depth'\) .
Then take any \(\nach\) such that \(\mach \approx_\components \nach\), and
\(\nach \hookrightarrow \NPCT\langle f_\depth \rangle\). A system enjoys
{\em stepwise stack confidentiality} in the coroutine setting if for all
such \(\MPCT\) and \(\NPCT\), \(\MPCT \doteq \NPCT\).

\definition Let \(\stackid\) be a stack identifier, \(\depth\) be any depth, and
\(f_\depth ~ \mach ~ (\_, \rtm, \_, \stackid)\) hold for \(|\rtm ~ \stackid| \geq \depth\).
For any reachable segment \(\MPCT \langle f_\depth \rangle\),
let \(\head(\MPCT) = \mach,\pol,(\domMap,\rtm,\yts,\stackid)\). Let
\(\components\) be the set of components \(\component\) such that
\(\domMap ~ \component = \instack{\stackid}{\_}\).
Then take any \(\nach\) such that \(\mach \approx_\components \nach\), and
\(\nach \hookrightarrow \NPCT | \neg f_\depth\). A system enjoys
{\em stepwise coroutine confidentiality} in the coroutine setting if for all
such \(\MPCT\) and \(\NPCT\), \(\MPCT \doteq \NPCT\).

%\subsection{Well-bracketed Control Flow}
%\label{sec:wbcf}

%Both the integrity and confidentiality components of stack safety
%are reasonable even in programs that violate expected
%control flow. But even though stack safety prevents control-flow violations
%from being used to gain privileges, it is worth protecting control flow as well.
%Indeed, all of the enforcement mechanisms we discuss do preserve well-bracketed
%control flow.

%When we focus on subroutines, the essence of well-bracketed control flow
%is that a caller can expect that if its callee ever returns, it returns
%to the caller at the appropriate return target. This is tantamount to saying
%that we never ``skip'' a return, or perform a ``false'' one. To formalize this in
%the simple model:

%\definition A system enjoys {\em well-bracketed control flow} if for all
%reachable states \(\mpcstatename\) and \(\mpcstatename'\) such that
%\(\mpcstatename \stepstocon \mpcstatename'\), if
%\(\pi_c(\mpcstatename) = (\_, \_::\rts)\) and \(\pi_m(\mpcstatename)\)
%is annotated as a return, then \(\pi_c(\mpcstatename) = (\_, \rts)\).

%Extending this idea to coroutines, we use the same notion that there are
%no ``false'' yields.

%\definition A system enjoys {\em well-structured control flow} if for all
%reachable states \(\mpcstatename\) and \(\mpcstatename'\) such that
%\(\mpcstatename \stepstocon \mpcstatename'\), if
%\(\pi_c(\mpcstatename) = (\_,\_,\_,\stackid)\) and \(\pi_m(\mpcstatename)\)
%is annotated as a return, then \(\pi_c(\mpcstatename) = (\_,\_,\_,\stackid')\)
%where \(\stackid \not = \stackid'\).

\section{Enforcement}
\label{sec:enforcement}

In this section we examine how an existing enforcement mechanism,
Depth Isolation from \citet{DBLP:conf/sp/RoesslerD18}, can implement
the formal stack-safety property described in the previous section: it in
fact enforces a stronger, inductive version of that property which strengthens the
local state encapsulation component by checking for integrity and
confidentiality violations at every step of the callee's execution trace. \bcp{Examples?}

%\paragraph*{Micro-Policies}
%
The enforcement policy of \citeauthor{DBLP:conf/sp/RoesslerD18} relies
on a programmable, tag-based reference monitor that runs alongside the
program.  To control this monitor, they use a programming model that
allows fine-grained manipulation of metadata tags to encode so-called
\emph{micro-policies}~\citep{pump_oakland2015}.
In such a system, all values in
memory addresses and registers (including the $\PCname$)
are enriched with an abstract metadata tag, which can represent
arbitrary information about the value. A micro-policy is defined as a set of tags and a
collection of software-defined rules, indexed by machine opcode.
At each step of the machine, the relevant rule is applied to the tags on
the instruction's inputs ($\PCname$, registers, memory) and on the instruction itself, and produces
one of two outcomes: either the instruction is allowed to execute (and generates
tags for the result of the operation and the new $\PCname$), or the machine
fail-stops with a policy violation.
\citeauthor{pump_oakland2015}\bcp{and PIPE, if there are any references to
  cite} have shown that a wide range of micro-policies can
be defined using this scheme.

Efficient execution of these micro-policies relies on hardware implementations,
such as the PUMP architecture~\citep{pump:asplos2015}.  Tags are represented
as word-size bit vectors, stored separately from the regular memory and registers.
The hardware incorporates a rule cache
to allow quick retrieval of rule outputs for mapped inputs. If the cache misses,
the hardware traps to a software handler (running in a privileged context or
on a co-processor) to compute the rule result. To obtain adequate performance,
it is important to design micro-policies so that they hit in the cache as
much as possible. Thus, practical policies maintain a small working set of
distinct tags.
%% %
%% \rb{Maybe go into more detail, examples later\ldots or try to segue into stack
%% policies}

\subsection{A Conservative Tag Policy}
\label{sec:conservative}

\bcp{Is this the only subsection in this section?}

The first micro-policy we present to enforce the definition of stack
safety developed in \cref{sec:lse-and-wbcf} is a small variation
on the Depth Isolation policy presented by
\citet{DBLP:conf/sp/RoesslerD18}. Their policy tags the stack memory
with ownership information associated to each stack frame (and to each
separate object inside that frame, a more finely grained access
control that we do not need to consider here), and tags registers containing
stack pointers with access permission information. The policy also
uses tags on instructions to identify the code
sequences that have permission to manage the stack, say during calls and returns.
%% \rb{Conceivably on alloc
%%   operations, etc., which we have not yet mentioned.}

Our micro-policy works as follows. (We assume for ease of exposition that no
arguments are passed on the stack.)
%% \rb{Come back to this later?}.
The micro-policy maintains tags on values in memory of the form $\tagStackDepth{n}$,
indicating locations that belong to the stack frame at activation depth $n$,
or $\tagNoDepth$; it tags the $\PCname$ with tags of the form $\tagPCDepth{n}$.
During normal execution, the micro-policy rules only permit load and store operations
when the target memory is tagged with the same depth as the current {\PCname} tag.
Initially, the entire stack is tagged $\tagNoDepth$,
%\rb{actually, the ``initial frame'' should be tagged with its proper depth}
and the {\PCname} has tag $\tagPCDepth{0}$, where $n$ is the current activation depth.
These tags are altered
at exactly those points in
the program where the contour of the stack-safety property changes:

\begin{itemize}

\item From caller to callee, when the machine executes an instruction marked as a call in
  the call map.
  At this point, the current
  $\tagPCDepth{n}$ tag is incremented to $\tagPCDepth{(n + 1)}$, and the function
  entry sequence initializes  all locations of the new function frame with
  tag $\tagStackDepth{(n + 1)}$.

\item From callee back to caller, when execution is about to reach the return point
  $\ret{c}$
  corresponding to the call at $c$.  The exit sequence retags all locations in the function
  frame with $\tagNoDepth$. At the actual return instruction, the
  $\PCname$ tag is decremented.
  \rb{Problem: this is currently the first mention of $\ret{\cdot}$}

\end{itemize}

This discipline suffices to enforce local stack encapsulation.
To implement it,  we define ``blessed'' instruction sequences
intended to appear at the entry and exit of each function,
which manipulate tags as just described in addition to performing the
usual calling convention tasks of saving/restoring the return address to/from
the stack and adjusting the stack pointer. With the aid of an additional tag on
the $\rsp$ register, these sequences also serve to enforce
well-bracketed control flow.
The micro-policy guarantees atomic execution of these sequences
using a combination of tags on the instructions
and an additional tag on the $\PCname$; we omit the details here.

There remains the question of how to ensure that the sequences are
invoked at the right places. For a program to enjoy the stack-safety property
(with respect to a particular call map), all we need require is that
the entry sequence be initiated at any instruction marked as a call
in the call map. This is achieved by giving these instructions (another) special tag,
and it is easy to check statically that this has been done correctly.
If the code fails to initiate an exit
sequence at a point where the stack-safety property expects a return,
the micro-policy will incorrectly behave as if execution
is continuing in the callee, but since the callee never has stronger access
rights than the caller, this is harmless.

%% APT: removed because the tag part is too mysterious and the instruction part
%% is standard.
%% This is the entry
%% sequence:
%% %
%% \setcounter{pcctr}{1}
%% \[
%%   \begin{array}{l|l|l|l}
%%      & \mathit{Instruction} & \mathit{Tags} & \\
%%     \hline
%%     \row{\sw ~ \rsp ~ \tta ~ 1}{[\rsp+1] \leftarrow \rra}{\tagHa, \tagInstr}
%%         {Store return address in stack}
%%     \row{\addi ~ \rsp ~ \rsp ~ 2}{\rsp \leftarrow \rsp + 2}{\tagHb, \tagInstr}
%%         {Increment stack pointer by frame size}
%%   \end{array}
%% \]
%% %
%% And the exit sequence:
%% %
%% \setcounter{pcctr}{1}
%% \[
%%   \begin{array}{l|l|l|l}
%%      & \mathit{Instruction} & \mathit{Tags} & \\
%%     \hline
%%     \row{\lw ~ \rra ~ \rsp ~ \negate 1}{\rra \leftarrow [\rsp - 1]}{\tagRa, \tagInstr}
%%         {Load return address}
%%     \row{\addi ~ \rsp ~ \rsp ~ \negate 2}{\rsp \leftarrow \rsp - 2}{\tagRb, \tagInstr}
%%         {Decrement stack pointer by frame size}
%%     \row{\jalr ~ \rra ~ \rra ~ 0\apt{wrong}}{\PCname \leftarrow \rra}{\tagRc, \tagInstr}
%%         {Return to caller ($\jalr$)}
%%   \end{array}
%% \]

%% \rb{Positioning of the above sequences, relation to running example. The tagging
%%   of stack frames would be considered part of these sequences.}

%% These sequences also enforce the $\SP$ discipline. When stack frames
%% are of fixed size, it suffices to increment $\SP$ and tag it with a
%% dedicated tag in the header sequence, which is later checked after
%% decrementing $\SP$ during the exit sequence to authorize the
%% return. The well-formedness of the blessed sequences can be checked
%% statically.

%% Only programs whose sequences are well-formed are
%% protected by the micro-policy, although programs are still protected
%% (by fail-stopping) even if the intended sequences are missing. By
%% tracking the execution of the blessed sequences in the tag state of
%% the {\PCname} register, the micro-policy ensures that the entry
%% sequences of calls declared in the call map are executed correctly.

Note that in this policy both the entry and exit sequences
must write to each element of
the frame, which can be quite expensive, especially for programs that
allocate large but sparsely populated frames; we return to this point
in \cref{sec:lazy}.
%
For example, the program in \cref{fig:passingsimple} manages the stack
in this way. Here, the call pseudo-instruction to {\tt f} in line 5
hides a blessed sequence of instructions that allocate space to save
its return address to {\tt main} and seal the protected part of the
stack before {\tt f}'s body starts executing at 10. Conversely, the
return instruction in line 15 restores and jumps to the return address
and deallocates its frame space, and unseals the corresponding parts
of the stack. The function itself allocates and deallocates its own
private frame space in instructions 10 and 14. Additionally, unlike in
the simplified presentation of the policy, the caller shares the
arguments passed to the callee through the stack, and for this creates
and initializes (3--4) and destructs (6) the argument part of the
callee's stack frame.
%
%\rb{Note that we have also considered and implemented more flexible schemes
%  based on allowing operations above the current activation depth. How do we put
%  these together?}
%\rb{Do we want to discuss variations involving frame pointers or more implicit
%  return addresses, argument passing on the stack, alloc/dealloc, etc.?}
%\rb{Where to talk about the role of the compiler?}

%% Because the access rights of callers subsume those of callees, it is essential
%% that the micro-policy change tags when the property thinks that a call occurs,
%% but it is safe for the micro-policy to fail to reset to the caller's tag when the
%% property thinks a return occurs. (Dually, it is safe for the micro-policy to
%% change tags even without executing a call point, but not for it to reset tags to
%% the caller when the property thinks execution is still in the caller).

%% Because the property does not identify return sites, we have no way to require
%% that ``intended returns'' are tagged correspondingly. Some unintended results
%% might also be so tagged, but as long as the jump has the effect of a legal
%% return, it is OK.

\section{Randomized Testing}
\label{sec:testing}

\section{Stack Safety for Lazy Enforcement}
\label{sec:lazy}

The conservative policy described in the previous section, while
testable and enforceable, is rather slow to be of practical use. What
we want, instead, is a somewhat more permissive policy that can also
be efficiently implemented. Most of the performance overhead incurred
stems from the need to set stack activation tags as a frame is created
and reset them as it is destructed, as
\citet{DBLP:conf/sp/RoesslerD18} confirm in their evaluation. To
mitigate those costs, they propose a number of optimizations.

First, they eliminate the initialization pass. In this {\em Lazy Tagging}
system, the stack is initially tagged as ``clear.'' A function is permitted
to overwrite clear memory, and when it does so its depth tag overwrites
the clear tag, claiming the memory; the remainder of the policy functions
as {\em Depth Isolation}. On return, the callee must clear its frame, which
is still a heavy cost.

To eliminate this cost, \citet{DBLP:conf/sp/RoesslerD18} introduce
{\em Lazy Tagging and Lazy Clearing}. Under this optimizations,
the policy does not initialize stack frames on entry or clear them on exit,
and it permits all writes to the stack, even when the $\PCname$ tag
does not match the memory tag. Instead it propagates the \(\PCname\) tag
to written memory. Then reads from the stack do still require $\PCname$
tag and memory tag to match. Thus, even if a callee illicitly writes to
a private location in its caller, the caller will eventually detect this
if it ever tries to read from that location.
%
These \emph{lazy policies} admit more efficient implementations, but
they deliberately allow violations of stack integrity temporarily,
with the checks deferred until the point a violation truly becomes
harmful. The natural question then is, how do we characterize the
protections provided by these policies once stack safety is broken?
How does a harmful violation that must be caught later differ from a
harmless one?

\paragraph*{Deferred Stepwise Integrity}

We first notice that, in principle, these properties appear to enforce
confidentiality outright. So we wish in particular to find an integrity
property. The intuition is that sealed memory can change, but it is
then tainted, and its original owner cannot actually read it. This
leads us to a hybrid of integrity and confidentiality.

For simplicity, our definitions return to the simple stack without sharing,
but all properties in this section can be extended to more complex settings
very naturally.

\definition Let \(\depth\) be any depth and \(f_\depth ~ \mach ~ (\_, \rts)\) hold for
\(|\rts| \geq \depth\). For any reachable segment \(\MPCT \langle f_\depth \rangle\),
let \(\head(\MPCT) = \mach,\pol,(\domMap,\rts)\) and
\(\components\) be the set of components \(\component\) such that \(\domMap ~ \component = \sealed{\_}\).

Then, supposing that \(\last(\MPCT) = \mach',\pol',\context'\) -- meaning that the
call does return -- let \(\components'\) be the set of components that differ
between \(\mach\) and \(\mach'\). Take any \(\nach\) such that
\(\mach' \approx_{\components'} \nach\), \(\mach \hookrightarrow \MPCT\), and
\(\nach \hookrightarrow \NPCT\). A system enjoys
{\em deferred stepwise integrity} if for all such \(\MPCT\) and \(\NPCT\),
\(\MPCT \doteq \NPCT\).

In short, if sealed components are modified during the call, the caller may not
access their values. ``No reads after dangerous writes.''

\subsection{Observability}

We argue that while deferred stepwise integrity meaningfully captures the intuition
behind lazy tagging, it stops short of the kind of extensional characterization
that would be viable as a weakest reasonable definition. Consider a hypothetical
taint-tracking policy, that allows dangerous writes to be copied as long as the
taint remains hidden from outside observers. In order to characterize
such a policy, we would need a model of observation.
Similarly, for such a taint-based policy, lockstep confidentiality would be too
strong, as we might permit secrets to move about as long as they were not
leaked to the outside world.

So we extend our model with a notion of observations. We abstract over an
observation type \(\obs \in \OBSS\), containing at minimum the silent observation
\(\tau\). Our machine step function now takes a state and returns a state and an observation,
which is carried over into the policy and context step functions.
%
\[\mach \stepsto[\obs] \mach' \in \MACHS \rightarrow \MACHS \times \OBSS \]
%

\paragraph*{Observations-Of}

The ``observations-of'' operator, written \(\obsof\), takes a machine trace and
creates a trace of observations corresponding to the observations of each step in
the trace. It uses a coinductive helper \(\mathit{obsforward}\) that relates an
initial observation, a machine-trace, and an observation trace:

\judgmentthree{\(\mpcstatename \stepstocon[\obs'] \mpcstatename'\)}
              {\(\mpcstatename' \hookrightarrow \MPCT\)}
              {\(\mathit{obsforward} ~ \obs' ~ \MPCT = \obsT\)}
              {\(\mathit{obsforward} ~ \obs ~ \mpcstatename \MPCT = \obs\obsT\)}%
\judgment{}
         {\(\mathit{obsforward} ~ \obs ~ \mpcstatename = \obs\)}

Then to take the observations from any particular trace, we start with a \(\tau\) observation:

\judgment{\(\mathit{obsforward} ~ \tau ~ \MPCT = \obsT\)}
         {\(\obsof(\MPCT) = \obsT\)}

\paragraph*{Observational Similarity}

We say that a trace of observations $\obsT_1$ is a prefix of $\obsT_2$
as far as an external observer is concerned, written \(\obsT_1 \lesssim
\obsT_2\), if the sequence of
non-silent observations of $\obsT_1$ is a prefix of those of
$\obsT_2$; that is, we operate up to deletion of \(\tau\) observations,
coinductively:

\begin{minipage}{.3\textwidth}
  \judgment{}{\(\obsT \lesssim \obsT\)}
\end{minipage}
\begin{minipage}{.3\textwidth}
  \judgment{}{\(\tau \lesssim \obsT\)}
\end{minipage}
\begin{minipage}{.3\textwidth}
  \judgment{}{\(w \lesssim w\obsT\)}
\end{minipage}

\begin{minipage}{.3\textwidth}
  \judgment{\(\obsT_1 \lesssim \obsT_2\)}
           {\(\tau \obsT_1 \lesssim \obsT_2\)}
\end{minipage}
\begin{minipage}{.3\textwidth}
  \judgment{\(\obsT_1 \lesssim \obsT_2\)}
           {\(\obsT_1 \lesssim \tau \obsT_2\)}
\end{minipage}
\begin{minipage}{.3\textwidth}
  \judgment{\(\obsT_1 \lesssim \obsT_2\)}
           {\(w\obsT_1 \lesssim w\obsT_2\)}
\end{minipage}

\smallskip
We then define similarity of observation traces as traces prefixing each other:
\[\obsT_1 \simeq \obsT_2 \triangleq \obsT_1 \lesssim \obsT_2 \land \obsT_2 \lesssim \obsT_1\]
%
%\leo{I still don't know how to do this transition. Maybe contrast
%with the ``perhaps surprisingly'' later?} Note that
%similarity of observation traces could alternatively be defined coinductively,
%just like the prefix relation, but by dropping the asymmetric rules that allow
%for a finite trace ($\tau \lesssim \obsT$ or $w \lesssim w\obsT$).\apt{Is this immportant?}\leo{I agree. Probably not}

Note that an infinite silent trace is a
prefix of (and similar to) any other trace. While this might seem
surprising at first, it makes sense in a timing-insensitive context:
an external observer looking at two machine runs cannot (computably)
distinguish between a machine that steps forever and a machine that
steps for a long time before producing some output.

\subsection{Observational Stack Safety}

We now can sketch observational versions of both integrity and confidentiality.
Integrity is already straightforward, simply weakening the condition that must hold
after the return.

\definition Let \(\depth\) be any depth and \(f_\depth ~ \mach ~ (\_, \rts)\) hold for
\(|\rts| \geq \depth\). For any reachable segment \(\MPCT \langle f_\depth \rangle\),
let \(\head(\MPCT) = \mach,\pol,(\domMap,\rts)\) and
\(\components\) be the set of components \(\component\) such that \(\domMap ~ \component = \sealed{\_}\).

Then, again supposing that \(\last(\MPCT) = \mach',\pol',\context'\),
let \(\components'\) be the set of components that differ
between \(\mach\) and \(\mach'\). Take any \(\nach\) such that
\(\mach' \approx_{\components'} \nach\), \(\mach \hookrightarrow \MPCT\), and
\(\nach \hookrightarrow \NPCT\). A system enjoys
{\em observational integrity} if for all such \(\MPCT\) and \(\NPCT\),
\(\obsof(\MPCT) \simeq \obsof(\NPCT)\).

We amend our intuitive statement to: ``No visible reads after dangerous writes.''

Confidentiality is more sophisticated. We must capture the intuition that
secrets do not escape during a call {\em and} are not leaked during a return.
This require us to be able to quantify over components that preserve our \(\diamond\)
relation from earlier.

\definition Let \(\components\) be a set of components and \(\mach,\mach'\) and \(\nach,\nach'\)
be pairs of states. The {\em unsafe set} of \(\components\), written
\(\bar{\Diamond}(\components,\mach,\mach',\nach,\nach')\), is the set of all components
\(\component \in \components\) such that if
\(\mach ~ \component \not = \mach' ~ \component\) or
\(\nach ~ \component \not = \nach' ~ \component\), then
\(\mach' ~ \component = \nach' ~ \component\).

\definition Let \(\depth\) be any depth and \(f_\depth ~ \mach ~ (\_, \rts)\) hold for
\(|\rts| \geq \depth\). For any reachable segment \(\MPCT \langle f_\depth \rangle\),
let \(\head(\MPCT) = \mach,\pol,(\domMap,\rts)\) and
\(\components\) be the set of components \(\component\) such that \(\domMap ~ \component \not = \outside\).
Let \(\nach\) be a variant such that \(\mach \approx_\components \nach\) and
\(\nach \hookrightarrow \NPCT | \neg f_\depth \).

A system enjoys {\em observational confidentiality} if for all such
\(\MPCT\) and \(\NPCT\), the following conditions all hold:

\begin{itemize}
\item They are observationally equivalent, \(\obsof(\MPCT) = \obsof(\NPCT)\),
  meaning that no secrets leak during the call
\item \(f_\depth ~ \last(\MPCT) \leftrightarrow f_\depth ~ \last(\NPCT)\) --
  if one call returns, so does the other
\item Assuming both calls return, let \(\last(\MPCT) = \mach',\pol',\context'\)
  and \(\pi_\mach(\last(\NPCT)) = \nach'\).
  Let \(\components' = \bar{\Diamond}(\components,\mach,\mach',\nach,\nach'\).
  And let \(\nach''\) be a state such that \(\mach' \approx_\components \nach''\).
  Then if \(\mach',\pol',\context' \hookrightarrow \MPCT'\) and
  \(\nach',\pol',\context' \hookrightarrow \NPCT'\), it must be
  that \(\obsof(\MPCT') \simeq \obsof(\NPCT')\)
\end{itemize}

This last condition is the most interesting: we identify the components that
contain leaked information on either side of the first pair of variant traces.
This becomes the basis for the second variation, because all of these values
must be protected from leaking for the remainder of the program.

\sna{More to write here. The gist is: this is the extensional property.
  The one that we would tentatively propose as a real characterization of stack
  safety.}

\subsection{Connection to Policies}

    Observable properties allow us to defer enforcement until a property
    violation would become visible, as in lazy policies. But by formalizing
    the connection to the eager policy, we can identify when lazy policies
    miss violations that can become visible. The optimized lazy tag policy from
    \citet{DBLP:conf/sp/RoesslerD18} tags each stack slot with the call depth
    at which it was written, and enforces that it must be read from the
    same depth. However, this allows violations of observable stack safety!
    Say we have functions {\tt foo}, {\tt bar} and {\tt baz}, where {\tt foo}
    calls {\tt bar}, and then after {\tt bar} returns {\tt foo} calls
    {\tt baz}. Since {\tt bar} and {\tt baz} are at the same stack depth, if
    {\tt bar} writes to {\tt foo}'s frame, {\tt baz} can then read the tainted
    write and use it to create an observable difference that violates both
    observable integrity and confidentiality without violating the policy.

    The lazy policy can be repaired if, instead of tagging each function
    activation with its depth in the stack, we generate a fresh activation
    identifier on each call, which prevents a program from exploiting stale
    tags from previous activations. A related mechanism was explored in the
    Static Authorities policy of \citet{DBLP:conf/sp/RoesslerD18}, which
    associates a unique activation identifier to each function, which is
    however shared by all activations of the same function in order to obtain
    good cacheability of tag rules. Static Authorities comes closer to
    observable stack safety, though the example above would still exist
    if {\tt bar} and {\tt baz} were merged into a single function that tracked
    whether it had been called previously and changed its behavior accordingly.
    \ifaftersubmission\apt{More blatently, static authorities doesn't work for recursive calls.}\fi

\section{Related Work}
\label{sec:relwork}

%% CHERI \rb{cite}

\paragraph{Formal Stack Safety on Capability Machines}
%
To our knowledge, the only line of work to date that has attempted
a positive and formal characterization of
stack safety is \citet{Skorstengaard+19b}, who introduce a calling
convention that uses local capabilities to preserve local state
encapsulation and well-bracketed control flow, using a logical
relation to reason about the stack safety of concrete
programs. Although the required hardware support is readily available
in capability machines like CHERI, this technique incurs significant
costs, because it requires the entire unused part of the stack to be
cleared whenever a security boundary is crossed. Their logical
relation captures capability safety without ``externally observable
side-effects (like console output or memory access traces)''
and can be used to reason about individual programs. \bcp{Not very clear:}In the
discussion, ``while [the authors] claim that [their] calling
convention enforces control-flow correctness, [they] do not prove a
general theorem that shows this, because it is not clear what such a
theorem should look like,'' noting that the correctness property
enjoyed by their technique ``is not made very explicit.''

StkTokens \citep{Skorstengaard+19} continues this line of work. Like
the earlier paper, it aims to protect the stack by enforcing local state
encapsulation and well-bracketed control flow. It does so by defining a new
calling convention that makes use of linear capabilities for stack and return
pointers. The convention operates on a single shared stack and requires that
protected components avoid compromising their own security by following certain
simple rules---like not leaking their private capability seals. More
precisely but still informally, local state encapsulation is defined as
restricting accesses to the range of memory allocated to the current stack
frame, and well-bracketed control flow as only allowing returns from the topmost
frame to the immediately adjacent frame below. Formally, it improves on
\citet{Skorstengaard+19b} by building those properties into the semantics of a
capability machine with a built-in call stack and call and return instructions,
which is proven fully abstract with respect to a more concrete capability
machine that replaces those pseudo-instructions with their calling convention.
Their proof of full abstraction uses a standard notion of components, which
import and export functions through their interfaces; their model of
observations is limited to cotermination.

We conjecture that the linear capability machines introduced by
\citep{Skorstengaard+19}, in combination with the StkTokens calling convention,
can be modeled in our framework and satisfy our definition of stack safety in
\cref{sec:lse}. Further, we expect them to satisfy the even stronger
property developed in \cref{sec:enforcement}. Proving (or testing) these conjectures would
involve exposing their formal definitions of local state encapsulation and
well-bracketed control flow, which are built-in as part of the
semantics of the capability machines and not given explicitly.
The main practical limitation of StkTokens
is its reliance on linear capabilities, as it is unclear now they could be added
to practical capability machines, especially in terms of efficiency (previous
work on micro-policies \citep{yannis-report} has shown how to use that framework
to implement linear return capabilities.\bcp{But IIUC the Dover realization
  of micro-policies can't do this...?})\rb{Correct}
%
In recent work, \citet{Georges+21} address these efficiency concerns
with the introduction of a new type of uninitialized capability that
avoids the need to clear the stack while being more amenable to
practical hardware implementation, although like
\citet{Skorstengaard+19b} it lacks a general stack safety theorem,
instead using a similar logical relation to reason about the safety of
individual programs.

\paragraph{Protecting the Stack with Micro-Policies}
%
\citet{DBLP:conf/sp/RoesslerD18} consider a standard attacker model where all
attacks against stack data are in scope (but not side channels or hardware
attacks), and study the protection of stack data through three families of
micro-policies that tag stack objects with a pair of frame and object identifiers
used to validate accesses to the stack: Return Address Protection (which
prevents an adversary from overwriting designated return addresses), Static
Authorities (which only allows the code of a function to access the stack frames
of its own dynamic instances), and Depth Isolation (described in
\cref{sec:enforcement}). All these policies exploit various kinds of spatial and
temporal locality of stack memory and local call graphs, as well as information
generated by the compilation toolchain, to strike various balances between
precision and cacheability, evaluated through benchmarks that demonstrate
limited performance overhead. In addition to the baseline ``eager'' policies,
they propose a number of lazy optimizations, notably Lazy Tagging and Lazy
Clearing, discussed in \cref{sec:lazy}, with associated improvements in
performance.

\paragraph{Heap Safety as a Security Property}

Heap safety, like stack safety, may be framed as a security property
in the form of
noninterference~\citep{DBLP:conf/post/AmorimHP18}\ifaftersubmission\bcp{Should
be ``Azedevo et al.''}\fi. Just
as~\citeauthor{DBLP:conf/post/AmorimHP18} give a rigorous
characterization of the meaning of (heap) memory safety, in this paper
we aim to do the same for the stack.\bcp{Let's describe their work in a bit
  more detail.}
%
Their model describing safety of shared stack-allocated
objects\bcp{??} extends naturally to heap safety, complicated by the fact that
these objects are not deallocated by returns\bcp{??}, and therefore a
function's privilege may increase or decrease after its entry
point. For instance, if a callee allocates a heap object and returns
the pointer to its caller, that object's addresses become accessible
in contradiction to a contour computed at the caller's entry. So such
an extension must enforce trace confidentiality and integrity
properties separately on continuous segments of each call, with
contours computed at each crossing between caller and callee.  \bcp{Didn't
  understand last sentence.}

\section{Future Work}
  \label{sec:future}

  We have presented a pure notion of stack safety that omits many complicating
  factors common to real systems. In particular, our model as presented exploits
  the simplifying assumption that privilege never increases during a function
  call, and that therefore a contour computed once on the entry to a function
  describes its privilege throughout. Our future plans are to extend the model
  with common language features, some of which violate this assumption. Here we
  describe the basic principles by which our model can be extended to some of
  these cases. We also plan to develop a testing framework for quickly
  checking the validity of different enforcement mechanisms.

  \paragraph*{Capability machines}
  %
  %Other policies are implemented in hardware or built on hardware mechanisms.
  Capability machines such as CHERI \citep{Woodruff+14} extend conventional
  architectures to support efficient and fine-grained control over memory
  accesses, which can be used to implement security policies, including
  the stack protection policies of
  \citet{Skorstengaard+19b} and (with an extension to support linear capabilities) \citet{Skorstengaard+19}.
  %\bcp{I thought these policies used
  %  linear capabilities?}\leo{Only the second one AFAIK. What would you
  %  change to characterize it more accurately?}\rb{StkPointers uses linear
  %  capabilities, its predecessor users local capabilities (and unlike
  %  StkPointers could be implemented in CHERI)}.
  Capabilities package up pointers
  (ordinary data) with base and bounds information (policy data),
  and they require rewriting source code (e.g., to use capabilities in place
  of ordinary pointers), so separating ordinary machine state from policy
  state is delicate; we conjecture it can be done with some effort.

\BCP{This section seems incomplete?}

  %% Mapping the underlying protection
  %% mechanisms to policy states and step function would be a delicate operation, as
  %% those are tightly coupled to the hardware design. However, separating the
  %% enforcement aspects of a concrete policy (such as StkTokens, where policy state
  %% would encompass details like the abstract call stack, the capability sealing
  %% scheme, etc. \rb{is this digging too deep? or on the contrary, say more?}) from
  %% the machine would be relatively straightforward.  \bcp{Not sure that any of
  %%   this is comprehensible... :-(}

  %\paragraph*{Software-only policies}
  %
  %Some policies are implemented in software, by modifying code in the
  %compiler, for example to perform bounds checking~\citep{NagarakatteZMZ09} or
  %to insert stack canaries~\citep{Cowan+98}. Such approaches can, in principle,
  %be evaluated in terms of our formal notion of stack safety with a trivial null
  %policy. But any state maintained by the enforcement mechanism must be in normal
  %memory and subject to the same confidentiality and integrity concerns (to be
  %varied, rolled back, etc.), a heavy constraint. Alternately it can be stripped
  %out as policy state provided that it can be proven to protect itself.

%  \paragraph{Stack-derived Pointers}

%    Common programming idioms involve a caller passing a pointer to
%    its local data into a callee. Using standard notions of pointer
%    provenance~\citep{provenance}, we can extend our model to
%    distinguish safe use of this idiom from true stack safety
%    violations. We require additional annotation to identify which
%    addresses within a function's stack frame correspond to distinct
%    objects. Then a valid pointer to an object is one derived through
%    legal arithmetic from its base address, and at each call point, an
%    object's addresses are marked low confidentiality and integrity if
%    a valid pointer exists in a register or in the transitive closure
%    of the callee's accessible memory. All pointers to an object cease
%    to be valid when it is deallocated along with its stack frame.

%  \paragraph{Non-stack control flow}

%    Control structures beyond well-bracketed calls and returns require
%    modifications to the model. Tail calls, for instance, reuse the caller's
%    stack frame for its callee, and every nested tail call returns
%    simultaneously to the top non-tail-calling function. Under normal stack
%    integrity, the first function in a chain of tail calls returns and violates
%    integrity, while subsequent tail calls never actually return at all.
%    Instead we need a variant that treats the entry of a tail call as the
%    return of its caller.

%    More complex is the addition of a coroutine model, in which multiple stacks
%    respect stack safety internally, but might also yield to one another.
%    Here we must distinguish three levels of integrity: accessible, in the same
%    stack, and in another stack. Stack integrity requires that same-stack data
%    be unchanged when a function returns, but other-stack data may change. A
%    similar {\em yield integrity} property requires that, from a yield out of
%    a coroutine to the next yield back into it, all accessible and same-stack
%    data are unchanged. Confidentiality properties are split similarly.

%    \paragraph{Random Testing}

%    Part of the motivation for the strong lockstep stack-safety property
%    is that it uncovers errors quickly, and therefore should be quicker
%    to test, just like the smarter testing properties
%    of~\citet{TestingNI:ICFP}. We have already developed a testing
%    framework for debugging our policy enforcement for this property, and
%    it would be interesting to explore whether a testing-amenable variant
%    of the lazy property also exists.

% \rb{Notes from meeting:
%   \begin{itemize}
%   \item Explain how the theory applies to production compilers like GCC and
%     LLVM, which aspects are directly applicable, what are the effects of
%     alternative calling conventions. Other less charted waters: tail calls,
%     coroutines, jumps into supervisor mode.
%   \item CHERI and its operation, documentation and practical configurations and
%     operation.
%   \item Compare and contrast with StkTokens.
%   \item Under what circumstances can stack safety exist without an enforcement
%     mechanism?
%   \item Zeroing out of allocated space inside blessed sequences, eager vs lazy.
%     Possibility of blessed alloc and dealloc sequences, which also manipulate SP
%     and would not be legal without blessing. In such an extended setting, the
%     return sequence would check for matching addresses (it would be nice to have
%     testing for this!).
%   \end{itemize}
% }

%% Acknowledgments
\begin{acks}                            %% acks environment is optional
                                        %% contents suppressed with 'anonymous'
  %% Commands \grantsponsor{<sponsorID>}{<name>}{<url>} and
  %% \grantnum[<url>]{<sponsorID>}{<number>} should be used to
  %% acknowledge financial support and will be used by metadata
  %% extraction tools.
  This material is based upon work supported by the
  \grantsponsor{GS100000001}{National Science
    Foundation}{http://dx.doi.org/10.13039/100000001} under Grant
  No.~\grantnum{GS100000001}{nnnnnnn} and Grant
  No.~\grantnum{GS100000001}{mmmmmmm}.  Any opinions, findings, and
  conclusions or recommendations expressed in this material are those
  of the author and do not necessarily reflect the views of the
  National Science Foundation.
\end{acks}


%% Bibliography
\bibliography{bcp.bib,local.bib}


%% Appendix
%\appendix
%\section{Appendix}
%Text of appendix \ldots

\end{document}
