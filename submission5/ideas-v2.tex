The paper is structured as follows. In \cref{sec:example}, we will walk through a function
call in a simple example machine and discuss how each of our properties applies to it,
informally. Then in \cref{sec:machine} we will introduce the machine model that we test, and give
its {\em security semantics}, described in brief below. We formalize the definitions of
stack safety that derive from the security semantics in \cref{sec:props}.
\Cref{sec:testing} describes the testing framework, and
\cref{sec:relwork,sec:future} discuss related and future work.
The remainder of this section introduces ideas that will be important throughout.
%
%% \rb{Or shorter: references in contributions + final sentence at end of intro}

%% \rb{Turn these back into paragraphs later?}
\subsection{Permissiveness, Laziness, and Tagged Architectures}

Security properties tend to be conservative, identifying violations that would be
harmless in practice. But enforcement mechanisms make trade-offs in the name of performance
and of compatibility with existing codebases. In a crucial motivating example for this work,
Roessler and DeHon \cite{} present a collection of tag-based enforcement mechanisms
using PIPE \cite{???}, a tagged hardware reference monitor. In monitors such as PIPE and
STAR \cite{???}, every value in memory and registers is paired with a {\em metadata tag}.
A set of software-defined rules, termed a micro-policy, determines the tags for operation
results based on the tags of their inputs---or, if the input tags are in a dangerous
combination, it may halt the machine (a ``failstop.'')

Roessler and DeHon find that the intuitive way to enforce stack safety in this setting
is inefficient when allocating and clearing stack frames, because each word must be initialized
with a tag corresponding to the (depth of the) current function activation. They propose
{\em Lazy Tagging and Clearing}, in which tags are not initialized at allocation, but only
when a function writes to memory. This means that writes cannot be prevented, necessarily
violating existing models of stack safety. Instead the micro-policy enforces that memory be
read by the same fuction that wrote it. The owner of the data will never see the changes,
so it is secure in the sense that the attacker is unsuccessful at changing program behavior
(except, perhaps, by causing a fault.) We elaborate that sense, and test whether the
micro-policy fulfills it.

This key motivation leads us to present integrity and confidentiality as predicates on (pairs of)
machine states that make claims about the future observable behavior of execution from those states.
That is not to say that our properties are specialized for lazy micro-policies. It also supports
normal ``eager'' micro-policies, and should support other enforcement mechanisms. In fact, \(\clri\)
and \(\clrc\) are even weaker than necessary to support Roessler and Dehon's micro-policy,
while \(\clec\) and \(\clei\) are stronger properties that they do not attempt to enforce.
%
\rb{Emphasize the point about eager properties, here, when discussing other
  enforcement types?}
%
\rb{(Mentioned properties are still ``forward references'' of a sort as they
  haven't been formally introduced yet)}
%% \rb{Does this discussion (at least in part) belong in the general background
%%   section, some even in key ideas? It still feels a little out of order to me,
%%   and occasionally too detailed for the introduction?}

\rb{Moved last bit to the framework discussion below, does this work?}

\rb{Test: the comparison of enforcement mechanisms \emph{and} the hierarchy of
  properties \emph{before} the security semantics. Does this work? Easy to move
  around/consolidate. Some things still feel a bit out of order. The comparison
  motivates the hierarchy, and also parts of the security semantics -- too much
  at once? Also: bad subsection/paragraph title ahead.}

\subsection{A Formal and Flexible Security Framework}

While a concrete aim is to provide a stack-safety property suitable for testing
lazy enforcement for a micro-policy architecture, the formal framework we
develop could be adapted to other protection mechanisms.
%
The need for these policy-specific adaptations is a natural and predictable
consequence of the protean nature of the call stack (discussed in \cref{XXX}),
and does not detract from the generality of our properties and their
underlying \rb{high-level?} design principles.
%
Nevertheless, some enforcement methods lends themselves to modeling more easily
than others. Micro-policies in particular stand out for their elegance, which
results in particularly clean instances of our properties, and their
flexibility, which enables them to implement very general forms of stack
protection \rb{(and so we get to exercise our properties fully)}.

Let us revisit our menagerie of stack protections from \cref{XXX} to motivate
our choices. \cref{tab:relwork} summarizes and compares those techniques along
several representative axes.

\newcommand*{\rot}[1]{\hbox to1em{\hss\rotatebox[origin=br]{-60}{#1}}}
\newcommand*{\tabyes}{\CIRCLE\xspace}
\newcommand*{\tabno}{\Circle\xspace}
\newcommand*{\tabjein}{\RIGHTcircle\xspace}

\begin{table}
\begin{threeparttable}
  \begin{tabular}{l | c c c c | c c c | c}
    \toprule
    \multicolumn{1}{l |}{Mechanism}
    & \multicolumn{4}{l |}{Features}
    & \multicolumn{3}{l |}{Properties}
    & %% Formal criteria
    \\
    \midrule
    %% Technique
    \multicolumn{1}{c}{}
    & \rot{Strict enforcement}
    & \rot{Deferred checking}
    & \rot{Low-level code support}
    & \multicolumn{1}{c}{\rot{Partial program support}}
    & \rot{Well-bracketed control flow}
    & \rot{Data integrity}
    & \multicolumn{1}{c}{\rot{Data confidentiality}}
    & \rot{Formal criteria}
    \\
    \midrule
    Stack canaries  & \tabno   & \tabjein & \tabyes  & \tabyes  & \tabjein & \tabno   & \tabno   &          \\
    Shadow stacks   & \tabno   & \tabjein &          & \tabyes  & \tabjein & \tabno   & \tabno   &          \\
    Bounds checking & \tabyes  & \tabno   & \tabno   & \tabno   & \tabyes  &          &          &          \\
    Split stacks    & \tabyes  & \tabno   & \tabno   & \tabyes  & \tabyes  & \tabno   & \tabno   &          \\
    \midrule
    Capabilities    &          & \tabno   &          & \tabyes  &          &          &          & \tabjein \\
    Micro-policies  & \tabyes  & \tabyes  & \tabyes  & \tabyes  & \tabyes  & \tabyes  & \tabyes  & \tabyes  \\
    \bottomrule
  \end{tabular}
  \caption{Comparison of Stack Protection Mechanisms}
  \begin{tablenotes}
  \item \tabyes indicates a protection mechanism possesses the feature
  \item \tabno indicates the mechanism lacks the feature
  \item \tabjein indicates partial or incomplete support
  \item \rb{Complete and check columns, describe. Capitalize?}
  \item \rb{Order? Split int/conf? Deferred/lazy at the end of features}
  \item \rb{2+ lines for capabilities, before and after using our draft?}
  \end{tablenotes}
  \label{tab:relwork}
\end{threeparttable}
\end{table}

\subsubsection{Strict Property Enforcement}
%
We are primarily interested in the \emph{strict} enforcement of a set of
security properties of choice, i.e., no violations are allowed. Some \rb{most?}
of the most commonly deployed countermeasures, like stack canaries and shadow
stacks, offer only \emph{approximate} enforcement. As such, they will not
satisfy any of our properties per se. We could use our framework to model them
with some effort, but the interest of this exercise would be limited: testing
would easily find the expected, unavoidable property violations. A more complex,
\emph{probabilistic} version of the framework could offer adequate reasoning
capabilities for these defenses.

\subsubsection{Deferred and Lazy Properties}
%
We seek to support both \emph{eager} property enforcement (where potential
violations are detected immediately) as well as more permissive \emph{lazy}
properties (as motivated in \cref{XXX}).
%
Micro-policies are well suited to both types of properties. There is an
interesting connection between lazy enforcement and the deferred checks of stack
canaries and shadow stacks. Other techniques, like split stacks and bounds
checking---and the closely related hardware capabilities---are fundamentally
eager techniques.

\subsubsection{Low-Level Component Support}
%
Our framework considers a \emph{single-, low-level view} of all program code and
data; a key advantage of this approach is that it allows us to remain
\emph{language-agnostic}.
%
Some software-based protection mechanisms are designed specifically with
a \emph{high-level language} in mind, e.g. bounds checking and split stacks.
They tend to offer stronger security guarantees than other software techniques.
In exchange, their threat model becomes weaker, as they make additional
assumptions about the admissible behavior of program components
%
(for example, bounds checking only applies to code that has been instrumented by
a \emph{secure compiler} \ref{}).
%
In general, a multi-level extension of the framework offering adequate reasoning
capabilities for this class of policies would need to model multiple source
language and a notion of code provenance---a reflection of the call stack as the
melting pot of language abstractions.
%
The common case of one source and one target language mediated by a compiler may
be solved by a simpler lifting of our framework from the machine to the source
level \rb{e.g., SoftBound could use a C language semantics and Tagged C for
security modeling (lifted properties would likely hold under the expected
assumptions)}.

\subsubsection{Partial Program Protection}
%
A consequence of the use of high-level software techniques is the difficulty of
keeping their shields up against a \emph{low-level attacker} that is not bound
to play by their rules.
%
Insofar as their protections continue to work against untrusted code, like split
stacks, they do so by relying on additional support, e.g. from the operating
system.
%
\rb{(The same remarks about the language-agnostic framework apply here. Split
  stacks are probably relatively easy to model without big changes, but would
  still benefit from a higher level of abstraction.)}
%
\rb{(partial programs and split stacks? jein?)}

\subsubsection{Enforced Properties}
%
\rb{(FIXME: Present with the hierarchy of properties? Basically all policies try
  to protect {\wbcf}. Without it things are quite strange. Some of the more
  restrictive software methods and the hardware methods can cover additional
  properties.)}

\subsubsection{Software vs. Hardware}
%
One first general observation about software-based techniques is that they
are \emph{less modular} than their hardware counterparts, that is, the
countermeasures are more tightly coupled with the program code they protect.
%
By contrast, although programs must effectively leverage the protection
mechanisms offered by the hardware, the distinction between those protections
and the program logic itself is much cleaner.
%
Our framework relies on a \emph{separation} of security-relevant operations and
regular program code \rb{stress that any scalable, composable solution must do
this!}.
%
While the various extensions suggested in this section could help abstract the
complexities of software safeguards, hardware techniques remain the most elegant
and general exemplars for the methods we seek to demonstrate.
%
\rb{(This is also related to our single-level model)}

\subsubsection{Formal Models}
%
A second general observation about software-based techniques is that they tend
to be \emph{less absolute}: they enforce weaker properties or under stronger
assumptions, and their threat models are usually fairly informal.
%
%
The closest previous attempt
to produce a formal characterization of stack safety does so using a {\em fully abstract overlay semantics},
a style of correct-by-construction machine \cite{SkorstengaardSTKJFP}.
But this approach
builds the security properties into the semantics of the machine without externalizing them or their meaning.
It is also not suitable for the {\em lazy} enforcement mechanism
of Roessler and DeHon, in which stores are not checked, and illegal stores are
only caught when their location is later read.
%
\rb{Need to discuss overlay semantics, but also note that it is very intensional
  and very specific, as the authors acknowledge. I'd introduce this a bit more
  gradually. Also, if phrased this way, it sounds like its main drawback is a
  lack of generality, to which we should contrast a more general approach, not
  just one that is specialized to a specific policy, i.e. lazy micro-policies
  (I'd try to present these more gradually as well).}
%
\rb{Also, testing certainly (and motivate), but not only?}

Overall, the generality and flexibility of micro-policies, arising from an
elegant design that clearly separates the program from its security, are the
most appropriate choice for an exploration of this new framework.
%
\rb{FIXME: Finish motivation. We focus on micro-policies, but\ldots Something
  about existing policies, exercising the framework most generally,
  representative experiments?}

\rb{Begins to anticipate subsequent discussion. This is even something of an
  understatement: some papers (esp. newer ones?), make concrete claims, though
  they remain informal. Arguably, none have a truly formal characterization in
  mind (with the qualified exception of \cite{SkorstengaardSTKJFP}): this is one
  of the novelties of this work!
\rb{(+SNA) Indeed, this point needs to be made carefully. We claim to go further
  than previous work, without showing a fully general framework (which would be
  highly parametric, etc.). Tie to the idea of micro-policies as a good place to
  start because of their flexibility, elegance, enabling a wide of interesting
  experiments, etc.
\rb{Basically, none reach the point of being an independent (and executable) specification}
}
}

\subsection{The Properties and Their Hierarchy}

The properties form a hierarchy, shown in [TODO: diagram].
Within properties that are orthogonal
some are likely to take priority in terms of their
importance to security. Almost all stack-safety enforcement mechanisms will try to enforce \(WBCF\),
as control-flow attacks such as ROP are a major vulnerability that can be targeted via the stack.
Many also cover some degree of (caller) integrity. Caller confidentiality has been fairly neglected,
and callee-oriented properties even moreso.

\subsection{Security Semantics and Property Structure}

A {\it security semantics} extends a machine
with additional context about the identity of current and pending
functions (which act as security principals) and about the registers and memory they require
to be secure. This added context is purely notional;
it does not affect the real machine. The security context
evolves dynamically through the execution of {\it security-relevant operations},
which include events like calls, returns, and frame manipulations.
Our security properties are phrased in terms of this context, often as predicates
on future states, e.g. of the form ``when control returns to the current function...'',
or relations on traces of future execution (hyper-properties).

The security-relevant operations typically correspond to underling machine instructions,
but the correspondence may not be obvious just by inspecting the machine code.
For example, in the tagged RISC-V machine we use in our examples and tests,
calls and returns are conventionally performed by {\tt jal} (``jump-and-link'')
and {\tt jalr} (``jump-and-link-register'') instructions, respectively, but these
instructions might also be used for other things. An instantiation of an operation
might require some external mechanism for distinguishing which instructions are
security-relevant; the most obvious technique is to add optional labels to instructions,
but the theory does not depend on this choice.

The set of security-relevant operations (\(\Psi\)) covered in this paper is given in
\cref{tab:psi}. Those that we explore in detail in \cref{sec:example}
are shown in white: calls, returns, and the allocation and deallocation of local memory.
In \cref{sec:machine}), we cover additional operations that add parameters to the examples',
to support various forms of memory sharing \ifexceptions , exceptions, \fi
and tail-call elimination. These are shown in light gray.
We give a proof-of-concept formalization of a more sophisticated, heap-like sharing model
based on pointer provenance in \cref{app:ptr}, represented by the dark grey
operations, but we do not test this.

\newcommand{\example}{\rowcolor{black!0}}
\newcommand{\testing}{\rowcolor{black!10}}
\newcommand{\theory}{\rowcolor{black!25}}

\begin{table}
  \begin{tabular}{| l | l |}
    \hline
    \(\psi \in \Psi\) & Parameters \\
    \hline
    \example \(\mathbf{call}\) & Target address, argument registers \\
    \testing & Stack argument offset \& size \\
    \example \(\mathbf{return}\) & \\
    \example \(\mathbf{alloc}\) & Offset \& size \\
    \testing & Public flag \\
    \example \(\mathbf{dealloc}\) & Offset \& size \\
    \testing \(\mathbf{tailcall}\) & As \(\mathbf{call}\) \\
    \hline
    \multicolumn{2}{|c|}{{\it Pointer provenance operations}} \\
    \hline
    \theory \(\mathbf{promote}\) & Register, offset \& size \\
    \theory \(\mathbf{propagate}\) & Source register/address \\
    \theory & Destination register/address \\
    \theory \(\mathbf{clear}\) & Target register/address \\
    \hline
  \end{tabular}
  \caption{Security-relevant Operations and their Parameters}
  \label{tab:psi}
\end{table}

Our security semantics has similarities to the overlay semantics that have been proposed
to characterize stack safety in the past~\cite{SkorstengaardSTK}, but it does not restrict
the behavior of the underlying machine in any way. Rather, it tracks additional context
about the history of security-relevant operations, which inform the criterion a machine
must satisfy in order to correctly obey the properties.

The security principals in our account of stack safety are function
activations: at any given time, pending activations
have security requirements that are recorded in a call stack, and the current
activation tracks context information that will inform its requirements when
it makes a call. Each activation has a {\it view}
of the system that maps each state element to a {\it security class}:
it may be private to an inactive principal (\(\sealed\)),
in an allocated object that is accessible to the current principal (\(\object\)),
available to be allocated (\(\unsealed\)), or
always accessible (\(\public\)).

\sna{Might like a better name than object, because morally it ought to apply to arguments as well.}

\subsection{Events}

It is common to specify the behavior of a system, including security properties, in
terms of traces of observable {\em events} rather than details of the machine state.
Lazy enforcement works because the deferred check catches errors before they appear
in the trace. The nature of events is a parameter of the machine, and can be specialized
to any notion of observable behavior.

\subsection{Threat Model}

We must trust that our method of distinguishing security-relevant operations is accurate; if it
involves labels placed on code by a compiler, that means trusting that the compiler placed
those labels correctly. If operations occur that are not recognized, those operations
might not be guaranteed to protect their principals---for instance, an unlabeled call
might not protect the caller's data from the callee. On the flip side, applying an incorrect
label most likely means that the property becomes too strong to be enforced.

Otherwise, we do not assume that the code adheres to any particular
calling convention or implements a source language construct.
In particular, while we are agnostic as to the source
language, we certainly aim to support C, and so any source function might contain undefined
behavior resulting in its compilation to arbitrary machine code. A given enforcement
mechanism may place additional constraints, particularly on the behavior of
call and return sequences. For instance, extant implementations tend to assume
implicitly that callee-saved registers are treated appropriately by whichever compiler
generated their code. Our properties explicitly require them to be treated as belonging
to the caller, which could be enforced by a micro-policy or by a well-behaved compiler.

In general, it is impossible to distinguish buggy machine code from an attacker; in
our examples we will identify one function or another as an attacker, but we do not
require any static division between trusted and untrusted code, and we aim to protect
even buggy code.

This is a strong threat model, but it does omit some important aspects of
security enforcement: In particular, hardware and timing attacks are out of scope.

\subsection{Limitations}

%Our concurrency model is fairly
%simplistic, assuming a fixed number of threads each with its own dedicated processor.
We model memory-safe stack objects, but not a heap. Regions outside of
stacks can be used however the compiler likes, including as a heap, but no protection is
built in and our properties assume that if a pointer to a stack object is stored there,
it is permanently compromised.

Our properties are termination insensitive: in order to support enforcement mechanisms
that failstop rather than execute dangerous code, the properties must treat failstops
as invisible to the attacker. We extend this insensitivity to all errors for simplicity.
This is consistent with our treatment of properties that specify the state after a return,
in that a callee that diverges for any reason will never reach the return, and will fulfill
the property vacuously.
%
\rb{Merge with threat model? Orthogonality of the heap, morally
  similar interesting setups on the stack? Termination sensitivity and
  capability machine semantics?}

%%%

\rb{\textbf{A menagerie of protection mechanisms.}
%
%% We have made explicit the key point that there is \emph{not} a single stack
%% abstraction. This paves the way for a \emph{general} idea of stack protection
%% that is widely applicable with high-level concepts that remain unchanged (and I
%% think it's particularly important to at least introduce these more immutable
%% parts early on). The fact that this general ``framework'' requires low-level or
%% specific specializations is a predictable consequence of the protean nature of
%% the call stack.
%% %
%% \\
%% %
%% Let us return to a selection of representative policies in our menagerie and
%% consider how they fit into the framework. At the highest level, we can classify
%% all those techniques into two large groups, software-based and hardware-based.
%% Traditionally, most practical protections are implemented in software, so we
%% start by considering this first group.
%% %
%% \\
%% %
%% We can make two general observations about these software based techniques.
%% Firstly, (as we will see: discussion? technical details?) these techniques are
%% \emph{less modular} than their hardware counterparts. That is, the
%% countermeasures are tightly coupled with the program code they protect, and
%% consequently more effort is required to disentangle the program from the
%% protections, and to understand the impact of the latter on the former. This is
%% true of any systematic attempt to study the security of stack protection
%% policies, including their adaptation to our framework. (Owing to this added
%% complexity, software policies will not be the cleanest or most elegant exemplars
%% of use of the framework, its salient features and practical utility; this partly
%% motivates the focus on hardware policies.)
%% %
%% \\
%% %
%% (The above remark holds \emph{at least in a ``single-level model''} that
%% considers assembly language/machine code programs. In general, (low-level)
%% protections are introduced by a compiler or similar tool. One could consider a
%% multi-level extension of the framework pairing models of high-level programs and
%% the security transformations performed by a (stack-secure) compiler.)
%% %
%% \\
%% %
%% Secondly, the guarantees generally offered by software-based policies are
%% \emph{less absolute}, being defined more or less informally and usually
%% providing few definite guarantees of absence of \emph{general} classes of
%% errors. (The following discussion elaborates on this. It is conceivable that an
%% extended and more sophisticated framework could add \emph{probabilistic
%% reasoning} capabilities to capture and reason about the effects of less strict
%% protection mechanisms. This would be another orthogonal development to what is
%% presented here. In this paper, we are interested in assessing the complete
%% \emph{absence} of broad classes of vulnerabilities --- modulo any modeling gaps
%% or defects. In passing, this would extend the security properties to e.g. larger
%% system verification efforts.)
%% %
%% \\
%% %
%% One by one, consider the following specimens of software protection mechanisms
%% (quick explanations inlined?):
%% %
%% \\
%% %
%% \begin{itemize}
%% %
%% \item \emph{Stack canaries}. The purpose of canaries is to \emph{increase} the
%%   difficulty of some attacks on the stack, usually (always?) buffer overflows.
%%   They can never provide any absolute security guarantees and therefore do not
%%   satisfy any of our properties (some of the properties, e.g. confidentiality,
%%   are completely out of scope.) Although tedious, we could we could model this
%%   mechanism in our framework with the appropriate adaptations --- and
%%   unsurprisingly, testing would find property violations, which we already know
%%   are there: the results of this effort would be unsurprising and not as useful
%%   or instructive as other experiments.
%%   \rb{(+SNA) They are kind of lazy, aren't they?}
%% %
%% \item \emph{Bounds checking}. This is fundamentally a high-level technique, even
%%   more obviously so than stack canaries (which could still detect violations
%%   caused by external code). Bounds checking only applies to code that has been
%%   instrumented by the ``secure compiler'' (i.e., to whole programs); it does not
%%   impose any constraints on adversarial code. The attacker model is therefore
%%   rather weak. (Certain variants, like SoftBound, are good targets for lifting
%%   our framework to the source level by e.g. using a C-language semantics and
%%   using Tagged C for security modeling; the lifted versions of our properties
%%   would likely hold under the expected assumptions. HardBound is probably an
%%   interesting transitional case because it uses capability-like hardware
%%   primitives for enforcement; it could be close to ongoing and future work to
%%   verify stack safety on capability machines, e.g. based on Cerise -- discuss
%%   together?)
%% %
%% \item \emph{Shadow stacks}. Like stack canaries, shadow stacks are
%%   ``probabilistic techniques'' in that they ``restrict the flexibility available
%%   in creating gadget chains'' (Shanbhogue et al.). This technique is solely
%%   concerned with control-flow integrity, and so closely related to our
%%   well-bracketed control flow. Nevertheless, and again, it is only an
%%   approximate technique, and our testing framework would be able to find the
%%   expected attacks against our security property.
%% %
%% \item \emph{Split stacks}. The security claims of split stacks are stronger than
%%   those of most software countermeasures, i.e., they are meant to guarantee
%%   their titular code-pointer integrity. They explicitly offer no protections on
%%   stack data (i.e., integrity and confidentiality), and so restrict themselves
%%   to well-bracketed control flow. (This is probably closer to our framework than
%%   most other software techniques, and it seems like it would be easier to model
%%   as well. However, this is again work based on high-level languages and would
%%   benefit from a higher level of abstraction.)
%% %
%% \item \ldots
%% %
%% \end{itemize}
%% %
%% Opposed to the above group of software techniques we have hardware-based
%% protection policies (more in background/technical preliminaries?). Although
%% programs must effectively leverage the protection mechanisms offered by the
%% hardware, the distinction between those protections and the program logic itself
%% is much cleaner: it is fairly simple to separate the use of the security
%% features in the hardware (and their effects in program code) from regular
%% program code. They also tend to aspire to offer complete protection against
%% entire classes of attacks. Both aspects make this family of techniques a better
%% fit for illustrating and showcasing our techniques.
%% %
%% \\
%% %
CHERI and lightweight formal methods?
%
Questionably that they are strictly more realistic in the sense that it is very
costly or requires special sophisticated/experimental types of capability not
available/perhaps not very feasible to implement.
%
Problems of adapting this work to capability machines: security semantics tightly
coupled to regular machine semantics and to a complex calling convention, with
more complex side conditions...
%
\\
%
Why favor micro-policies? Their design is more flexible and arguably more
general than capability machines (of course, they pay a performance price for
this). This generality and ease of programming make them good candidates for an
exploration the design space of stack protection policies and proof of concept
implementations. On the practical side, there is a number of existing policies
available to examine (not just lazy policies!) --- and we find how they satisfy
our properties, how they do not, and errors in some of them, and see how to
correct them. Lazy enforcement is particularly novel, but not the only
possibility. This additional breadth is beneficial for the proof of concept
stage.
%
\\
%
Regarding capability machines, we know that most designs based on capability
machines do not satisfy our properties (or do so incurring significant costs),
and that fixes are not trivial and involve new types of capabilities (in a
sense, are they any more ``realistic?'' Only the latest Cerise paper would seem
to satisfy our properties, is directly influenced by previous versions of this
work, and is a reasonable target for ongoing/further study -- but this is not
easy, and even the author(s) estimate it would be a significant amount of work,
especially full proofs (references, communications, etc.).
\rb{(+SNA) Discuss the degree in which the various papers satisfy the
  properties\ldots and the degrees to which they \emph{aim to}. E.g. originally
  they don't talk about confidentiality. Do they have callee integrity? Maybe
  they aren't thinking about it? (This kind of discussion is a good motivation
  for the formal properties. Also related to the hierarchy. And: how to convey
  that the latest paper draws inspiration directly from this work?}
%
\\
%
\ldots
%
}
%% \rb{(+SNA) Any interesting tables comparing specimens in the menagerie?}
%% \rb{(+SNA) A general multi-level extension of the framework would need to model
%%   multiple source languages. Among other things, the stack is the meeting place
%%   of all sorts of different language abstractions! One of the advantages of the
%%   single-level model is that it lets us be agnostic about the language. (On the
%%   other hand, it is likely that a single source and target mediated by the
%%   compiler suffice to model many existing techniques, coming e.g. from systems
%%   programming and C-like languages.}

\rb{Connect this (novel characterization?) to the shortcomings of previous work (later!). I'd give at least a quick
  explanation of what the various properties roughly entail in practice (e.g.,
  secrecy of private data, etc.). The comment about the single extension of
  exceptions (still in intro) seems disconnected from the rest.}

\rb{Deferred/lazy connection of canaries and shadow stacks -- arguably these helps them support things other software policies need more help with?}

\rb{Capabilities and HardBound?}
