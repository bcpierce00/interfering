\documentclass{article}

\usepackage{geometry}
\usepackage{amsmath, amssymb}
\usepackage{mathtools}

\begin{document}

\section{Security Overlay Semantics (Concrete)}

A {\it security overlay semantics} divides a program's execution into
{\it principals}, and specifies what it means for such principals to be
protected from one another by a high-level abstraction implemented
in a concrete machine. Specifications may take the form of predicates on
future states: statements of the form, ``When control returns to me...'',
or of relations on traces of future execution (hyper-properties.)

What this means in practice is that we augment a concrete machine with
additional {\it overlay state}, which tracks information related to the
principals and their privileges, as well as which is active.
We identify some sequences of instructions
as implementing {\it overlay operations}. An overlay operation either
transfers control between principals, or modifies the active principal's
privileges. And we enrich the underlying machine with an {\it overlay step}
that modifies the overlay state accordingly.

Let \(\Psi\) be the set of overlay operations ranged over by \(\psi\).

We begin with the base machine, whose states are drawn from a set \(\mathcal{M}\),
and which transitions via partial step function \(\rightharpoonup\).
We define several operations on states. \(m[k]\) accesses the word in register
or address \(k\). \(m\langle PC \rangle\) maps a state to a pair of a word
and an optional indexed transition \((\psi,i)?\) found at the address pointed to by the
program counter.

If \(m\langle PC \rangle = (w, \bot)\) or \((w, (\psi,i))\) for \(i > 0),
then we step by \(\rightharpoonup\).
If \(m\langle PC \rangle = (w, (\psi,0))\), we instead step by \(\xRightarrow{\psi}\),
which executes until it reaches a state \(m'\) such that \(m\langle PC \rangle = (w',\psi?)\)
where \(\psi? = \bot\) or \((\psi',i)\) for \(\psi \not = \psi'\).

\subsection{Starting Overlay Operations in the Middle}

An overlay operation can contain multiple instructions, and its security guarantees
may not hold if they are executed out of order. In fact, they almost certainly will not!
So we do not treat such executions as overlay operations, and our properties can be
statisfied even if such executions behave insecurely. It is the responsibility of the
enforcement mechanism to guarantee that this does not happen, for instance by tagging
the operation as a blessed sequence.

\subsection{The Compiler's Responsibility}

The code that is acted on by a security overlay semantics is precisely that code
that the compiler provides, down to any instruction tags that are needed to enforce
the associated micro-policy. The compiler is also responsible for labeling overlay
operations as such. As noted above, this puts the burden on the compiler to be aware
of the available overlay operations and use them intentionally.

In fact, in the strictest sense, the overlay semantics has {\it nothing} to say
about which sequences make up an operation. It is for the compiler to determine that
when it constructs the sequence. So, the compiler can make the sequence as short or
as long as it wants; there ends up not being much reason for a sequence longer than
a single instruction, except in the very strange case where the same sequence might be
entered from the start {\it or} in the middle, and the compiler wants to assert protections
only in the former case.

\subsection{Further Compilation}

The overlay semantics makes absolutely no changes to the behavior of the system,
so further compilation is not necessary.

\section{Security Overlay Semantics (Abstract)}

A {\it security overlay semantics} implements a high-level abstraction
using the instructions available to a concrete machine, and makes assertions
about its security. These assertions divide a program's execution into
{\it principals} and specify what it means for such principals to be
protected from one another by the high-level abstraction.
Specifications may take the form of predicates on
future states: statements of the form, ``When control returns to me...'',
or of relations on traces of future execution (hyper-properties.)

What this means in practice is that we augment a concrete machine with
additional {\it overlay state}, which tracks information related to the
principals and their privileges, as well as which is active.
We identify some sequences of instructions
as implementing {\it overlay transitions}. An overlay transition either
transfers control between principals, or modifies the active principal's
privileges. And we enrich the underlying machine with an {\it overlay step}
that executes the instructions of an overlay transition atomically
and modifies the overlay state accordingly.

Let \(\Psi\) be the set of overlay transitions ranged over by \(\psi\).

We begin with the base machine, whose states are drawn from a set \(\mathcal{M}\),
and which transitions via partial step function \(\rightharpoonup\).
We define several operations on states. \(m[k]\) accesses the word in register
or address \(k\). \(m\langle PC \rangle\) maps a state to a pair of a word
and an optional transition \(\psi?\) found at the address pointed to by the
program counter.

If \(m\langle PC \rangle = (w, \bot)\), then we step by \(\rightharpoonup\).
If \(m\langle PC \rangle = (w, (\psi,0))\), we instead step by \(\xRightarrow{\psi}\),
which executes until it reaches a state \(m'\) such that \(m\langle PC \rangle = (w',\psi?)\)
where \(\psi? = \bot\) or \((\psi',i)\) for \(\psi \not = \psi'\). We do not step
from any state where \(m\langle PC \rangle = (w, (\psi,i))\) for \(i > 0\).

\subsection{Starting Overlay Operations in the Middle}

An overlay operation can contain multiple instructions, and its security guarantees
may not hold if they are executed out of order. In fact, they almost certainly will not!
So we do not treat such executions as overlay operations, and our properties can be
statisfied even if such executions behave insecurely. It is the responsibility of the
enforcement mechanism to guarantee that this does not happen, for instance by tagging
the operation as a blessed sequence.

\subsection{The Compiler's Responsibility}

\subsection{Further Compilation / Protection}

A security overlay semantics uses the same instructions as the base machine,
with the same semantics, and protected by the same security mechanism. But it
is a more abstract machine in that it is also 

\section{A Barely Abstract Stack Machine}

\subsection{The Compiler's Responsibility}

\subsection{Further Compilation / Protection}

\section{A Slightly More Abstract Stack Machine}

\subsection{The Compiler's Responsibility}

\subsection{Further Compilation / Protection}

\end{document}
