\begin{figure}[b]
    \begin{minipage}{.3\textwidth}
\begin{verbatim}
int main() {
  int x = 42, y = 0;
  y = f(10);
  print (x + y);
}
\end{verbatim}
    \end{minipage}
    \begin{minipage}{.3\textwidth}
\begin{verbatim}
int f(z) {
  print z;
  return (z+z);
}
\end{verbatim}
    \end{minipage}
\caption{A Simple Call}
\label{fig:simple-program}
\end{figure}


\newcommand*{\add}{\textsc{add}}
\newcommand*{\addi}{\textsc{addi}}
\newcommand*{\sw}{\textsc{sw}}
\newcommand*{\lw}{\textsc{lw}}
\newcommand*{\jal}{\textsc{jal}}
\newcommand*{\jalr}{\textsc{jalr}}
%\newcommand*{\rsp}{\textsc{sp}}
\newcommand*{\rra}{\textsc{ra}}
\newcommand*{\rout}{\textsc{out}}

\newcommand*{\tagInstr}{\textsc{instr}}
\newcommand*{\tagCall}{\textsc{call}}
\newcommand*{\tagHa}{\textsc{h1}}
\newcommand*{\tagHb}{\textsc{h2}}
\newcommand*{\tagRa}{\textsc{r1}}
\newcommand*{\tagRb}{\textsc{r2}}
\newcommand*{\tagRc}{\textsc{r3}}
\newcommand*{\tagNoDepth}{\textsc{unused}}
\newcommand*{\tagStackDepth}[1]{\textsc{stack} ~ #1}
\newcommand*{\tagPCDepth}[1]{\textsc{pc} ~ #1}
\newcommand*{\tagSP}{\textsc{sp}}

\newcounter{pcctr}
\setcounter{pcctr}{0}
%  \thepcctr & \stepcounter{pcctr} #1 & #2 & #3 & \text{#4}\\
\newcommand*{\row}[4]{
  \thepcctr & \stepcounter{pcctr} #2 & #3 & \text{#4}\\
}
\newcommand*{\tracerow}[5]{
  #1 & #2 & #3 & #4 & \text{#5}\\
}
\newcommand*{\summary}[6]{
\[
\begin{array}{cccccc}
  \PCname = #1 & r_0 = #2 & \rra = #3 & \rsp = #4 & r_4 = #5 & r_5 = #6
\end{array}
\]
}
\newcommand{\negate}{\textrm{-}}

\colorlet{lgray}{gray!40}

\newcommand{\instrc}{lgray}
\newcommand{\mainsealc}{cyan}
\newcommand{\fsealc}{green}
\newcommand{\unsealc}{gray}
\newcommand{\emptyoutc}{white} % or gray for consistency with unusedc ?
\newcommand{\fulloutc}{white}

\begin{figure}

\begin{center}
\begin{tabular}{l l}
Entry to {\tt main}
 &
\memoryaddrs{25em}
\memory{3}{\instrc}[{\makebox[0pt]{Instructions}}]%
\hspace*{3pt}
$\cdots$
\memory{1}{\emptyoutc}[{\makebox[0pt]{Output}}]%
\memory{8}{\unsealc}[{\makebox[0pt]{Unsealed}}]
~$\cdots$
\\ \\
Call to {\tt f}
 &
\memoryaddrs{35em}
\memory{3}{\instrc}%
\hspace*{3pt}
$\cdots$
\memory{1}{\emptyoutc}%
\memory{8}{\unsealc}
~$\cdots$
\MemoryLabel{-37.5em}{0.75em}{42}
\MemoryLabel{-33.5em}{0.75em}{0}
\\ \\
Entry to {\tt f} &
\memoryaddrs{35em}
\memory{3}{\instrc}%
\hspace*{3pt}
$\cdots$
\memory{1}{\emptyoutc}%
\memory{2}{\mainsealc}[{\makebox[0pt]{Sealed(0)}}]%
\memory{6}{\unsealc}[{\makebox[0pt]{Unsealed}}]
~$\cdots$
\\ \\
Return to {\tt main}
 &
\memoryaddrs{25em}
\memory{3}{\instrc}%
\hspace*{3pt}
$\cdots$
\memory{1}{\emptyoutc}%
\memory{8}{\unsealc}
~$\cdots$
\MemoryLabel{-37.5em}{0.75em}{42}
\MemoryLabel{-33.5em}{0.75em}{0}
\\
\end{tabular}
\end{center}

\vspace{\abovedisplayskip}

\caption{Detailed Execution Trace
%  \rb{A suggestion for the description of $\jal$ and $\jalr$:
%    call(function label) [if labels are added next to the entrypoints]
%    or call(address of entrypoint) and return()?}
}
\label{fig:simple-trace}
\end{figure}

\Cref{fig:simple-program} shows a high-level view of a running example,
written in C-like pseudo-code.  It consists of a {\tt main} function that
performs a call to a function {\tt f}, which prints its argument, performs
arithmetic, and returns a result. \Cref{fig:simple-trace} shows an
execution trace of the corresponding machine instructions. These might have
been generated from the high-level code by a (rather simple-minded) compiler,
but we focus on the behavior of the machine instructions. Note that for this
simple example, arguments and return values are passed on the stack.

To streamline the example, we use a small instruction set reminiscent of
RISC-V and make a few simplifying assumptions about the organization of
memory.
%
First, as noted above, since our goal is to formalize stack safety, we
ignore the heap and assume that all variables are stack allocated.
%
We also assume that the machine is word- (not byte-) addressed, that program
instructions start from location 0, that the machine communicates with the
outside world through a single memory-mapped output port at address 99 (so
a {\tt print} statement will be compiled to a store to this
location), and that the stack begins at location 100 and grows upwards.
% \leo{Explain more?  How to connect to realistic one? Observations? DMA?
%   Interrupt handlers?}\bcp{Yes.}

A sketch of the initial memory layout appears at the top of
\Cref{fig:simple-trace}: the instructions for {\tt main} begin at 0,
and the (uninitialized) stack begins at 100. The entire stack is currently
{\em unsealed}, meaning that it is eligible to be written to.
Directly below the memory sketch, we summarize the parts
of the register file that are relevant at this point in the example.

The rest of \Cref{fig:simple-trace} traces the execution of the
program, showing the machine state at call and return points.
Instructions are intended to be similar to RISC-V, but are described
in a (hopefully more readable) ad-hoc notation rather than assembler
syntax. The $\jal$ (jump-and-link) instruction used for calls atomically
stores the return address in a register and jumps to a specified new $\PCname$;
the $\jalr$ (jump-and-link-via-register) instruction used for returns jumps
to the address stored in a register. Register $r_0$ is hardwired to zero,
so that, e.g., instruction 3 stores $0$ into address 101.

At the call from {\tt main} to {\tt f}, {\tt main}'s locals (as
designated by the stack pointer) are {\em sealed} and labeled with the depth
of their owner, which is 0. Sealing is the fundamental integrity contract
between the caller and callee: {\tt main} expects that {\tt x} and {\tt y}
will be unchanged when it gets control back. This gives us our simple,
intuitive statements of integrity, in their {\em eager} and {\em ultra-eager}
forms, either of which supports a {\em call rule}.

\begin{itemize}
\item {\bf Eager Integrity}: on entry to a subroutine, whichever components are
  sealed will have the same values when the subroutine returns
\item {\bf Ultra-eager Integrity}: at any step, whichever components are sealed
  will be invariant
\item {\bf Call Rule}: from the \(\jal\) instruction of a call, if the step is
  successful and there is eventually a corresponding return, that return is to
  the appropriate return address and all sealed components are unchanged
\end{itemize}

Meanwhile confidentiality intuitively states that the callee, {\tt f}, is unaware
of the context of its call (with the exception, of course, of its arguments).
We state this in terms of noninterference: if we say ``components \(\components\)
interfere with {\tt f} up to its return,'' we mean that at some point up to when it returns,
{\tt f} would display different behavior if called from a state with a different
assignment of values to \(\components\). Here behavior encompasses uses of {\tt print},
return values, and changes to state that will persist beyond the return.
For instance, if {\tt f} computed the offset to access {\tt x} and branched on its value
to decide whether to double or triple its argument, a change in {\tt x} could translate
to a change in {\tt f}'s return value, making {\tt x} interfere with {\tt f}.

Note that it is not sufficient to prevent sealed memory from interfering: unsealed
memory that is, from the callee's perspective, uninitialized is also part of the calling
context. Our model therefore does not support an ultra-eager confidentiality property.
We have only eager confidentiality, which supports a {\em transparency rule}. But the
notion of non-interference also lets us weaken integrity to make a {\em lazy} property
for use with lazy tag policies.

\begin{itemize}
\item {\bf Eager Confidentiality}: the state of the stack on entry to a subroutine
  does not interfere with its behavior up to the point that it returns
\item {\bf Transparency Rule}: from the \(\jal\) instruction of a call, if the step is
  successful, the entry state can be arbitrarily scrambled without changing the callee's
  behavior up to return
\item {\bf Lazy Integrity}:  on entry to a subroutine, whichever components are
  sealed will either have the same values when the subroutine returns, {\em or} they will
  not interfere with the caller's behavior up to its own return
\end{itemize}
